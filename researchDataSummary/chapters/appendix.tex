\setchapterstyle{lines}
\labch{ch:appendix}

\chapter{Source Code}

\section{Supercell Construction}
\labsec{sec:generateSupercell}

\begin{remark}
    This source code is MATLAB format under Version R2019b/R2020a, other version is not tested. The output "POSCAR" file can not be used in VASP directly because this file DO NOT remove duplicated atoms. It should be imported in VESTA, remove duplicated atoms(built-in function) and then export.
\end{remark}

\begin{lstlisting}[language=Matlab]
    %% Determine the supercell length
    clear variables;
    nMax = 40;
    mMax = 40;
    a = 3.3171310374808005;
    cutMaxA = 10* a;
    
    theta1 = zeros(nMax, mMax);
    theta2 = zeros(nMax, mMax);
    theta3 = zeros(nMax, mMax);
    theta4 = zeros(nMax, mMax);
    length1 = zeros(nMax, mMax);
    length2 = zeros(nMax, mMax);
    length3 = zeros(nMax, mMax);
    length4 = zeros(nMax, mMax);
    
    for numIdxN = 1: nMax
        for numIdxM = 1: mMax
            theta1(numIdxN, numIdxM) = 2*atand((2*numIdxN - 1)/(sqrt(3)* (2*numIdxM - 1)));
            theta2(numIdxN, numIdxM) = 2*atand(numIdxN/(sqrt(3)* numIdxM));
            theta3(numIdxN, numIdxM) = 2*atand((sqrt(3)* (2*numIdxM - 1))/(2*numIdxN - 1));
            theta4(numIdxN, numIdxM) = 2*atand((sqrt(3)* numIdxM)/numIdxN);
            length1(numIdxN, numIdxM) = a/2 * sqrt((2*numIdxN - 1)^2 + 3*(2*numIdxM - 1)^2);
            length2(numIdxN, numIdxM) = a* sqrt(numIdxN^2 + 3* numIdxM^2);
        end
    end
    length3 = length1;
    length4 = length2;
    
    [sortedTheta1, sortedIndex1] = sort(theta1(:));
    [sortedn1, sortedm1] = ind2sub(size(theta1), sortedIndex1);
    [sortedTheta2, sortedIndex2] = sort(theta2(:));
    [sortedn2, sortedm2] = ind2sub(size(theta2), sortedIndex2);
    
    % figure();
    % hold on;
    % scatter(sortedTheta1, length1(sortedIndex1), 5, 'b');
    % scatter(sortedTheta2, length2(sortedIndex2), 5, 'r');
    % hold off;
    
    thetaMix = [theta1, theta2, theta3, theta4];
    lengthMix = [length1, length2, length3, length4];
    [sortedThetaMix, sortedIndexMix] = sort(thetaMix(:));
    [sortednMix, sortedmMix] = ind2sub(size(thetaMix), sortedIndexMix);
    while ~prod(sortedmMix <= 20)
        sortedmMix(sortedmMix > 20) = sortedmMix(sortedmMix > 20) - 20;
    end
    sortedMixType = nan(size(thetaMix));
    sortedMixType = 1*(sortedIndexMix(:) <= (numel(thetaMix)/4)) + 2*(sortedIndexMix(:) > (numel(thetaMix)/4) & sortedIndexMix(:) <= (numel(thetaMix)/2)) ...
        + 3* (sortedIndexMix(:) > (numel(thetaMix)/2) & sortedIndexMix(:) <= (numel(thetaMix)*3/4)) + 4*(sortedIndexMix(:) > (numel(thetaMix)*3/4));
    
    % figure;
    % scatter(sortedThetaMix, lengthMix(sortedIndexMix), 5);
    
    [thetaMixUnique, sortedIndexMixUnique] = uniquetol(thetaMix);
    lengthMixUnique = nan(size(thetaMixUnique));
    sortedmMixUnique = nan(size(thetaMixUnique));
    sortednMixUnique = nan(size(thetaMixUnique));
    sortedMixTypeUnique = nan(size(thetaMixUnique));
    for numIdx = 1: length(thetaMixUnique)
        searchIndex = abs(thetaMixUnique(numIdx) -  sortedThetaMix) < 1e-8;
        lengthMixUnique(numIdx) = min(lengthMix(sortedIndexMix(searchIndex)));
        tmpIndex = searchIndex & abs(lengthMix(sortedIndexMix) - lengthMixUnique(numIdx))< 1e-8;
        sortedmMixUnique(numIdx) = sortedmMix(tmpIndex);
        sortednMixUnique(numIdx) = sortednMix(tmpIndex);
        sortedMixTypeUnique(numIdx) = sortedMixType(tmpIndex);
    end
    
    % figure;
    % scatter(thetaMixUnique, lengthMixUnique, 5);
    % plot(thetaMixUnique, lengthMixUnique);
    
    cutIndex = lengthMixUnique < cutMaxA;
    thetaMixUniqueCut = thetaMixUnique(cutIndex);
    lengthMixUniqueCut = lengthMixUnique(cutIndex);
    sortedmMixUniqueCut = sortedmMixUnique(cutIndex);
    sortednMixUniqueCut = sortednMixUnique(cutIndex);
    sortedMixTypeUniqueCut = sortedMixTypeUnique(cutIndex);
    
    figure;
    scatter(thetaMixUniqueCut, lengthMixUniqueCut, 5);
    xlim([0, 60]);
    
    % Clear variables
    clear length1 length2 length3 length4 lengthMix theta1 theta2 theta3 theta4 thetaMix
    clear cutMaxA mMax nMax numIdx numIdxM numIdxN tmpIndex cutIndex
    clear lengthMixUnique searchIndex sortedmMix sortedMixType sortedMixTypeUnique sortedIndexMix sortedmMix sortedmMixUnique
    clear sortedIndexMixUnique sortednMix sortednMixUnique sortedThetaMix thetaMixUnique
    %% Generates the lattice
    % a = 3.441;  % a-axies of the lattice
    b = a;      % b-axies of the lattice
    gamma = 120;% angle of <a, b>
    sizeLattice = 21;  % Scale of the system (should be odd)
    
    % Initialize of the variables
    centerOrder = (sizeLattice + 1) ./ 2;
    lattice.x = zeros(sizeLattice);
    lattice.y = zeros(sizeLattice);
    % Difference in x and y in the primitive cell
    deltaA = [a, 0];
    deltaB = [b*cosd(gamma), b*sind(gamma)];
    % Set the zero of the axies
    lattice.x(sizeLattice, 1) = 0;
    lattice.y(sizeLattice, 1) = 0;
    % Initialize the position of the center
    lattice.x(centerOrder, centerOrder) = ((centerOrder - 1)*deltaA(1) + (sizeLattice - centerOrder)*deltaB(1));
    lattice.y(centerOrder, centerOrder) = ((centerOrder - 1)*deltaA(2) + (sizeLattice - centerOrder)*deltaB(2));
    % Calculate the distance from the center
    for row = sizeLattice: -1: 1
        for column = 1: sizeLattice
            lattice.x(row, column) = ((column - 1)*deltaA(1) + (sizeLattice - row)*deltaB(1));
            lattice.y(row, column) = ((column - 1)*deltaA(2) + (sizeLattice - row)*deltaB(2));
            lattice.hoppingA(row, column) = column - centerOrder;
            lattice.hoppingB(row, column) = centerOrder - row;
        end
    end
    % Move the center to the axis zero
    lattice.x = lattice.x - lattice.x(centerOrder, centerOrder);
    lattice.y = lattice.y - lattice.y(centerOrder, centerOrder);
    
    %% Rotation
    % Construct layers
    d = 1.58106107700766;   % h-phase:1.59498653898723, t-phase:1.58106107700766
    D = 2.809105554065346;  % h-phase:3.67176692979057, t-phase:3.12122839340594
                            % Origin: 3.12122839340594
    vacuumLength = 20;
    numberOfLayers = 2;
    rotationNumIdx = 23;
    % for rotationNumIdx = 13: 13
    rotationDegree = thetaMixUniqueCut(rotationNumIdx)/2;
    supercellLattice = lengthMixUniqueCut(rotationNumIdx);
    supercellBoundx = [0, supercellLattice, supercellLattice/2, -supercellLattice/2, 0];
    supercellBoundy = [0, 0, sqrt(3)/2 * supercellLattice, sqrt(3)/2 * supercellLattice, 0];
    offsetSe1 = [deltaA', deltaB']*[2/3; 1/3];
    % offsetSe1 = [deltaA', deltaB']*[1/3; 2/3];
    offsetSe2 = [deltaA', deltaB']*[1/3; 2/3];
    layer{1}.Se1.x = lattice.x + offsetSe1(1);
    layer{1}.Se1.y = lattice.y + offsetSe1(2);
    layer{1}.Se1.z = zeros(size(lattice.x));
    layer{1}.V.x = lattice.x;
    layer{1}.V.y = lattice.y;
    layer{1}.V.z = zeros(size(lattice.x)) + d;
    layer{1}.Se2.x = lattice.x + offsetSe2(1);
    layer{1}.Se2.y = lattice.y + offsetSe2(2);
    layer{1}.Se2.z = zeros(size(lattice.x)) + 2*d;
    
    layer{2} = layer{1};
    layer{2}.Se1.z = layer{1}.Se1.z + D + 2*d;
    layer{2}.Se2.z = layer{1}.Se2.z + D + 2*d;
    layer{2}.V.z = layer{1}.V.z + D + 2*d;
    
    % Rotation Process
    for i = 1: numberOfLayers
        % layer0: counterclock; layer1: clock
        switch i
            case 1
                rotationMatrix = ...
                    [cosd(rotationDegree), -sind(rotationDegree), 0; sind(rotationDegree), cosd(rotationDegree), 0; 0 0 1];
            case 2
                rotationMatrix = ...
                    [cosd(-rotationDegree), -sind(-rotationDegree), 0; sind(-rotationDegree), cosd(-rotationDegree), 0; 0 0 1];
        end
        switch sortedMixTypeUniqueCut(rotationNumIdx)
            case {1, 2}
                correctMatrix = [cosd(-30), -sind(-30), 0; sind(-30), cosd(-30), 0; 0 0 1];
            case {3, 4}
                correctMatrix = [1 0 0; 0 1 0; 0 0 1];
        end
        for numIdx = 1: numel(layer{i}.V.x)
            tmpMat = [layer{i}.Se1.x(numIdx), layer{i}.Se1.y(numIdx), layer{i}.Se1.z(numIdx)]';
            tmpMat = correctMatrix * rotationMatrix * tmpMat;
            layer{i}.Se1.x(numIdx) = tmpMat(1);
            layer{i}.Se1.y(numIdx) = tmpMat(2);
            layer{i}.Se1.z(numIdx) = tmpMat(3);
            
            tmpMat = [layer{i}.Se2.x(numIdx), layer{i}.Se2.y(numIdx), layer{i}.Se2.z(numIdx)]';
            tmpMat = correctMatrix * rotationMatrix * tmpMat;
            layer{i}.Se2.x(numIdx) = tmpMat(1);
            layer{i}.Se2.y(numIdx) = tmpMat(2);
            layer{i}.Se2.z(numIdx) = tmpMat(3);
            
            tmpMat = [layer{i}.V.x(numIdx), layer{i}.V.y(numIdx), layer{i}.V.z(numIdx)]';
            tmpMat = correctMatrix * rotationMatrix * tmpMat;
            layer{i}.V.x(numIdx) = tmpMat(1);
            layer{i}.V.y(numIdx) = tmpMat(2);
            layer{i}.V.z(numIdx) = tmpMat(3);
        end
        % Cut the supercell
        layer{i}.V.inSupercell = inpolygon(layer{i}.V.x, layer{i}.V.y, supercellBoundx, supercellBoundy);
        layer{i}.Se1.inSupercell = inpolygon(layer{i}.Se1.x, layer{i}.Se1.y, supercellBoundx, supercellBoundy);
        layer{i}.Se2.inSupercell = inpolygon(layer{i}.Se2.x, layer{i}.Se2.y, supercellBoundx, supercellBoundy);
    end
    
    % 
    % fig = figure;
    % hold on;
    % % scatter3(layer{1}.Se1.x(:), layer{1}.Se1.y(:), layer{1}.Se1.z(:), 'green');
    % scatter3(layer{1}.V.x(:), layer{1}.V.y(:), layer{1}.V.z(:), 'red');
    % % scatter3(layer{1}.Se2.x(:), layer{1}.Se2.y(:), layer{1}.Se2.z(:), 'green');
    % % scatter3(layer{2}.Se1.x(:), layer{2}.Se1.y(:), layer{2}.Se1.z(:), 'green');
    % scatter3(layer{2}.V.x(:), layer{2}.V.y(:), layer{2}.V.z(:), 'blue');
    % % scatter3(layer{2}.Se2.x(:), layer{2}.Se2.y(:), layer{2}.Se2.z(:), 'green');
    % hold off
    % maxLim = max(abs([fig.Children.XLim, fig.Children.YLim]));
    % fig.Children.XLim = [-maxLim, maxLim];
    % fig.Children.YLim = [-maxLim, maxLim];
    
    fig = figure;
    hold on;
    % scatter3(layer{1}.Se1.x(:), layer{1}.Se1.y(:), layer{1}.Se1.z(:), 'green');
    scatter3(layer{1}.V.x(layer{1}.V.inSupercell), layer{1}.V.y(layer{1}.V.inSupercell), layer{1}.V.z(layer{1}.V.inSupercell), 'red');
    % scatter3(layer{1}.Se2.x(:), layer{1}.Se2.y(:), layer{1}.Se2.z(:), 'green');
    % scatter3(layer{2}.Se1.x(:), layer{2}.Se1.y(:), layer{2}.Se1.z(:), 'green');
    scatter3(layer{2}.V.x(layer{2}.V.inSupercell), layer{2}.V.y(layer{2}.V.inSupercell), layer{2}.V.z(layer{2}.V.inSupercell), 'blue');
    % scatter3(layer{2}.Se2.x(:), layer{2}.Se2.y(:), layer{2}.Se2.z(:), 'green');
    hold off
    maxLim = max(abs([fig.Children.XLim, fig.Children.YLim]));
    fig.Children.XLim = [-maxLim, maxLim];
    fig.Children.YLim = [-maxLim, maxLim];
    
    % Calculate POSCAR
    supercellC = vacuumLength + 4*d + D;
    VatomNumber = 0;
    SeatomNumber = 0;
    superCell.V.x = layer{1}.V.x(layer{1}.V.inSupercell);
    superCell.V.y = layer{1}.V.y(layer{1}.V.inSupercell);
    superCell.V.z = layer{1}.V.z(layer{1}.V.inSupercell);
    superCell.Se.x = cat(1, layer{1}.Se1.x(layer{1}.Se1.inSupercell), layer{1}.Se2.x(layer{1}.Se2.inSupercell));
    superCell.Se.y = cat(1, layer{1}.Se1.y(layer{1}.Se1.inSupercell), layer{1}.Se2.y(layer{1}.Se2.inSupercell));
    superCell.Se.z = cat(1, layer{1}.Se1.z(layer{1}.Se1.inSupercell), layer{1}.Se2.z(layer{1}.Se2.inSupercell));
    for i = 1: numberOfLayers
        VatomNumber = VatomNumber + length(layer{i}.V.x(layer{i}.V.inSupercell));
        SeatomNumber = SeatomNumber + length(layer{i}.Se1.x(layer{i}.Se1.inSupercell)) + length(layer{i}.Se2.x(layer{i}.Se2.inSupercell));
        if i >= 2
            superCell.V.x = cat(1, superCell.V.x, layer{i}.V.x(layer{i}.V.inSupercell));
            superCell.V.y = cat(1, superCell.V.y, layer{i}.V.y(layer{i}.V.inSupercell));
            superCell.V.z = cat(1, superCell.V.z, layer{i}.V.z(layer{i}.V.inSupercell));
            superCell.Se.x = cat(1, superCell.Se.x, layer{i}.Se1.x(layer{i}.Se1.inSupercell), layer{i}.Se2.x(layer{i}.Se2.inSupercell));
            superCell.Se.y = cat(1, superCell.Se.y, layer{i}.Se1.y(layer{i}.Se1.inSupercell), layer{i}.Se2.y(layer{i}.Se2.inSupercell));
            superCell.Se.z = cat(1, superCell.Se.z, layer{i}.Se1.z(layer{i}.Se1.inSupercell), layer{i}.Se2.z(layer{i}.Se2.inSupercell));
        end
    end
    % Write POSCAR
    fileId = fopen("POSCAR", 'w');
    fprintf(fileId, "Degree: %.5f\n", thetaMixUniqueCut(rotationNumIdx));
    fprintf(fileId, "   1.00000000000000 \n");
    fprintf(fileId, "%24.15f %24.15f %24.15f\n", [supercellLattice, 0, 0]);
    fprintf(fileId, "%24.15f %24.15f %24.15f\n", [-supercellLattice/2, sqrt(3)/2*supercellLattice, 0]);
    fprintf(fileId, "%24.15f %24.15f %24.15f\n", [0, 0, supercellC]);
    fprintf(fileId, "   V    Se\n");
    fprintf(fileId, "%6d %6d\n", [VatomNumber, SeatomNumber]);
    fprintf(fileId, "Cartesian\n");
    for numIdx = 1: VatomNumber
        fprintf(fileId, "% 18.16f % 21.16f % 21.16f\n", [superCell.V.x(numIdx), superCell.V.y(numIdx), superCell.V.z(numIdx)]);
    end
    
    for numIdx = 1: SeatomNumber
        fprintf(fileId, "% 18.16f % 21.16f % 21.16f\n", [superCell.Se.x(numIdx), superCell.Se.y(numIdx), superCell.Se.z(numIdx)]);
    end
    fclose(fileId);
    % end    
\end{lstlisting}

\section{Primitive cell}
\labsec{sec:primitiveCell}


\begin{lstlisting}[style=kaolstplain, linewidth=1.5\textwidth]
V1 Se2 T-Phase Structure
    1.00000000000000
      3.3171310374808005    0.0000000000021438   -0.0000000000000300
     -1.6585655187866546    2.8727197461676814    0.0000000000000371
     -0.0000000000002724    0.0000000000001877   27.2846672137727637
    V    Se
      2     4
 Direct
  -0.0000000000000002  0.0000000000000003  0.0456342266658590       
   0.0000000000000002 -0.0000000000000003  0.2751370207888941       
   0.6666666746654853  0.3333333495243738 -0.0126915455264852       
   0.3333333497015403  0.6666666993837396  0.1031743004030291       
   0.6666666746654855  0.3333333495243735  0.2175692596287150       
   0.3333333497015404  0.6666666993837395  0.3334908956017944       
 
   0.00000000E+00  0.00000000E+00  0.00000000E+00
   0.00000000E+00  0.00000000E+00  0.00000000E+00
   0.00000000E+00  0.00000000E+00  0.00000000E+00
   0.00000000E+00  0.00000000E+00  0.00000000E+00
   0.00000000E+00  0.00000000E+00  0.00000000E+00
   0.00000000E+00  0.00000000E+00  0.00000000E+00
\end{lstlisting}