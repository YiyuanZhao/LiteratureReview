%----------------------------------------------------------------------------------------
%	EXAMPLE AND DOCUMENTATION OF THE KAOBOOK CLASS
%----------------------------------------------------------------------------------------

\documentclass[
    a4paper, % Page size
    fontsize=10pt, % Base font size
    twoside=true, % Use different layouts for even and odd pages (in particular, if twoside=true, the margin column will be always on the outside)
	%open=any, % If twoside=true, uncomment this to force new chapters to start on any page, not only on right (odd) pages
	%chapterentrydots=true, % Uncomment to output dots from the chapter name to the page number in the table of contents
	numbers=noenddot, % Comment to output dots after chapter numbers; the most common values for this option are: enddot, noenddot and auto (see the KOMAScript documentation for an in-depth explanation)
]{kaobook}

%----------------------------------------------------------------------------------------
%	PACKAGES AND OTHER DOCUMENT CONFIGURATIONS
%----------------------------------------------------------------------------------------

% Choose the language
\ifxetexorluatex
	\usepackage{polyglossia}
	\setmainlanguage{english}
\else
	\usepackage[english]{babel} % Load characters and hyphenation
\fi
\usepackage[english=british]{csquotes}	% English quotes

% Load packages for testing
\usepackage{blindtext}
%\usepackage{showframe} % Uncomment to show boxes around the text area, margin, header and footer
%\usepackage{showlabels} % Uncomment to output the content of \label commands to the document where they are used

% Load the bibliography package
\usepackage{kaobiblio}
\addbibresource{main.bib} % Bibliography file

% Load mathematical packages for theorems and related environments
\usepackage[framed=true]{kaotheorems}

% Load the package for hyperreferences
\usepackage{kaorefs}
% Custumized package
\usepackage{subfigure}
\graphicspath{{images/}} % Paths in which to look for images

\makeindex[columns=3, title=Alphabetical Index, intoc] % Make LaTeX produce the files required to compile the index

\makeglossaries % Make LaTeX produce the files required to compile the glossary
\newglossaryentry{computer}{
	name=computer,
	description={is a programmable machine that receives input, stores and manipulates data, and provides output in a useful format}
}

% Glossary entries (used in text with e.g. \acrfull{fpsLabel} or \acrshort{fpsLabel})
\newacronym[longplural={Frames per Second}]{fpsLabel}{FPS}{Frame per Second}
\newacronym[longplural={Tables of Contents}]{tocLabel}{TOC}{Table of Contents}

 % Include the glossary definitions

\makenomenclature % Make LaTeX produce the files required to compile the nomenclature

% Reset sidenote counter at chapters
%\counterwithin*{sidenote}{chapter}

%----------------------------------------------------------------------------------------

\begin{document}

%----------------------------------------------------------------------------------------
%	BOOK INFORMATION
%----------------------------------------------------------------------------------------

\titlehead{Research Summary}
% \subject{Use this document as a template}

\title[VSe2 DFT calculation and TB-monolayer model construction]{VSe$_2$ DFT calculation and\\ TB-monolayer model construction}
\subtitle{A summary for current research}

\author[Yiyuan Zhao]{Yiyuan Zhao}

\date{\today}

\publishers{SCES, Tongji University}

%----------------------------------------------------------------------------------------

\frontmatter % Denotes the start of the pre-document content, uses roman numerals

%----------------------------------------------------------------------------------------
%	OPENING PAGE
%----------------------------------------------------------------------------------------

%\makeatletter
%\extratitle{
%	% In the title page, the title is vspaced by 9.5\baselineskip
%	\vspace*{9\baselineskip}
%	\vspace*{\parskip}
%	\begin{center}
%		% In the title page, \huge is set after the komafont for title
%		\usekomafont{title}\huge\@title
%	\end{center}
%}
%\makeatother

%----------------------------------------------------------------------------------------
%	COPYRIGHT PAGE
%----------------------------------------------------------------------------------------

% \makeatletter
% \uppertitleback{\@titlehead} % Header

% \lowertitleback{
% 	\textbf{Disclaimer}\\
% 	You can edit this page to suit your needs. For instance, here we have a no copyright statement, a colophon and some other information. This page is based on the corresponding page of Ken Arroyo Ohori's thesis, with minimal changes.
	
% 	\medskip
	
% 	\textbf{No copyright}\\
% 	\cczero\ This book is released into the public domain using the CC0 code. To the extent possible under law, I waive all copyright and related or neighbouring rights to this work.
	
% 	To view a copy of the CC0 code, visit: \\\url{http://creativecommons.org/publicdomain/zero/1.0/}
	
% 	\medskip
	
% 	\textbf{Colophon} \\
% 	This document was typeset with the help of \href{https://sourceforge.net/projects/koma-script/}{\KOMAScript} and \href{https://www.latex-project.org/}{\LaTeX} using the \href{https://github.com/fmarotta/kaobook/}{kaobook} class.
	
% 	The source code of this book is available at:\\\url{https://github.com/fmarotta/kaobook}
	
% 	(You are welcome to contribute!)
	
% 	\medskip
	
% 	\textbf{Publisher} \\
% 	First printed in May 2019 by \@publishers
% }
% \makeatother

%----------------------------------------------------------------------------------------
%	DEDICATION
%----------------------------------------------------------------------------------------

% \dedication{
% 	The harmony of the world is made manifest in Form and Number, and the heart and soul and all the poetry of Natural Philosophy are embodied in the concept of mathematical beauty.\\
% 	\flushright -- D'Arcy Wentworth Thompson
% }

%----------------------------------------------------------------------------------------
%	OUTPUT TITLE PAGE AND PREVIOUS
%----------------------------------------------------------------------------------------

% Note that \maketitle outputs the pages before here

\maketitle

%----------------------------------------------------------------------------------------
%	PREFACE
%----------------------------------------------------------------------------------------

% \input{chapters/preface.tex}
% \index{preface}

%----------------------------------------------------------------------------------------
%	TABLE OF CONTENTS & LIST OF FIGURES/TABLES
%----------------------------------------------------------------------------------------

\begingroup % Local scope for the following commands

% Define the style for the TOC, LOF, and LOT
%\setstretch{1} % Uncomment to modify line spacing in the ToC
%\hypersetup{linkcolor=blue} % Uncomment to set the colour of links in the ToC
\setlength{\textheight}{230\hscale} % Manually adjust the height of the ToC pages

% Turn on compatibility mode for the etoc package
\etocstandarddisplaystyle % "toc display" as if etoc was not loaded
\etocstandardlines % "toc lines as if etoc was not loaded

\tableofcontents % Output the table of contents

\listoffigures % Output the list of figures

% Comment both of the following lines to have the LOF and the LOT on different pages
\let\cleardoublepage\bigskip
\let\clearpage\bigskip

\listoftables % Output the list of tables

\endgroup

%----------------------------------------------------------------------------------------
%	MAIN BODY
%----------------------------------------------------------------------------------------

\mainmatter % Denotes the start of the main document content, resets page numbering and uses arabic numbers
\setchapterstyle{kao} % Choose the default chapter heading style

% \setchapterpreamble[u]{\margintoc}
\chapter{Introduction}
\labch{intro}

\section{The Main Ideas}

Many modern printed textbooks have adopted a layout with prominent 
margins where small figures, tables, remarks and just about everything 
else can be displayed. Arguably, this layout helps to organise the 
	discussion by separating the main text from the ancillary material, 
	which at the same time is very close to the point in the text where 
	it is referenced.

This document does not aim to be an apology of wide margins, for there 
are many better suited authors for this task; the purpose of all these 
words is just to fill the space so that the reader can see how a book 
written with the kaobook class looks like. Meanwhile, I shall also try 
to illustrate the features of the class.

The main ideas behind kaobook come from this 
\href{https://3d.bk.tudelft.nl/ken/en/2016/04/17/a-1.5-column-layout-in-latex.html}{blog 
	post}, and actually the name of the class is dedicated to the author 
of the post, Ken Arroyo Ohori, which has kindly allowed me to create a 
class based on his thesis. Therefore, if you want to know more reasons 
to prefer a 1.5-column layout for your books, be sure to read his blog 
post.

Another source of inspiration, as you may have noticed, is the 
\href{https://github.com/Tufte-LaTeX/tufte-latex}{Tufte-Latex Class}. 
The fact that the design is similar is due to the fact that it is very 
difficult to improve something which is already so good. However, I like 
to think that this class is more flexible than Tufte-Latex. For 
instance, I have tried to use only standard packages and to implement as 
little as possible from scratch;\sidenote{This also means that 
understanding and contributing to the class development is made easier. 
Indeed, many things still need to be improved, so if you are interested, 
check out the repository on github!} therefore, it should be pretty easy 
to customise anything, provided that you read the documentation of the 
package that provides that feature.

In this book I shall illustrate the main features of the class and 
provide information about how to use and change things. Let us get 
started.

\section{What This Class Does}
\labsec{does}

The \Class{kaobook} class focuses more about the document structure than 
about the style. Indeed, it is a well-known \LaTeX\xspace principle that 
structure and style should be separated as much as possible (see also 
\vrefsec{doesnot}). This means that this class will only provide 
commands, environments and in general, the opportunity to do things, 
which the user may or may not use. Actually, some stylistic matters are 
embedded in the class, but the user is able to customise them with ease.

The main features are the following:

\begin{description}
	\item[Page Layout] The text width is reduced to improve readability 
	and make space for the margins, where any sort of elements can be 
	displayed.
	\item[Chapter Headings] As opposed to Tufte-Latex, we provide a 
	variety of chapter headings among which to choose; examples will be 
	seen in later chapters.
	\item[Page Headers] They span the whole page, margins included, and, 
	in twoside mode, display alternatively the chapter and the section 
	name.\sidenote[][-2mm]{This is another departure from Tufte's 
	design.}
	\item[Matters] The commands \Command{frontmatter}, 
	\Command{mainmatter} and \Command{backmatter} have been redefined in 
	order to have automatically wide margins in the main matter, and 
	narrow margins in the front and back matters. However, the page 
	style can be changed at any moment, even in the middle of the 
	document.
	\item[Margin text] We provide commands \Command{sidenote} and 
	\Command{marginnote} to put text in the 
	margins.\sidenote[][-2mm]{Sidenotes (like this!) are numbered while 
	marginnotes are not}
	\item[Margin figs/tabs] A couple of useful environments is 
	\Environment{marginfigure} and \Environment{margintable}, which, not 
	surprisingly, allow you to put figures and tables in the margins 
	(\cfr \reffig{marginmonalisa}).
	\item[Margin toc] Finally, since we have wide margins, why don't add 
	a little table of contents in them? See \Command{margintoc} for 
	that.
	\item[Hyperref] \Package{hyperref} is loaded and by default we try 
	to add bookmarks in a sensible way; in particular, the bookmarks 
	levels are automatically reset at \Command{appendix} and 
	\Command{backmatter}. Moreover, we also provide a small package to 
	ease the hyperreferencing of other parts of the text.
	\item[Bibliography] We want the reader to be able to know what has 
	been cited without having to go to the end of the document every 
	time, so citations go in the margins as well as at the end, as in 
	Tufte-Latex. Unlike that class, however, you are free to customise 
	the citations as you wish.
\end{description}

\begin{marginfigure}[-5.5cm]
	\includegraphics{monalisa}
	\caption[The Mona Lisa]{The Mona Lisa.\\ 
	\url{https://commons.wikimedia.org/wiki/File:Mona_Lisa,_by_Leonardo_da_Vinci,_from_C2RMF_retouched.jpg}}
	\labfig{marginmonalisa}
\end{marginfigure}

The order of the title pages, table of contents and preface can be 
easily changed, as in any \LaTeX\ document. In addition, the class is 
based on \KOMAScript's \Class{scrbook}, therefore it inherits all the 
goodies of that.

\section{What This Class Does Not Do}
\labsec{doesnot}

As anticipated, further customisation of the book is left to the user. 
Indeed, every book may have sidenotes, margin figures and so on, but 
each book will have its own fonts, toc style, special environments and 
so on. For this reason, in addition to the class, we provide only 
sensible defaults, but if these features are not needed, they can be 
left out. These special packages are located in the \Path{style} 
directory, which is organised as follows:

\begin{description}
	\item[kao.sty] This package contains the most important definitions 
	of macros and specifications of page layout. It is the heart of the 
	\Class{kaobook}.
	\item[kaobiblio.sty] Contains commands to add citations and 
	customise the bibliography.
	\item[packages.sty] Loads additional packages to decorate the 
	writing with special contents (for instance, the \Package{listing} 
	package is loaded here as it is not required in every book). There 
	are also defined some useful commands to print the same words always 
	in the same way, \eg latin words in italics or \Package{packages} in 
	verbatim.
	\item[kaorefs.sty] Some useful commands to manage labeling and 
	referencing, again to ensure that the same elements are referenced 
	always in a consistent way.
	\item[environments.sty] Provides special environments, like boxes. 
	Both simple and complex environments are available; by complex we 
	mean that they are endowed with a counter, floating and can be put 
	in a special table of contents.\sidenote[][-2mm]{See 
	\vrefch{mathematics} for some examples.}
	\item[theorems.sty] The style of mathematical environments. 
	Actually, there are two such packages: one is for plain theorems,
	\ie the theorems are printed in plain text; the other uses 
	\Package{mdframed} to draw a box around theorems. You can plug the 
	most appropriate style into its document.
\end{description}

\marginnote[2mm]{The audacious users might feel tempted to edit some of 
these packages. I'd be immensely happy if they sent me examples of what 
they have been able to do!}

In the rest of the book, I shall assume that the reader is not a novice 
in the use of \LaTeX, and refer to the documentation of the packages 
used in this class for things that are already explained there. 
Moreover, I assume that the reader is willing to make minor edits to the 
provided packages for styles, environments and commands, if he or she 
does not like the default settings.

\section{How to Use This Class}

Either if you are using the template from 
\href{http://latextemplates.org/template/kaobook}{latextemplates}, or if 
you cloned the GitHub 
\href{https://www.github.com/fmarotta/kaobook}{repository}, there are 
infinite ways to use the \Class{kaobook} class in practice, but we will 
discuss only two of them. The first is to find the \Path{main.tex} file 
which I used to write this book, and edit it; this will probably involve 
a lot of text-deleting, copying-and-pasting, and rewriting. The second 
way is to start almost from scratch and use the \Path{skeleton.tex} 
file, which is a cleaned-up version of the \Path{main.tex}; even if you 
choose the second way, you may find it useful to draw inspiration from 
the \Path{main.tex} file.

To compile the document, assuming that its name is \Path{main.tex}, you 
will have to run the following sequence of commands:

\begin{lstlisting}[style=kaolstplain,linewidth=1.5\textwidth]
pdflatex main # Compile template
makeindex main.nlo -s nomencl.ist -o main.nls # Compile nomenclature
makeindex main # Compile index
biber main # Compile bibliography
makeglossaries main # Compile glossary
pdflatex main # Compile template again
pdflatex main # Compile template again
\end{lstlisting}

You may need to compile the template some more times in order for some 
errors to disappear. For any support requests, please ask a question on 
\url{tex.stackexchange.org} with the tag \enquote{kaobook}, open an 
issue on GitHub, or contact the author via e-mail.


\pagelayout{wide} % No margins
\addpart{Bilayer Twisted vanadium diselenide}
\pagelayout{margin} % Restore margins

\setchapterpreamble[u]{\margintoc}
\chapter{Crystal structure in twisted bilayer vanadium diselenide}
\labch{Crystal structure in twisted bilayer vanadium diselenide}

\section{Supercell Determination}
\labsec{sec:Supercell Determination}

Supercell is constructed by staking different V atom in primitive cell into the same place on horizontal position(aligned with c-axis). The periodical structure is divided into meshes which marks every V atom. The c-axis aligned staking (AA staking) will be achived under rotation operation. Thus constructs a supercell with a given lattice parameter $L$.(See in \reffig{fig:case1_2}).

We denotes $(m,n)$ \sidenote{$(m,n)$ is divided into two diffent rotation type, marked as $(m,n)$(Rotation Case 1) and $(m^{'}, n^{'})$(Rotation Case 2) seperately.} pair to describe the selected V atoms that to be stacked into AA type. The top layer rotates $\theta/2$ clockwise while the bottom layer rotates $\theta/2$ counterclockwise.Varities of pairs can be chosen to gain the supercell structure, each pair gives a rotaion degree $\theta$ and supercell lattice parameter $L$, which is the feature to construct a supercell we need.
\begin{figure}[ht]
	\includegraphics{case1_2}
	\caption[Supercell mesh in case 1-2]{
		Supercell mesh in case 1-2. m(n) labels horizontal(vertical) V atoms in bilayer periodical lattice on the top view. The blue line indicates two V atoms awaiting to be rotated(One from the bottom layer and another from the top layer). These two V atoms will be AA staked after rotation. Lattice parameter of this supercell is defined as the length of blue line.}
	\labfig{fig:case1_2}
\end{figure}

\begin{figure}[ht]
	\includegraphics{case3_4}
	\caption[Supercell mesh in case 3-4]{
		Supercell mesh in case 3-4. m(n) labels horizontal(vertical) V atoms in bilayer periodical lattice on the top view. The blue line indicates two V atoms awaiting to be rotated(One from the bottom layer and another from the top layer). These two V atoms will be AA staked after rotation. Lattice parameter of this supercell is defined as the length of blue line.}
	\labfig{fig:case3_4}
\end{figure}

The rotation in \reffig{fig:case1_2} doesn't include every possible supercell configuration. (\eg the rotation $(1, 2)$ rotaion in \reffig{fig:case3_4} can not be achived with y-axis symmetry.) An extra x-axis rotation axis is needed for completeness.

Considering these rotation method, we would be able to gain the rotation degree $\theta$ and lattice parameter $a$ in each $(m,n)$ condition. For Case 1-2:
\begin{eqnarray}
	\tan \frac{\theta}{2} &=& \frac{\frac{a}{2}(2n - 1)}{\frac{\sqrt{3}a}{2}(2m - 1)} = \frac{2n - 1}{\sqrt{3} (2m - 1)} \label{eqn:thetaCase1} \\
	L &=& \frac{a}{2} \sqrt{(2n - 1)^2 + 3(2m - 1)^2} \label{eqn:lengthCase1}\\
	\tan \frac{\theta^{'}}{2} &=& \frac{na}{\sqrt{3}ma} = \frac{n}{\sqrt{3}m}\label{eqn:thetaCase2} \\
	L^{'} &=& a\sqrt{3m^2 + n^2} \label{eqn:lengthCase2}
\end{eqnarray}

The same methods can be applied to Case 3-4:
\begin{eqnarray}
	\tan \frac{\theta}{2} &=& \frac{\frac{\sqrt{3}a}{2}(2m - 1)}{\frac{a}{2}(2n - 1)} = \frac{\sqrt{3}(2m - 1)}{2n - 1} \\
	L &=& \frac{a}{2} \sqrt{(2n - 1)^2 + 3(2m - 1)^2} \label{eqn:lengthCase3}\\
	\tan \frac{\theta^{'}}{2} &=& \frac{\sqrt{3}ma}{na} = \frac{\sqrt{3}m}{n} \label{eqn:thetaCase4} \\
	L^{'} &=& a\sqrt{3m^2 + n^2} \label{eqn:lengthCase4}
\end{eqnarray}

Specific $\theta, L$ has been determined np to now. However Varities of $(m,n)$ may be responsible for the same rotation degree.(See in \reffig{fig:ConfigureRepetition}). The smallest supercell lattice parameter at each rotation degree is what we concerned about.(\ie The 'primitive' supercell). The filtered supercell lattice(preserve only the smallest lattice parameter at each rotation degree) shows at \reffig{fig:ConfigureRepetitionFiltered}, which shouws periodicity with 30 degree. This 30 degree period indicates that exploring properties at 0~30° would give a complete description for all rotation degree if we only consider V atoms.

\begin{marginfigure}[]
	\includegraphics{ConfigureRepetition}
	\caption[Supercell lattice parameter from different $(m,n)$ rotation in Case 1-2.]{
		Supercell lattice parameter from different $(m,n)$ rotation in Case 1-2. Y-axis describes superlattice in unit cell length($a_0$), X-axis describes the rotation degree $\theta$.It can be found that multiple superlattice length corresponds to one specific rotation degree.
	}
	\labfig{fig:ConfigureRepetition}
\end{marginfigure}

\begin{marginfigure}[]
	\includegraphics{ConfigureRepetitionFiltered}
	\caption[Filtered supercell lattice parameter from different $(m,n)$ rotation for all cases(Case 1-4).]{
		Filtered supercell lattice parameter from different $(m,n)$ rotation for all cases(Case 1-4). Y-axis describes superlattice in Angstrom, X-axis describes the rotation degree $\theta$. The superlattice shows periodicity with 30 degree.
	}
	\labfig{fig:ConfigureRepetitionFiltered}
\end{marginfigure}

Once we completed filtering, we gained unique $\theta, L$ at each rotaion degree.It is possible to construct the supercell with given parameter from bilayer primitive cell. Rotation of the top/bottom layer will give rise to twisted bilayers stucture. The last step is cutting twisted bilayer into supercell. The detailed MATLAB code is posted in \refsec{sec:generateSupercell}.

In summary, we need 4 steps to gain the feature of a supercell via rotation:
\begin{description}
	\item[Gain suppercell feature] Obtain the supercell $\theta, L$ from eqn(\ref{eqn:thetaCase1})-(\ref{eqn:lengthCase4}).
	\item[Filter the configuration] Find the minimum supercell lattice length in each rotation degree.
	\item[Construct supercell] Construct the supercell with known $\theta, L$.
	\item[Cut supercell] Cut the layers into supercell with constructed configuration and lattice parameter $L$.
\end{description}

\section{Supercell Analysis}

VSe$_2$ is triangular lattice, which hints that three-fold symmetry is always satisfied. We select $\theta \in [0, 60]$° as our range of analysis, which fulfills the completeness of this issue. We lists all possible configurations constructed in this research in \reftab{tab:supercell}. Relation between the number of atoms and rotation degree is ploted in \reffig{fig:ConfigurationRepetitionPlot}.

\begin{table}[ht]
	\caption[Supercell lattice parameter length]{
		Supercell lattice parameter length with rotation degree/atom number.
	}
	\labtab{tab:supercell}
	\begin{tabular}{lll}
	\toprule
	Rotation Degree/° & Supercell Length/A & Atom Number \\
	\midrule
	6.008983          & 32.82505           & 546         \\
	7.340993          & 26.87507           & 366         \\
	9.430008          & 20.93079           & 222         \\
	10.41744          & 32.82505           & 546         \\
	11.63505          & 29.39992           & 438         \\
	13.17355          & 14.99897           & 114         \\
	15.17818          & 22.56415           & 258         \\
	16.42642          & 24.087             & 294         \\
	17.89655          & 19.15868           & 186         \\
	21.78679          & 9.10403            & 42          \\
	24.4327           & 28.1658            & 402         \\
	26.00782          & 30.58428           & 474         \\
	27.79577          & 12.4067            & 78          \\
	29.40931          & 33.88992           & 582         \\
	30.59069          & 33.88992           & 582         \\
	32.20423          & 12.4067            & 78          \\
	33.99218          & 30.58428           & 474         \\
	35.5673           & 28.1658            & 402         \\
	38.21321          & 9.10403            & 42          \\
	42.10345          & 19.15868           & 186         \\
	43.57358          & 24.087             & 294         \\
	44.82182          & 22.56415           & 258         \\
	46.82645          & 14.99897           & 114         \\
	48.36495          & 29.39992           & 438         \\
	49.58256          & 32.82505           & 546         \\
	50.56999          & 20.93079           & 222         \\
	52.65901          & 26.87507           & 366         \\
	53.99102          & 32.82505           & 546         \\
	60                & 3.441              & 6           \\
	\bottomrule
	\end{tabular}
	\end{table}

	\begin{kaobox}[frametitle=Notes]
		\reftab{tab:supercell} and \reffig{fig:ConfigurationRepetitionPlot} filters supercell lattice parameter $L \le 10 a_0$, which indicates a smaller supercell.
	\end{kaobox}

	\begin{marginfigure}[]
		\includegraphics{ConfigurationRepetitionPlot}
		\caption[Each roation degree and its corresponding number of atoms]{
			Each roation degree(x-axis) and its corresponding number of atoms(y-axis) presented in \reftab{tab:supercell}.
		}
		\labfig{fig:ConfigurationRepetitionPlot}
	\end{marginfigure}

From \reffig{fig:ConfigureRepetitionFiltered} we would find that rotation configuration usually contains hundreds of atoms in a supercell. The scaling of supercell hinders our systematic DFT exploration. VSe$_2$ primitive cell holds six-fold rotation symmetry, which gives 60° periodicity. In practice, the shape of supercell holds 30° periodicity(See in \reffig{fig:ConfigurationRepetitionPlot}) while the exact lattice holds 60° symmetry because of the presence of Se atoms.(\ie The presence of Se atoms decreases the symmetry of supercell). We select configurations with total number of atoms between 42-186 to proceed DFT calculation after balancing variaty and efficiency. The configuration used in DFT calculation lists in \reftab{tab:supercellSelection}.

The supercell lattice length is determined by DFT structure optimization. Details will be discussed in \refsec{sec:Primitive_cell_struture}. Structure with rotation degree 60° exhibits 1T structure, but with different staking method. Bilayer VSe$_2$ is AA-stacked in experiment data.

\begin{table}[h]
	\caption[Filtered supercell structure that will be explored in DFT calculation]{
		Filtered supercell structure that will be explored in DFT calculation.
	}
	\labtab{tab:supercellSelection}
	\begin{tabular}{lll}
	\toprule
	Rotation Degree/° & Supercell Length/A & Atom Number \\
	\midrule
	13.174            & 14.999             & 114         \\
	21.787            & 9.104              & 42          \\
	27.796            & 12.407             & 78          \\
	32.204            & 12.407             & 78          \\
	38.213            & 9.104              & 42          \\
	46.826            & 14.999             & 114         \\
	\bottomrule
	\end{tabular}
\end{table}


\section{Primitive Cell Struture}
\labsec{sec:Primitive_Cell_Struture}

We use primitive cell to construct DFT supercell in the \refsec{sec:Supercell Determination}. Optimizing twisted bilayer VSe$_2$ supercell via DFT is time comsuming, while the optimized structure may lose its symmetry, which leads to incorrect physical meaning. Thus we optimize structure of the bilayer primitive cell at first, then build supercell by total relaxed primitive cell directly.

\begin{figure}[]
	\includegraphics{primitiveCell}
	\caption[Untwisted AA staking bilayer VSe$_2$ primitive cell.]{
		Untwisted AA staking bilayer VSe$_2$ primitive cell. We assume this structure is free standing. \ie Layers are grown without base material( \eg HOPE, MoS$_2$ etc.). $D$ denotes distance between layers and $d$ denotes vertical distance between V and Se atom.
	}
	\labfig{fig:primitiveCell}
\end{figure}

\reffig{fig:primitiveCell} shows structure of the bilayer primitive cell. Fractional coordination of V, Se atoms in XY plane are fixed by its space group symmetry(1T structure configuration). Thus lattice parameter$a_0$, layers spacing $D$ and V-Se vertical spacing $d$ fully determines bilayer structure. These parameters are gained by a DFT structure optimization with non-magnetic 1T bilayer configuration. Hubbard-U is set to 0, tense convergence criteria is set as EDIFFG=-1.0E-3, energy convergence criteria is set as EDIFF=1.0E-8. The fully relaxed structure is given in \reftab{tab:primitiveCellStructure}. Constructed POSCAR file can be seen in \refsec{sec:primitiveCell}. The thickness of the film is $~ 6.28$ Å. 

\begin{figure}[ht]
	\caption[VSe$_2$ configuration in experiment data]{
		VSe$_2$ configuration in experiment data.
	}
	\labfig{fig:experimentConfig}
	\subfigure[Structure in Ref \cite{C9NR06076F}]{
		\includegraphics[width=0.42\linewidth]{thicknessRef1}
	}
	\subfigure[Structure in Ref \cite{adma.201903779}]{
		\includegraphics[width=0.42\linewidth]{thicknessRef2}
	}
	\subfigure[Structure in Ref \cite{PhysRevMaterials.4.084002}]{
		\includegraphics[width=0.42\linewidth]{thicknessRef3}
	}
	\subfigure[Structure in Ref \cite{C8NR09258C}]{
		\includegraphics[width=0.42\linewidth]{thicknessRef5}
	}
\end{figure}

\begin{marginfigure}
	\includegraphics{thicknessRef4}
	\caption[Structure in Ref \cite{PhysRevB.102.115149}]{
		Structure in Ref \cite{PhysRevB.102.115149}.
	}
\end{marginfigure}

For varifying the reliability of our structure, we compare our structure data (lattice parmater $a$, film thickness) with other works. Discovery of structure \sidecite{C9NR06076F} gives thickness with 6.8 Å. Chemically exfoliated VSe$_2$ monolayers research \sidecite{adma.201903779} gives layerthickness of $1.0 \pm 0.1$ nm and $a = 3.35 \pm 0.01$ Å. MBE growth from ML up to 30ML \sidecite{PhysRevMaterials.4.084002} gives the thickness of ~18.3 nm(30MLs) and $a = 3.356$ Å. Research on electronic and magnetic properties of mono- and bilayer VSe$_2$ \sidecite{PhysRevB.102.115149} gives layer thickness $0.6$ nm and $a = 0.34$ nm. Atomistic real-space observation of the van der Waals layered structure of VSe$_2$ for the first time \sidecite{C8NR09258C} gives the thickness\sidenote{Note that the total thickness of the film is 100 nm} of $6.27 \pm 0.3$ Å and $a = 3.26 \pm 0.2$ Å .

A summary of these experiment data shows in \reftab{tab:experimentData}. The total relaxed structure consists with current experiment data in the lattice constant and ML(Monolayer) layer thickness. Once we determined structure of primitive cell, we' re able to expand the primitive cell to gain the supercell structure using methods in \refsec{sec:Supercell Determination}.

\begin{margintable}
	\caption[Structure of the primitive cell and its lattice properties considering space group symmetry]{
		Structure of the primitive cell and its lattice properties considering space group symmetry.
	}
	\labtab{tab:primitiveCellStructure}
	\begin{tabular}{ll}
	\toprule
	Property & Value (Å)\\
	\midrule
	$a$ 	& 3.317131 \\
	$D$ 	& 3.121228 \\
	$d$ 	& 1.581061 \\
	$D + 2d$& 6.283351 \\
	\bottomrule
	\end{tabular}
\end{margintable}

\begin{table}
	\caption[Structure information about VSe$_2$ extracted from experiment data]{
		Structure information about VSe$_2$(Lattice constant and layer thickness) extracted from experiment data. Our DFT calculation parameter shows in the first line.
	}
	\labtab{tab:experimentData}
	\begin{tabular}{lll}
	\toprule
	Source                          & Lattice Constant & Layer Thickness \\
	\midrule
	Our DFT Calculation             & 3.317 Å          & 6.283 Å         \\
	Duvjir et. al, Nanoscale, 2019  & /                & 6.8 Å           \\
	Wei Yu et. al, Nanoscale, 2019  & 3.35 ± 0.01 Å    & 1.0 ± 0.1 nm    \\
	Phys. Rev. Materials, 2020(07)  & 3.356 Å          & ~1ML(6.1 Å)     \\
	Phys. Rev. B 102, 115149 (2020) & 0.34 nm          & 0.6 nm          \\
	Nanoscale, 2019(02)             & 3.26±0.2 Å       & 6.27±0.3 Å  	 \\
	\bottomrule
	\end{tabular}
\end{table}

\chapter{DFT calculation of twisted bilayer vanadium diselenide}
\labch{DFT calculation of twisted bilayer vanadium diselenide}

\section{Primitive Cell Testing}
\labsec{Primitive Cell Testing}

DFT calculation works with varities of control parameters, which each parameter may play crucial role in calculation. Thus, several tests should be applied to enhance strength of our reliability. References \sidecite{PhysRevB.96.235147} tested the effect of pseusopotential in relaxed lattice parameter. Here we tests the exchange-correlation type, pseudopotential, vdW correlation and Hubbard-U in our primitive cell configuration to evaluate their effects in DFT calculation.

\subsection{Hubbard-U term}
Hubbard-U

% \input{chapters/textnotes.tex}
% \input{chapters/figsntabs.tex}
% \input{chapters/references.tex}

% \pagelayout{wide} % No margins
% \addpart{Design and Additional Features}
% \pagelayout{margin} % Restore margins

% \input{chapters/layout.tex}
% \input{chapters/mathematics.tex}

\appendix % From here onwards, chapters are numbered with letters, as is the appendix convention

\pagelayout{wide} % No margins
\addpart{Appendix}
\pagelayout{margin} % Restore margins

\setchapterstyle{lines}
\labch{ch:appendix}

\chapter{Source Code}

\section{Supercell Construction}
\labsec{sec:generateSupercell}

\begin{remark}
    This source code is MATLAB format under Version R2019b/R2020a, other version is not tested. The output "POSCAR" file can not be used in VASP directly because this file DO NOT remove duplicated atoms. It should be imported in VESTA, remove duplicated atoms(built-in function) and then export.
\end{remark}

\begin{lstlisting}[language=Matlab]
    %% Determine the supercell length
    clear variables;
    nMax = 40;
    mMax = 40;
    a = 3.3171310374808005;
    cutMaxA = 10* a;
    
    theta1 = zeros(nMax, mMax);
    theta2 = zeros(nMax, mMax);
    theta3 = zeros(nMax, mMax);
    theta4 = zeros(nMax, mMax);
    length1 = zeros(nMax, mMax);
    length2 = zeros(nMax, mMax);
    length3 = zeros(nMax, mMax);
    length4 = zeros(nMax, mMax);
    
    for numIdxN = 1: nMax
        for numIdxM = 1: mMax
            theta1(numIdxN, numIdxM) = 2*atand((2*numIdxN - 1)/(sqrt(3)* (2*numIdxM - 1)));
            theta2(numIdxN, numIdxM) = 2*atand(numIdxN/(sqrt(3)* numIdxM));
            theta3(numIdxN, numIdxM) = 2*atand((sqrt(3)* (2*numIdxM - 1))/(2*numIdxN - 1));
            theta4(numIdxN, numIdxM) = 2*atand((sqrt(3)* numIdxM)/numIdxN);
            length1(numIdxN, numIdxM) = a/2 * sqrt((2*numIdxN - 1)^2 + 3*(2*numIdxM - 1)^2);
            length2(numIdxN, numIdxM) = a* sqrt(numIdxN^2 + 3* numIdxM^2);
        end
    end
    length3 = length1;
    length4 = length2;
    
    [sortedTheta1, sortedIndex1] = sort(theta1(:));
    [sortedn1, sortedm1] = ind2sub(size(theta1), sortedIndex1);
    [sortedTheta2, sortedIndex2] = sort(theta2(:));
    [sortedn2, sortedm2] = ind2sub(size(theta2), sortedIndex2);
    
    % figure();
    % hold on;
    % scatter(sortedTheta1, length1(sortedIndex1), 5, 'b');
    % scatter(sortedTheta2, length2(sortedIndex2), 5, 'r');
    % hold off;
    
    thetaMix = [theta1, theta2, theta3, theta4];
    lengthMix = [length1, length2, length3, length4];
    [sortedThetaMix, sortedIndexMix] = sort(thetaMix(:));
    [sortednMix, sortedmMix] = ind2sub(size(thetaMix), sortedIndexMix);
    while ~prod(sortedmMix <= 20)
        sortedmMix(sortedmMix > 20) = sortedmMix(sortedmMix > 20) - 20;
    end
    sortedMixType = nan(size(thetaMix));
    sortedMixType = 1*(sortedIndexMix(:) <= (numel(thetaMix)/4)) + 2*(sortedIndexMix(:) > (numel(thetaMix)/4) & sortedIndexMix(:) <= (numel(thetaMix)/2)) ...
        + 3* (sortedIndexMix(:) > (numel(thetaMix)/2) & sortedIndexMix(:) <= (numel(thetaMix)*3/4)) + 4*(sortedIndexMix(:) > (numel(thetaMix)*3/4));
    
    % figure;
    % scatter(sortedThetaMix, lengthMix(sortedIndexMix), 5);
    
    [thetaMixUnique, sortedIndexMixUnique] = uniquetol(thetaMix);
    lengthMixUnique = nan(size(thetaMixUnique));
    sortedmMixUnique = nan(size(thetaMixUnique));
    sortednMixUnique = nan(size(thetaMixUnique));
    sortedMixTypeUnique = nan(size(thetaMixUnique));
    for numIdx = 1: length(thetaMixUnique)
        searchIndex = abs(thetaMixUnique(numIdx) -  sortedThetaMix) < 1e-8;
        lengthMixUnique(numIdx) = min(lengthMix(sortedIndexMix(searchIndex)));
        tmpIndex = searchIndex & abs(lengthMix(sortedIndexMix) - lengthMixUnique(numIdx))< 1e-8;
        sortedmMixUnique(numIdx) = sortedmMix(tmpIndex);
        sortednMixUnique(numIdx) = sortednMix(tmpIndex);
        sortedMixTypeUnique(numIdx) = sortedMixType(tmpIndex);
    end
    
    % figure;
    % scatter(thetaMixUnique, lengthMixUnique, 5);
    % plot(thetaMixUnique, lengthMixUnique);
    
    cutIndex = lengthMixUnique < cutMaxA;
    thetaMixUniqueCut = thetaMixUnique(cutIndex);
    lengthMixUniqueCut = lengthMixUnique(cutIndex);
    sortedmMixUniqueCut = sortedmMixUnique(cutIndex);
    sortednMixUniqueCut = sortednMixUnique(cutIndex);
    sortedMixTypeUniqueCut = sortedMixTypeUnique(cutIndex);
    
    figure;
    scatter(thetaMixUniqueCut, lengthMixUniqueCut, 5);
    xlim([0, 60]);
    
    % Clear variables
    clear length1 length2 length3 length4 lengthMix theta1 theta2 theta3 theta4 thetaMix
    clear cutMaxA mMax nMax numIdx numIdxM numIdxN tmpIndex cutIndex
    clear lengthMixUnique searchIndex sortedmMix sortedMixType sortedMixTypeUnique sortedIndexMix sortedmMix sortedmMixUnique
    clear sortedIndexMixUnique sortednMix sortednMixUnique sortedThetaMix thetaMixUnique
    %% Generates the lattice
    % a = 3.441;  % a-axies of the lattice
    b = a;      % b-axies of the lattice
    gamma = 120;% angle of <a, b>
    sizeLattice = 21;  % Scale of the system (should be odd)
    
    % Initialize of the variables
    centerOrder = (sizeLattice + 1) ./ 2;
    lattice.x = zeros(sizeLattice);
    lattice.y = zeros(sizeLattice);
    % Difference in x and y in the primitive cell
    deltaA = [a, 0];
    deltaB = [b*cosd(gamma), b*sind(gamma)];
    % Set the zero of the axies
    lattice.x(sizeLattice, 1) = 0;
    lattice.y(sizeLattice, 1) = 0;
    % Initialize the position of the center
    lattice.x(centerOrder, centerOrder) = ((centerOrder - 1)*deltaA(1) + (sizeLattice - centerOrder)*deltaB(1));
    lattice.y(centerOrder, centerOrder) = ((centerOrder - 1)*deltaA(2) + (sizeLattice - centerOrder)*deltaB(2));
    % Calculate the distance from the center
    for row = sizeLattice: -1: 1
        for column = 1: sizeLattice
            lattice.x(row, column) = ((column - 1)*deltaA(1) + (sizeLattice - row)*deltaB(1));
            lattice.y(row, column) = ((column - 1)*deltaA(2) + (sizeLattice - row)*deltaB(2));
            lattice.hoppingA(row, column) = column - centerOrder;
            lattice.hoppingB(row, column) = centerOrder - row;
        end
    end
    % Move the center to the axis zero
    lattice.x = lattice.x - lattice.x(centerOrder, centerOrder);
    lattice.y = lattice.y - lattice.y(centerOrder, centerOrder);
    
    %% Rotation
    % Construct layers
    d = 1.58106107700766;   % h-phase:1.59498653898723, t-phase:1.58106107700766
    D = 2.809105554065346;  % h-phase:3.67176692979057, t-phase:3.12122839340594
                            % Origin: 3.12122839340594
    vacuumLength = 20;
    numberOfLayers = 2;
    rotationNumIdx = 23;
    % for rotationNumIdx = 13: 13
    rotationDegree = thetaMixUniqueCut(rotationNumIdx)/2;
    supercellLattice = lengthMixUniqueCut(rotationNumIdx);
    supercellBoundx = [0, supercellLattice, supercellLattice/2, -supercellLattice/2, 0];
    supercellBoundy = [0, 0, sqrt(3)/2 * supercellLattice, sqrt(3)/2 * supercellLattice, 0];
    offsetSe1 = [deltaA', deltaB']*[2/3; 1/3];
    % offsetSe1 = [deltaA', deltaB']*[1/3; 2/3];
    offsetSe2 = [deltaA', deltaB']*[1/3; 2/3];
    layer{1}.Se1.x = lattice.x + offsetSe1(1);
    layer{1}.Se1.y = lattice.y + offsetSe1(2);
    layer{1}.Se1.z = zeros(size(lattice.x));
    layer{1}.V.x = lattice.x;
    layer{1}.V.y = lattice.y;
    layer{1}.V.z = zeros(size(lattice.x)) + d;
    layer{1}.Se2.x = lattice.x + offsetSe2(1);
    layer{1}.Se2.y = lattice.y + offsetSe2(2);
    layer{1}.Se2.z = zeros(size(lattice.x)) + 2*d;
    
    layer{2} = layer{1};
    layer{2}.Se1.z = layer{1}.Se1.z + D + 2*d;
    layer{2}.Se2.z = layer{1}.Se2.z + D + 2*d;
    layer{2}.V.z = layer{1}.V.z + D + 2*d;
    
    % Rotation Process
    for i = 1: numberOfLayers
        % layer0: counterclock; layer1: clock
        switch i
            case 1
                rotationMatrix = ...
                    [cosd(rotationDegree), -sind(rotationDegree), 0; sind(rotationDegree), cosd(rotationDegree), 0; 0 0 1];
            case 2
                rotationMatrix = ...
                    [cosd(-rotationDegree), -sind(-rotationDegree), 0; sind(-rotationDegree), cosd(-rotationDegree), 0; 0 0 1];
        end
        switch sortedMixTypeUniqueCut(rotationNumIdx)
            case {1, 2}
                correctMatrix = [cosd(-30), -sind(-30), 0; sind(-30), cosd(-30), 0; 0 0 1];
            case {3, 4}
                correctMatrix = [1 0 0; 0 1 0; 0 0 1];
        end
        for numIdx = 1: numel(layer{i}.V.x)
            tmpMat = [layer{i}.Se1.x(numIdx), layer{i}.Se1.y(numIdx), layer{i}.Se1.z(numIdx)]';
            tmpMat = correctMatrix * rotationMatrix * tmpMat;
            layer{i}.Se1.x(numIdx) = tmpMat(1);
            layer{i}.Se1.y(numIdx) = tmpMat(2);
            layer{i}.Se1.z(numIdx) = tmpMat(3);
            
            tmpMat = [layer{i}.Se2.x(numIdx), layer{i}.Se2.y(numIdx), layer{i}.Se2.z(numIdx)]';
            tmpMat = correctMatrix * rotationMatrix * tmpMat;
            layer{i}.Se2.x(numIdx) = tmpMat(1);
            layer{i}.Se2.y(numIdx) = tmpMat(2);
            layer{i}.Se2.z(numIdx) = tmpMat(3);
            
            tmpMat = [layer{i}.V.x(numIdx), layer{i}.V.y(numIdx), layer{i}.V.z(numIdx)]';
            tmpMat = correctMatrix * rotationMatrix * tmpMat;
            layer{i}.V.x(numIdx) = tmpMat(1);
            layer{i}.V.y(numIdx) = tmpMat(2);
            layer{i}.V.z(numIdx) = tmpMat(3);
        end
        % Cut the supercell
        layer{i}.V.inSupercell = inpolygon(layer{i}.V.x, layer{i}.V.y, supercellBoundx, supercellBoundy);
        layer{i}.Se1.inSupercell = inpolygon(layer{i}.Se1.x, layer{i}.Se1.y, supercellBoundx, supercellBoundy);
        layer{i}.Se2.inSupercell = inpolygon(layer{i}.Se2.x, layer{i}.Se2.y, supercellBoundx, supercellBoundy);
    end
    
    % 
    % fig = figure;
    % hold on;
    % % scatter3(layer{1}.Se1.x(:), layer{1}.Se1.y(:), layer{1}.Se1.z(:), 'green');
    % scatter3(layer{1}.V.x(:), layer{1}.V.y(:), layer{1}.V.z(:), 'red');
    % % scatter3(layer{1}.Se2.x(:), layer{1}.Se2.y(:), layer{1}.Se2.z(:), 'green');
    % % scatter3(layer{2}.Se1.x(:), layer{2}.Se1.y(:), layer{2}.Se1.z(:), 'green');
    % scatter3(layer{2}.V.x(:), layer{2}.V.y(:), layer{2}.V.z(:), 'blue');
    % % scatter3(layer{2}.Se2.x(:), layer{2}.Se2.y(:), layer{2}.Se2.z(:), 'green');
    % hold off
    % maxLim = max(abs([fig.Children.XLim, fig.Children.YLim]));
    % fig.Children.XLim = [-maxLim, maxLim];
    % fig.Children.YLim = [-maxLim, maxLim];
    
    fig = figure;
    hold on;
    % scatter3(layer{1}.Se1.x(:), layer{1}.Se1.y(:), layer{1}.Se1.z(:), 'green');
    scatter3(layer{1}.V.x(layer{1}.V.inSupercell), layer{1}.V.y(layer{1}.V.inSupercell), layer{1}.V.z(layer{1}.V.inSupercell), 'red');
    % scatter3(layer{1}.Se2.x(:), layer{1}.Se2.y(:), layer{1}.Se2.z(:), 'green');
    % scatter3(layer{2}.Se1.x(:), layer{2}.Se1.y(:), layer{2}.Se1.z(:), 'green');
    scatter3(layer{2}.V.x(layer{2}.V.inSupercell), layer{2}.V.y(layer{2}.V.inSupercell), layer{2}.V.z(layer{2}.V.inSupercell), 'blue');
    % scatter3(layer{2}.Se2.x(:), layer{2}.Se2.y(:), layer{2}.Se2.z(:), 'green');
    hold off
    maxLim = max(abs([fig.Children.XLim, fig.Children.YLim]));
    fig.Children.XLim = [-maxLim, maxLim];
    fig.Children.YLim = [-maxLim, maxLim];
    
    % Calculate POSCAR
    supercellC = vacuumLength + 4*d + D;
    VatomNumber = 0;
    SeatomNumber = 0;
    superCell.V.x = layer{1}.V.x(layer{1}.V.inSupercell);
    superCell.V.y = layer{1}.V.y(layer{1}.V.inSupercell);
    superCell.V.z = layer{1}.V.z(layer{1}.V.inSupercell);
    superCell.Se.x = cat(1, layer{1}.Se1.x(layer{1}.Se1.inSupercell), layer{1}.Se2.x(layer{1}.Se2.inSupercell));
    superCell.Se.y = cat(1, layer{1}.Se1.y(layer{1}.Se1.inSupercell), layer{1}.Se2.y(layer{1}.Se2.inSupercell));
    superCell.Se.z = cat(1, layer{1}.Se1.z(layer{1}.Se1.inSupercell), layer{1}.Se2.z(layer{1}.Se2.inSupercell));
    for i = 1: numberOfLayers
        VatomNumber = VatomNumber + length(layer{i}.V.x(layer{i}.V.inSupercell));
        SeatomNumber = SeatomNumber + length(layer{i}.Se1.x(layer{i}.Se1.inSupercell)) + length(layer{i}.Se2.x(layer{i}.Se2.inSupercell));
        if i >= 2
            superCell.V.x = cat(1, superCell.V.x, layer{i}.V.x(layer{i}.V.inSupercell));
            superCell.V.y = cat(1, superCell.V.y, layer{i}.V.y(layer{i}.V.inSupercell));
            superCell.V.z = cat(1, superCell.V.z, layer{i}.V.z(layer{i}.V.inSupercell));
            superCell.Se.x = cat(1, superCell.Se.x, layer{i}.Se1.x(layer{i}.Se1.inSupercell), layer{i}.Se2.x(layer{i}.Se2.inSupercell));
            superCell.Se.y = cat(1, superCell.Se.y, layer{i}.Se1.y(layer{i}.Se1.inSupercell), layer{i}.Se2.y(layer{i}.Se2.inSupercell));
            superCell.Se.z = cat(1, superCell.Se.z, layer{i}.Se1.z(layer{i}.Se1.inSupercell), layer{i}.Se2.z(layer{i}.Se2.inSupercell));
        end
    end
    % Write POSCAR
    fileId = fopen("POSCAR", 'w');
    fprintf(fileId, "Degree: %.5f\n", thetaMixUniqueCut(rotationNumIdx));
    fprintf(fileId, "   1.00000000000000 \n");
    fprintf(fileId, "%24.15f %24.15f %24.15f\n", [supercellLattice, 0, 0]);
    fprintf(fileId, "%24.15f %24.15f %24.15f\n", [-supercellLattice/2, sqrt(3)/2*supercellLattice, 0]);
    fprintf(fileId, "%24.15f %24.15f %24.15f\n", [0, 0, supercellC]);
    fprintf(fileId, "   V    Se\n");
    fprintf(fileId, "%6d %6d\n", [VatomNumber, SeatomNumber]);
    fprintf(fileId, "Cartesian\n");
    for numIdx = 1: VatomNumber
        fprintf(fileId, "% 18.16f % 21.16f % 21.16f\n", [superCell.V.x(numIdx), superCell.V.y(numIdx), superCell.V.z(numIdx)]);
    end
    
    for numIdx = 1: SeatomNumber
        fprintf(fileId, "% 18.16f % 21.16f % 21.16f\n", [superCell.Se.x(numIdx), superCell.Se.y(numIdx), superCell.Se.z(numIdx)]);
    end
    fclose(fileId);
    % end    
\end{lstlisting}

\section{Primitive cell}
\labsec{sec:primitiveCell}


\begin{lstlisting}[style=kaolstplain, linewidth=1.5\textwidth]
V1 Se2 T-Phase Structure
    1.00000000000000
      3.3171310374808005    0.0000000000021438   -0.0000000000000300
     -1.6585655187866546    2.8727197461676814    0.0000000000000371
     -0.0000000000002724    0.0000000000001877   27.2846672137727637
    V    Se
      2     4
 Direct
  -0.0000000000000002  0.0000000000000003  0.0456342266658590       
   0.0000000000000002 -0.0000000000000003  0.2751370207888941       
   0.6666666746654853  0.3333333495243738 -0.0126915455264852       
   0.3333333497015403  0.6666666993837396  0.1031743004030291       
   0.6666666746654855  0.3333333495243735  0.2175692596287150       
   0.3333333497015404  0.6666666993837395  0.3334908956017944       
 
   0.00000000E+00  0.00000000E+00  0.00000000E+00
   0.00000000E+00  0.00000000E+00  0.00000000E+00
   0.00000000E+00  0.00000000E+00  0.00000000E+00
   0.00000000E+00  0.00000000E+00  0.00000000E+00
   0.00000000E+00  0.00000000E+00  0.00000000E+00
   0.00000000E+00  0.00000000E+00  0.00000000E+00
\end{lstlisting}

%----------------------------------------------------------------------------------------

\backmatter % Denotes the end of the main document content
\setchapterstyle{plain} % Output plain chapters from this point onwards

%----------------------------------------------------------------------------------------
%	BIBLIOGRAPHY
%----------------------------------------------------------------------------------------

% The bibliography needs to be compiled with biber using your LaTeX editor, or on the command line with 'biber main' from the template directory

\defbibnote{bibnote}{Here are the references in citation order.\par\bigskip} % Prepend this text to the bibliography
\printbibliography[heading=bibintoc, title=Bibliography, prenote=bibnote] % Add the bibliography heading to the ToC, set the title of the bibliography and output the bibliography note

%----------------------------------------------------------------------------------------
%	NOMENCLATURE
%----------------------------------------------------------------------------------------

% The nomenclature needs to be compiled on the command line with 'makeindex main.nlo -s nomencl.ist -o main.nls' from the template directory

\nomenclature{$c$}{Speed of light in a vacuum inertial frame}
\nomenclature{$h$}{Planck constant}

\renewcommand{\nomname}{Notation} % Rename the default 'Nomenclature'
\renewcommand{\nompreamble}{The next list describes several symbols that will be later used within the body of the document.} % Prepend this text to the nomenclature

\printnomenclature % Output the nomenclature

%----------------------------------------------------------------------------------------
%	GREEK ALPHABET
% 	Originally from https://gitlab.com/jim.hefferon/linear-algebra
%----------------------------------------------------------------------------------------

% \vspace{1cm}

% {\usekomafont{chapter}Greek Letters with Pronunciations} \\[2ex]
% \begin{center}
% 	\newcommand{\pronounced}[1]{\hspace*{.2em}\small\textit{#1}}
% 	\begin{tabular}{l l @{\hspace*{3em}} l l}
% 		\toprule
% 		Character & Name & Character & Name \\ 
% 		\midrule
% 		$\alpha$ & alpha \pronounced{AL-fuh} & $\nu$ & nu \pronounced{NEW} \\
% 		$\beta$ & beta \pronounced{BAY-tuh} & $\xi$, $\Xi$ & xi \pronounced{KSIGH} \\ 
% 		$\gamma$, $\Gamma$ & gamma \pronounced{GAM-muh} & o & omicron \pronounced{OM-uh-CRON} \\
% 		$\delta$, $\Delta$ & delta \pronounced{DEL-tuh} & $\pi$, $\Pi$ & pi \pronounced{PIE} \\
% 		$\epsilon$ & epsilon \pronounced{EP-suh-lon} & $\rho$ & rho \pronounced{ROW} \\
% 		$\zeta$ & zeta \pronounced{ZAY-tuh} & $\sigma$, $\Sigma$ & sigma \pronounced{SIG-muh} \\
% 		$\eta$ & eta \pronounced{AY-tuh} & $\tau$ & tau \pronounced{TOW (as in cow)} \\
% 		$\theta$, $\Theta$ & theta \pronounced{THAY-tuh} & $\upsilon$, $\Upsilon$ & upsilon \pronounced{OOP-suh-LON} \\
% 		$\iota$ & iota \pronounced{eye-OH-tuh} & $\phi$, $\Phi$ & phi \pronounced{FEE, or FI (as in hi)} \\
% 		$\kappa$ & kappa \pronounced{KAP-uh} & $\chi$ & chi \pronounced{KI (as in hi)} \\
% 		$\lambda$, $\Lambda$ & lambda \pronounced{LAM-duh} & $\psi$, $\Psi$ & psi \pronounced{SIGH, or PSIGH} \\
% 		$\mu$ & mu \pronounced{MEW} & $\omega$, $\Omega$ & omega \pronounced{oh-MAY-guh} \\
% 		\bottomrule
% 	\end{tabular} \\[1.5ex]
% 	Capitals shown are the ones that differ from Roman capitals.
% \end{center}

%----------------------------------------------------------------------------------------
%	GLOSSARY
%----------------------------------------------------------------------------------------

% The glossary needs to be compiled on the command line with 'makeglossaries main' from the template directory

\setglossarystyle{listgroup} % Set the style of the glossary (see https://en.wikibooks.org/wiki/LaTeX/Glossary for a reference)
\printglossary[title=Special Terms, toctitle=List of Terms] % Output the glossary, 'title' is the chapter heading for the glossary, toctitle is the table of contents heading

%----------------------------------------------------------------------------------------
%	INDEX
%----------------------------------------------------------------------------------------

% The index needs to be compiled on the command line with 'makeindex main' from the template directory

\printindex % Output the index

%----------------------------------------------------------------------------------------
%	BACK COVER
%----------------------------------------------------------------------------------------

% If you have a PDF/image file that you want to use as a back cover, uncomment the following lines

%\clearpage
%\thispagestyle{empty}
%\null%
%\clearpage
%\includepdf{cover-back.pdf}

%----------------------------------------------------------------------------------------

\end{document}
