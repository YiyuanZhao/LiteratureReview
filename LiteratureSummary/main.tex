%----------------------------------------------------------------------------------------
%	EXAMPLE AND DOCUMENTATION OF THE KAOBOOK CLASS
%----------------------------------------------------------------------------------------

\documentclass[
    a4paper, % Page size
    fontsize=10pt, % Base font size
    twoside=true, % Use different layouts for even and odd pages (in particular, if twoside=true, the margin column will be always on the outside)
	%open=any, % If twoside=true, uncomment this to force new chapters to start on any page, not only on right (odd) pages
	%chapterentrydots=true, % Uncomment to output dots from the chapter name to the page number in the table of contents
	numbers=noenddot, % Comment to output dots after chapter numbers; the most common values for this option are: enddot, noenddot and auto (see the KOMAScript documentation for an in-depth explanation)
]{kaobook}

%----------------------------------------------------------------------------------------
%	PACKAGES AND OTHER DOCUMENT CONFIGURATIONS
%----------------------------------------------------------------------------------------

% Choose the language
\ifxetexorluatex
	\usepackage{polyglossia}
	\setmainlanguage{english}
\else
	\usepackage[english]{babel} % Load characters and hyphenation
\fi
\usepackage[english=british]{csquotes}	% English quotes

% Load packages for testing
\usepackage{blindtext}
%\usepackage{showframe} % Uncomment to show boxes around the text area, margin, header and footer
%\usepackage{showlabels} % Uncomment to output the content of \label commands to the document where they are used

% Load the bibliography package
\usepackage{kaobiblio}
\addbibresource{main.bib} % Bibliography file

% Load mathematical packages for theorems and related environments
\usepackage[framed=true]{kaotheorems}

% Load the package for hyperreferences
\usepackage{kaorefs}
% Custumized package
\usepackage{subfigure}
\graphicspath{{images/}} % Paths in which to look for images

\makeindex[columns=3, title=Alphabetical Index, intoc] % Make LaTeX produce the files required to compile the index

\makeglossaries % Make LaTeX produce the files required to compile the glossary
\newglossaryentry{computer}{
	name=computer,
	description={is a programmable machine that receives input, stores and manipulates data, and provides output in a useful format}
}

% Glossary entries (used in text with e.g. \acrfull{fpsLabel} or \acrshort{fpsLabel})
\newacronym[longplural={Frames per Second}]{fpsLabel}{FPS}{Frame per Second}
\newacronym[longplural={Tables of Contents}]{tocLabel}{TOC}{Table of Contents}

 % Include the glossary definitions

\makenomenclature % Make LaTeX produce the files required to compile the nomenclature

% Reset sidenote counter at chapters
%\counterwithin*{sidenote}{chapter}

%----------------------------------------------------------------------------------------

\begin{document}

%----------------------------------------------------------------------------------------
%	BOOK INFORMATION
%----------------------------------------------------------------------------------------

\titlehead{Paper Reading Summary}
% \subject{Use this document as a template}

\title[TMDCs DFT Calculation and Experiment Facts]{TMDCs DFT Calculation and Experiment Facts}
\subtitle{A summary for research trending}

\author[Yiyuan Zhao]{Yiyuan Zhao}

\date{\today}

\publishers{SCES, Tongji University}

%----------------------------------------------------------------------------------------

\frontmatter % Denotes the start of the pre-document content, uses roman numerals

%----------------------------------------------------------------------------------------
%	OPENING PAGE
%----------------------------------------------------------------------------------------

%\makeatletter
%\extratitle{
%	% In the title page, the title is vspaced by 9.5\baselineskip
%	\vspace*{9\baselineskip}
%	\vspace*{\parskip}
%	\begin{center}
%		% In the title page, \huge is set after the komafont for title
%		\usekomafont{title}\huge\@title
%	\end{center}
%}
%\makeatother

%----------------------------------------------------------------------------------------
%	COPYRIGHT PAGE
%----------------------------------------------------------------------------------------

% \makeatletter
% \uppertitleback{\@titlehead} % Header

% \lowertitleback{
% 	\textbf{Disclaimer}\\
% 	You can edit this page to suit your needs. For instance, here we have a no copyright statement, a colophon and some other information. This page is based on the corresponding page of Ken Arroyo Ohori's thesis, with minimal changes.
	
% 	\medskip
	
% 	\textbf{No copyright}\\
% 	\cczero\ This book is released into the public domain using the CC0 code. To the extent possible under law, I waive all copyright and related or neighbouring rights to this work.
	
% 	To view a copy of the CC0 code, visit: \\\url{http://creativecommons.org/publicdomain/zero/1.0/}
	
% 	\medskip
	
% 	\textbf{Colophon} \\
% 	This document was typeset with the help of \href{https://sourceforge.net/projects/koma-script/}{\KOMAScript} and \href{https://www.latex-project.org/}{\LaTeX} using the \href{https://github.com/fmarotta/kaobook/}{kaobook} class.
	
% 	The source code of this book is available at:\\\url{https://github.com/fmarotta/kaobook}
	
% 	(You are welcome to contribute!)
	
% 	\medskip
	
% 	\textbf{Publisher} \\
% 	First printed in May 2019 by \@publishers
% }
% \makeatother

%----------------------------------------------------------------------------------------
%	DEDICATION
%----------------------------------------------------------------------------------------

% \dedication{
% 	The harmony of the world is made manifest in Form and Number, and the heart and soul and all the poetry of Natural Philosophy are embodied in the concept of mathematical beauty.\\
% 	\flushright -- D'Arcy Wentworth Thompson
% }

%----------------------------------------------------------------------------------------
%	OUTPUT TITLE PAGE AND PREVIOUS
%----------------------------------------------------------------------------------------

% Note that \maketitle outputs the pages before here

\maketitle

%----------------------------------------------------------------------------------------
%	PREFACE
%----------------------------------------------------------------------------------------

% \input{chapters/preface.tex}
% \index{preface}

%----------------------------------------------------------------------------------------
%	TABLE OF CONTENTS & LIST OF FIGURES/TABLES
%----------------------------------------------------------------------------------------

\begingroup % Local scope for the following commands

% Define the style for the TOC, LOF, and LOT
%\setstretch{1} % Uncomment to modify line spacing in the ToC
%\hypersetup{linkcolor=blue} % Uncomment to set the colour of links in the ToC
\setlength{\textheight}{230\hscale} % Manually adjust the height of the ToC pages

% Turn on compatibility mode for the etoc package
\etocstandarddisplaystyle % "toc display" as if etoc was not loaded
\etocstandardlines % "toc lines as if etoc was not loaded

\tableofcontents % Output the table of contents

\listoffigures % Output the list of figures

% Comment both of the following lines to have the LOF and the LOT on different pages
\let\cleardoublepage\bigskip
\let\clearpage\bigskip

% \listoftables % Output the list of tables

\endgroup

%----------------------------------------------------------------------------------------
%	MAIN BODY
%----------------------------------------------------------------------------------------

\mainmatter % Denotes the start of the main document content, resets page numbering and uses arabic numbers
\setchapterstyle{kao} % Choose the default chapter heading style

% \setchapterpreamble[u]{\margintoc}
\chapter{Introduction}
\labch{intro}

\section{The Main Ideas}

Many modern printed textbooks have adopted a layout with prominent 
margins where small figures, tables, remarks and just about everything 
else can be displayed. Arguably, this layout helps to organise the 
	discussion by separating the main text from the ancillary material, 
	which at the same time is very close to the point in the text where 
	it is referenced.

This document does not aim to be an apology of wide margins, for there 
are many better suited authors for this task; the purpose of all these 
words is just to fill the space so that the reader can see how a book 
written with the kaobook class looks like. Meanwhile, I shall also try 
to illustrate the features of the class.

The main ideas behind kaobook come from this 
\href{https://3d.bk.tudelft.nl/ken/en/2016/04/17/a-1.5-column-layout-in-latex.html}{blog 
	post}, and actually the name of the class is dedicated to the author 
of the post, Ken Arroyo Ohori, which has kindly allowed me to create a 
class based on his thesis. Therefore, if you want to know more reasons 
to prefer a 1.5-column layout for your books, be sure to read his blog 
post.

Another source of inspiration, as you may have noticed, is the 
\href{https://github.com/Tufte-LaTeX/tufte-latex}{Tufte-Latex Class}. 
The fact that the design is similar is due to the fact that it is very 
difficult to improve something which is already so good. However, I like 
to think that this class is more flexible than Tufte-Latex. For 
instance, I have tried to use only standard packages and to implement as 
little as possible from scratch;\sidenote{This also means that 
understanding and contributing to the class development is made easier. 
Indeed, many things still need to be improved, so if you are interested, 
check out the repository on github!} therefore, it should be pretty easy 
to customise anything, provided that you read the documentation of the 
package that provides that feature.

In this book I shall illustrate the main features of the class and 
provide information about how to use and change things. Let us get 
started.

\section{What This Class Does}
\labsec{does}

The \Class{kaobook} class focuses more about the document structure than 
about the style. Indeed, it is a well-known \LaTeX\xspace principle that 
structure and style should be separated as much as possible (see also 
\vrefsec{doesnot}). This means that this class will only provide 
commands, environments and in general, the opportunity to do things, 
which the user may or may not use. Actually, some stylistic matters are 
embedded in the class, but the user is able to customise them with ease.

The main features are the following:

\begin{description}
	\item[Page Layout] The text width is reduced to improve readability 
	and make space for the margins, where any sort of elements can be 
	displayed.
	\item[Chapter Headings] As opposed to Tufte-Latex, we provide a 
	variety of chapter headings among which to choose; examples will be 
	seen in later chapters.
	\item[Page Headers] They span the whole page, margins included, and, 
	in twoside mode, display alternatively the chapter and the section 
	name.\sidenote[][-2mm]{This is another departure from Tufte's 
	design.}
	\item[Matters] The commands \Command{frontmatter}, 
	\Command{mainmatter} and \Command{backmatter} have been redefined in 
	order to have automatically wide margins in the main matter, and 
	narrow margins in the front and back matters. However, the page 
	style can be changed at any moment, even in the middle of the 
	document.
	\item[Margin text] We provide commands \Command{sidenote} and 
	\Command{marginnote} to put text in the 
	margins.\sidenote[][-2mm]{Sidenotes (like this!) are numbered while 
	marginnotes are not}
	\item[Margin figs/tabs] A couple of useful environments is 
	\Environment{marginfigure} and \Environment{margintable}, which, not 
	surprisingly, allow you to put figures and tables in the margins 
	(\cfr \reffig{marginmonalisa}).
	\item[Margin toc] Finally, since we have wide margins, why don't add 
	a little table of contents in them? See \Command{margintoc} for 
	that.
	\item[Hyperref] \Package{hyperref} is loaded and by default we try 
	to add bookmarks in a sensible way; in particular, the bookmarks 
	levels are automatically reset at \Command{appendix} and 
	\Command{backmatter}. Moreover, we also provide a small package to 
	ease the hyperreferencing of other parts of the text.
	\item[Bibliography] We want the reader to be able to know what has 
	been cited without having to go to the end of the document every 
	time, so citations go in the margins as well as at the end, as in 
	Tufte-Latex. Unlike that class, however, you are free to customise 
	the citations as you wish.
\end{description}

\begin{marginfigure}[-5.5cm]
	\includegraphics{monalisa}
	\caption[The Mona Lisa]{The Mona Lisa.\\ 
	\url{https://commons.wikimedia.org/wiki/File:Mona_Lisa,_by_Leonardo_da_Vinci,_from_C2RMF_retouched.jpg}}
	\labfig{marginmonalisa}
\end{marginfigure}

The order of the title pages, table of contents and preface can be 
easily changed, as in any \LaTeX\ document. In addition, the class is 
based on \KOMAScript's \Class{scrbook}, therefore it inherits all the 
goodies of that.

\section{What This Class Does Not Do}
\labsec{doesnot}

As anticipated, further customisation of the book is left to the user. 
Indeed, every book may have sidenotes, margin figures and so on, but 
each book will have its own fonts, toc style, special environments and 
so on. For this reason, in addition to the class, we provide only 
sensible defaults, but if these features are not needed, they can be 
left out. These special packages are located in the \Path{style} 
directory, which is organised as follows:

\begin{description}
	\item[kao.sty] This package contains the most important definitions 
	of macros and specifications of page layout. It is the heart of the 
	\Class{kaobook}.
	\item[kaobiblio.sty] Contains commands to add citations and 
	customise the bibliography.
	\item[packages.sty] Loads additional packages to decorate the 
	writing with special contents (for instance, the \Package{listing} 
	package is loaded here as it is not required in every book). There 
	are also defined some useful commands to print the same words always 
	in the same way, \eg latin words in italics or \Package{packages} in 
	verbatim.
	\item[kaorefs.sty] Some useful commands to manage labeling and 
	referencing, again to ensure that the same elements are referenced 
	always in a consistent way.
	\item[environments.sty] Provides special environments, like boxes. 
	Both simple and complex environments are available; by complex we 
	mean that they are endowed with a counter, floating and can be put 
	in a special table of contents.\sidenote[][-2mm]{See 
	\vrefch{mathematics} for some examples.}
	\item[theorems.sty] The style of mathematical environments. 
	Actually, there are two such packages: one is for plain theorems,
	\ie the theorems are printed in plain text; the other uses 
	\Package{mdframed} to draw a box around theorems. You can plug the 
	most appropriate style into its document.
\end{description}

\marginnote[2mm]{The audacious users might feel tempted to edit some of 
these packages. I'd be immensely happy if they sent me examples of what 
they have been able to do!}

In the rest of the book, I shall assume that the reader is not a novice 
in the use of \LaTeX, and refer to the documentation of the packages 
used in this class for things that are already explained there. 
Moreover, I assume that the reader is willing to make minor edits to the 
provided packages for styles, environments and commands, if he or she 
does not like the default settings.

\section{How to Use This Class}

Either if you are using the template from 
\href{http://latextemplates.org/template/kaobook}{latextemplates}, or if 
you cloned the GitHub 
\href{https://www.github.com/fmarotta/kaobook}{repository}, there are 
infinite ways to use the \Class{kaobook} class in practice, but we will 
discuss only two of them. The first is to find the \Path{main.tex} file 
which I used to write this book, and edit it; this will probably involve 
a lot of text-deleting, copying-and-pasting, and rewriting. The second 
way is to start almost from scratch and use the \Path{skeleton.tex} 
file, which is a cleaned-up version of the \Path{main.tex}; even if you 
choose the second way, you may find it useful to draw inspiration from 
the \Path{main.tex} file.

To compile the document, assuming that its name is \Path{main.tex}, you 
will have to run the following sequence of commands:

\begin{lstlisting}[style=kaolstplain,linewidth=1.5\textwidth]
pdflatex main # Compile template
makeindex main.nlo -s nomencl.ist -o main.nls # Compile nomenclature
makeindex main # Compile index
biber main # Compile bibliography
makeglossaries main # Compile glossary
pdflatex main # Compile template again
pdflatex main # Compile template again
\end{lstlisting}

You may need to compile the template some more times in order for some 
errors to disappear. For any support requests, please ask a question on 
\url{tex.stackexchange.org} with the tag \enquote{kaobook}, open an 
issue on GitHub, or contact the author via e-mail.


\pagelayout{wide} % No margins
\addpart{Research Trending}
\pagelayout{margin} % Restore margins

\setchapterpreamble[u]{\margintoc}
\chapter{General TMDCs}
% \labch{ch:GeneralTMDCs}

\section{Ground State Determination}
% \labsec{sec:GroundStateDetermination}
\subsection{Experiment Facts}
Here're some expressions that describes goundstate of TMDCs.

\subsection{DFT Calculation}
Here're some expressions that describes goundstate of TMDCs.

\section{Magnetic Properties}
% \labsec{sec:MagneticProperties}
The \refthm{MerminWagnerTheory}(Mermin-Wagner theory) shows that a one- or two-dimensional isotropic spin-8 Heisenberg model with finite-range exchange interaction can be neither ferromagnetic nor antiferromagnetic\cite{PhysRevLett.17.1133}.

\section{CDW Configuration}
% \labsec{sec:CDWConfiguration}
Here're some expressions that describes CDW of TMDCs.

\section{Other}
% \labsec{sec:Other}
Particularly attractive materials are the transition-metal dichalcogenides (TMDs), due to their van der Waals layered structures and their wide range of material properties. Controlling the material thickness down to the single-atom scale enables tuning of the interacting electronic states and phases, and it is thus highly desirable to fabricate high-quality monolayer (ML) TMDs\cite{doi:10.1021/acsnano.9b02996}(Introduction).

When reduced to the monolayer (ML) limit, layered transition-metal dichalcogenides (TMDs) have shown strong dimensionality effects in numerous fundamental properties, such as band-gap type, superconductivity, charge-density wave (CDW) formation, and ferromagnetism. The substrate may play an important role due to strain and dielectric screening\cite{doi:10.1021/acs.nanolett.8b01764}(Introduction).

\chapter{VS$_2$}
% \labch{ch:VS2}

\section{Ground State Determination}
% \labsec{sec:GroundStateDetermination}
\subsection{Experiment Facts}
Here're some expressions that describes goundstate of TMDCs.

\subsection{DFT Calculation}
Here're some expressions that describes goundstate of TMDCs.

\section{Magnetic Properties}
% \labsec{sec:MagneticProperties}
Here're some expressions that describes Magnetic Properties of TMDCs.

\section{CDW Configuration}
% \labsec{sec:CDWConfiguration}
Here're some expressions that describes CDW of TMDCs.

\section{Other}
% \labsec{sec:Other}
Here're some expressions that describes Other Physics of TMDCs.

\chapter{VSe$_2$}
% \labch{ch:VS2}

\section{Ground State Determination}
% \labsec{sec:GroundStateDetermination}
\subsection{Experiment Facts}
VSe2 is the only TMDC to exhibit such characteristics experimentally on the monolayer scale. The peculiar physical properties of VSe2 such as high electrical conductivity, chargedensity-wave (CDW) transition, the Kondo effect, and the weak localization e!ect were revealed by electrical transport measurements only recently(Introduction). 
\subsubsection{Bulk}
The crystal structure of VSe2 was confirmed to be a hexagonal phase with lattice parameters of a = b = 3.3587 Å and c = 6.1075 Å (space group P3!m1, JCPDS card no. 89-1641)\cite{C8NR09258C}.
\subsubsection{Monolayer}
the apparent height of the VSe2 film is 6.9 Å, which agrees well with the unit cell height of bulk VSe2 (6.1 Å).
\subsubsection{Nanosheets}
We obtained the in-plane and out-of-plane lattice constants of 3.26 ± 0.2 Å and 6.27 ± 0.3 Å. The structural phase of the as-grown VSe2 nanosheets was 1T\cite{C8NR09258C}.
\subsection{DFT Calculation}
VSe2 is one of the few TMDCs that have been theoretically predicted to show ferromagnetic characteristics\cite{C8NR09258C}(Introduction). Fumega and Pardo reported that DFT calculations yield a ferromagnetic ground state, which is on the verge of instability. They demonstrated that structural rearrangement due to the charge density wave causes an energy gap opening at the Fermi level and, in turn, the quenching of ferromagnetism in the ML limit, offering a clue to explain the controversy\cite{doi:10.1021/acsnano.9b02996}. 
\subsubsection{Bulk}

\subsubsection{Monolayer}
Our calculations demonstrate that the transition from the experimentally observed T configuration to the H configuration is accompanied by a considerable change in the electronic structure, which is a redistribution of 3d electrons of vanadium between orbitals. On the other hand, the values of the magnetic moments and total energies of the ferro- and antiferromagnetic configurations change gradually between the two structural phases. The analysis of the calculated phonon dispersions has demonstrated a principal role of ferromagnetism in stabilization of the atomic structure of the VSe2 monolayer in the H phase and similar systems \cite{C9CP03726H}(Conclusion).

\subsubsection{Nanosheets}
The calculations for bi- and trilayers demonstrate that the energy barrier of the transition is similar to the monolayer. Strain, possibly induced by a substrate, provides the change of the most energetically favorable structure from H to T. Therefore, the experimental observation of the T configuration can result from the VSe2 structure stretching by more than 3 percent on substrates \cite{C9CP03726H}.
\section{Magnetic Properties}
% \labsec{sec:MagneticProperties}
\subsection{Experiment}
Our measurements indicate no ferromagnetism down to 10 K. Instead, they reveal that a charge-density wave that gaps the entire high-temperature Fermi surface is the dominant instability at low temperatures in monolayer VSe2 \cite{doi:10.1021/acs.nanolett.8b01649}. Bonilla et al. indicated that a ferromagnetic state persists up to room temperature in ML VSe2, but their report has several critical unexplained features. Their most debatable finding is the reported magnetization of $15 \mu_{B}$ per V atom, which disagrees with the density functional heory (DFT) calculations for ML VSe2. On the other hand, Feng et al.\cite{doi:10.1021/acs.nanolett.8b01649} showed using X-ray magnetic circular dichroism (XMCD) that there is a zero magnetic moment on the V atoms down to 10 K in an applied magnetic field of 9 T(Introduction). Our work provides clear experimental evidence that ML VSe2, without the signature of ferromagnetism, is on the verge of magnetic transition\cite{doi:10.1021/acsnano.9b02996}(Conclusion).

ML VSe2 has attracted special attention from the scientific community due to several recent discoveries, including in-plane piezoelectricity, a pseudogap with a Fermi arc at temperatures above the charge density wave transition (220 K for the monolayer), and especially the existence of ferromagnetism in the 2D system. Experimental results are rather contradictory. Strong room-temperature ferromagnetism with a huge magnetic moment per formula unit has been reported for monolayer VSe2 epitaxially grown on graphite. A local magnetic phase contrast has also been observed by magnetic force microscopy at room temperature at the edges of VSe2 flakes exfoliated from a threedimensional crystal. XMCD measurements evidence a spinfrustrated magnetic structure in VSe2 on graphite. Paramagnetism of bulk VSe2 makes these observations more intriguing. Another situation was reported for monolayers grown on bilayer graphene/silicon carbide substrates. In both studies the absence of exchange splitting of the vanadium 3d bands observed in angle-resolved photoemission spectroscopy experiments was reported \cite{C9CP03726H}(Introduction).

We also performed SQUID measurements on our ML VSe2 and detected a magnetic hysteresis signal. The observed magnetic signature does not show any correlations with our two types of charge ordering transitions and full energy gap at low temperatures\cite{doi:10.1021/acs.nanolett.8b01764}(Conclusion).

\subsection{Calculation}
It has been proposed that the presence of charge density waves could cause the quenching of monolayer ferromagnetism due to the band gap opening induced by Peierls distortion. Phonon spectra of several VSe2 and similar systems were also considered theoretically \cite{C9CP03726H}(Introduction).
\section{CDW Configuration}
% \labsec{sec:CDWConfiguration}
\subsection{Bulk}
VSe2 bulk is known to host a CDW in the bulk with an onset temperature of $T_c \approx 110 K$. $4 \times 4 \times 3$ CDW reconstruction of the bulk has a significant component in the
out-of-plane direction, which has been attributed to a nesting of the strongly three-dimensional Fermi surface at the corresponding wave vector\cite{doi:10.1021/acs.nanolett.8b01649}(Introduction).

Vanadium diselenide (VSe2), a member of the metallic TMD family, presents a rare example of a three-dimensional (3D) CDW phase that has a 4a × 4a × 3c nesting vector with a
transition temperature (TCDW) of 105 K and coexists with itinerant electrons of the residual Fermi surface (FS). The CDW distortion is small compared to the atomic corrugation,
mostly due to the weak 3D nesting condition\cite{doi:10.1021/acs.nanolett.8b01764}.

\subsection{Monolayer}
The pronounced energy gaps develop over the entire Fermi surface with decreasing temperature below $T_c = 140 \pm 5$ K\cite{doi:10.1021/acs.nanolett.8b01649}.

Angle resolved photoemission spectroscopy (ARPES) uncovers that Fermi-surface nesting becomes perfect in ML VSe2. Renormalization group analysis confirms that imperfect nesting in three dimensions universally flows into perfect nesting in two dimensions. As a result, the charge-density wave-transition temperature is dramatically enhanced to a value of 350 K compared to the 105 K in bulk VSe2\cite{doi:10.1021/acs.nanolett.8b01764}(Conclusion).

\subsection{Nanosheets}
VSe2 films could possess different structural and electronic orderings, such as a moire structure on MoS2 substrates6 and an insulating $4 \times \sqrt{3}3$ surface reconstruction on $Al_2O_3$ substrates, implying an important role of heterointerface for tuning electronic and structural properties of VSe2\cite{doi:10.1021/acs.nanolett.8b01764}.

\section{Other}
% \labsec{sec:Other}
Here're some expressions that describes Other Physics of TMDCs.

\chapter{VTe$_2$}
% \labch{ch:VS2}

\section{Ground State Determination}
% \labsec{sec:GroundStateDetermination}
\subsection{Experiment Facts}
Here're some expressions that describes goundstate of TMDCs.

\subsection{DFT Calculation}
Here're some expressions that describes goundstate of TMDCs.

\section{Magnetic Properties}
% \labsec{sec:MagneticProperties}
Here're some expressions that describes Magnetic Properties of TMDCs.

\section{CDW Configuration}
% \labsec{sec:CDWConfiguration}
Here're some expressions that describes CDW of TMDCs.

\section{Other}
% \labsec{sec:Other}
Here're some expressions that describes Other Physics of TMDCs.
% \input{chapters/textnotes.tex}
% \input{chapters/figsntabs.tex}
% \input{chapters/references.tex}

\pagelayout{wide} % No margins
\addpart{Summary of Paper}
\pagelayout{margin} % Restore margins

% \setchapterpreamble[u]{\margintoc}
\chapter{Electronic Structure and Enhanced Charge-Density Wave Order of Monolayer VSe2\cite{doi:10.1021/acs.nanolett.8b01649}}
% \labch{ch:SampleTitle}

\section{Summary Table}
% \labsec{sec:SummaryTable}

\begin{table}[h]
    \begin{tabular}{lccc}
    \hline
    Item  & Material         & Classification & Topic        \\  \hline
    Value & VSe2             & Experiment     & Magnetic+CDW \\  \hline
    Item  & Range            & System         & Publish Year \\  \hline
    Value & Structure        & Monolayer      & 2018         \\  \hline
    \end{tabular}
\end{table}

\section{Background}
% \labsec{sec:Background}
Yet, a consistent picture is still to emerge over how their charge-ordered states evolve when reducing materials thickness down to a single monolayer. In part, this reflects an intrinsic competition, whereby the microscopic interactions that drive such phase formation and the fluctuations that destabilize it are both expected to become strengthened in the two-dimensional limit compared to their bulk three-dimensional counterparts.

\section{Methodology}
% \labsec{sec:Methodology}

\subsection{Experiment}
Films are grown usint MBE meghod on epitaxial bilayer graphene/SiC as well as highly oriented pyrolytic graphite (HOPG) substrates. For a typical growth, the substrate is first annealed to 550C for 60 min before cooling to a growth temperature of 300C.

\subsection{DFT}
Software tools: Wien2k.
\begin{table}[h]
    \begin{tabular}{cccc}
    \toprule
    KPOINTS                 & ENCUT  & Pseudopotential & Hubbard-U \\
    \midrule
    $14 \times 14 \times 2$ & Unknown& PBE             & Unknown   \\
    \bottomrule
    \end{tabular}
\end{table}
Other parameter: Vacuum thickness 14.85 A.
\section{Conclusion}
% \labsec{sec:Conclusion}
Here, we use angleresolved photoemission to investigate the electronic structure of monolayer VSe2 grown on bilayer graphene/SiC. While the global electronic structure is similar to that of bulk VSe2, we show that,  for the monolayer, pronounced energy gaps develop over the entire Fermi surface with decreasing temperature below $T_c = 140 \pm 5$ K, concomitant with the emergence of charge-order superstructures evident in low-energy electron diffraction.

The nesting vector in monolayer is in stark contrast to bulk VSe2, where gaps are thought to only open over small portions of the Fermi surface that are well-nested. In the monolayer, we find that the flat portions of Fermi surface have a nesting vector along a* of $q_{nest} = 0.54 ± 0.04 Å^{-1}$ , which is close to the ordering wavevector of the CDW. Similarly, we note that the apparently even better Fermi surface nesting along the one-dimensional directions at 30◦ rotation from the a* direction that is evident in \reffig{fig:acs.nanolett.8b0164_ElectronicStructure}i, j does not appear to dominate the CDW ordering vector here, further confirming that nesting is not the driving force of the CDW in monolayer VSe2.
\begin{figure}[ht] 
    \includegraphics[width=0.8\linewidth]{./images/acs.nanolett.8b01649_ElectronicStructure}
	\caption[Normal-state electronic structure of bulk and monolayer VSe2]{
		Normal-state electronic structure of bulk and monolayer VSe2.
	}
	\labfig{fig:acs.nanolett.8b0164_ElectronicStructure}
\end{figure}

A lack of ferromagnetism here is further supported by measurements of V L2,3-edge X-ray magnetic circular dichroism (XMCD) from our monolayer VSe2 samples, shown in \reffig{fig:acs.nanolett.8b0164_AbsenceOfMagneticOrder}. The absence of an exchange splitting in the electronic structure, discussed above, further allows us to exclude that ferromagnetism develops down to our lowest ARPES measurement temperature of T = 10 K.

\begin{figure}[ht] 
    \includegraphics[width=0.8\linewidth]{./images/acs.nanolett.8b01649_AbsenceOfMagneticOrder}
	\caption[Absence of ferromagnetic order]{
		Absence of ferromagnetic order.
	}
	\labfig{fig:acs.nanolett.8b0164_AbsenceOfMagneticOrder}
\end{figure}

These observations point to a \textbf{charge density wave instability} in the monolayer that is strongly enhanced over that of the bulk. Moreover, our measurements of both the electronic structure and of Xray magnetic circular dichroism reveal \textbf{no signatures of a ferromagnetic ordering}, in contrast to the results of a recent experimental study as well as expectations from density functional theory.

\section{Defect}
% \labsec{sec:Defect}

\begin{enumerate}
    \item This research proposed their hypothesis that CDW has competition effect with magnetic order, thus their sample has no signal of magnetic momentum. This paper just raised the question.
    \item The paper excludes the nesting mechanism of CDW formation, but not raise their explanation.
\end{enumerate}
% \section{Notes}
% % \labsec{sec:Notes}
% We have some great notes here.
% \setchapterpreamble[u]{\margintoc}
\chapter{Atomistic real-space observation of the van der Waals layered structure and tailored morphology in VSe2.\cite{C8NR09258C}}
% \labch{ch:SampleTitle}

\section{Summary Table}
% \labsec{sec:SummaryTable}

\begin{table}[h]
    \begin{tabular}{lccc}
    \hline
    Item  & Material         & Classification & Topic        \\  \hline
    Value & VSe2             & Experiment     & Structure+Magmom    \\  \hline
    Item  & Range            & System         & Publish Year \\  \hline
    Value & Structure        & Nanosheets     & 2019         \\  \hline
    \end{tabular}
\end{table}

\section{Abstract}
Two-dimensional (2D) transition metal dichalcogenides with van der Waals gaps have attracted much attention due to their peculiarly distinctive physical properties from their bulk counterparts. Among them, vanadium diselenide (VSe2) has been considered to be a promising candidate for future spintronic devices, as room temperature ferromagnetism was reported recently. However, detailed crystallography and properties of VSe2 nanosheets have been less explored. Here, we report the atomistic real-space observation of the van der Waals layered structure of VSe2 for the first time. Furthermore, simply by controlling the carrier gas flow rate, a morphological variation of the surface area and thickness of VSe2 nanosheets was observed. The room temperature ferromagnetic feature of single VSe2 nanosheets was also revealed by magnetic force microscopy measurements. Our fndings will play a significant role in the research of intrinsic 2D ferromagnetic materials.

\section{Background}
% \labsec{sec:Background}
To progress further and explore the wide electrical and magnetic diversities of VSe2, the control over the sample preparation and the corresponding physical properties is necessary.

\section{Methodology}
% \labsec{sec:Methodology}

\subsection{Experiment}
Single crystalline VSe2 nanosheets is synthesized by chemical vapor deposition (CVD). Illustration of experiment shows in \reffig{fig:10.1039.c8nr09258c_Schematic}.

\begin{figure}[ht] 
    \includegraphics[width=0.8\linewidth]{./images/10.1039.c8nr09258c_Schematic}
	\caption[Schematic of VSe2 nanosheet growth]{
		Schematic of VSe2 nanosheet growth.
	}
	\labfig{fig:10.1039.c8nr09258c_Schematic}
\end{figure}

% \subsection{DFT}
% Software tools: Wien2k.
% \begin{table}[h]
%     \begin{tabular}{cccc}
%     \toprule
%     KPOINTS                 & ENCUT  & Pseudopotential & Hubbard-U \\
%     \midrule
%     $14 \times 14 \times 2$ & Unknown& PBE             & Unknown   \\
%     \bottomrule
%     \end{tabular}
% \end{table}
% Other parameter: Vacuum thickness 14.85 A.
\section{Conclusion}
% \labsec{sec:Conclusion}
By an X-ray di!raction analysis, the crystal structure of VSe2 was confirmed to be a hexagonal phase with lattice parameters of a = b = 3.3587 Å and c = 6.1075 Å (space group P3!m1, JCPDS card no. 89-1641). In particular, the angle between Se–V–Se bonds was estimated to be 97°, confirming that the structural phase of the as-grown VSe2 nanosheets was 1T.
\begin{figure}[ht] 
    \includegraphics[width=0.8\linewidth]{./images/10.1039.c8nr09258c_TEMImageOfVSe2Nanosheet}
	\caption[TEM image of a half-hexagonal VSe2 nanosheet on a Cu grid]{
		TEM image of a half-hexagonal VSe2 nanosheet on a Cu grid.
	}
	\labfig{fig:10.1039.c8nr09258c_TEMImageOfVSe2Nanosheet}
\end{figure}

The MFM is a reliable technique that can observe the magnetic regions of nanoscale magnetic materials. The MFM image of the VSe2 nanosheet with 30 nm thickness and a flat surface is shown brightly around the edge (See \reffig{fig:10.1039.c8nr09258c_TEMImageOfVSe2Nanosheet}), indicating the presence of magnetism in the nanosheet.

Here, we report the \textbf{atomistic real-space observation of the van der Waals layered structure of VSe2} for the first time. Furthermore, simply by controlling the carrier gas flow rate, a morphological variation of the surface area and thickness of VSe2 nanosheets was observed. The \textbf{room temperature ferromagnetic} feature of single VSe2 nanosheets was also revealed by magnetic force microscopy measurements. By \textbf{controlling the carrier gas flow rate} during the synthetic reaction, we observe the \textbf{morphological variation} of the surface area and thickness of the VSe2 nanosheets.

\section{Defect}
% \labsec{sec:Defect}

\begin{enumerate}
    \item Using CVD methods, which could not gain monolayer usually(Nanosheets at this time).
    \item The paper does not gives the magnetic moment of the nanosheets.
\end{enumerate}

\section{Notes}
% % \labsec{sec:Notes}
VSe2 nanosheets(~ 40 nm) are magnetic. However the temperature is unclear. 1T goundstate for nanosheets.
% \setchapterpreamble[u]{\margintoc}
\chapter{Absence of Ferromagnetism or Antiferromagnetism in One- or Two-Dimensional Isotropic Heisenberg Models\cite{PhysRevLett.17.1133}}
% \labch{ch:SampleTitle}

\section{Summary Table}
% \labsec{sec:SummaryTable}

\begin{table}[h]
    \begin{tabular}{lccc}
    \hline
    Item  & Material         & Classification & Topic        \\  \hline
    Value & 2D Material      & Theroy         & Magnetic     \\  \hline
    Item  & Range            & System         & Publish Year \\  \hline
    Value & Magnetic         & 2D Structure   & 1966         \\  \hline
    \end{tabular}
\end{table}

\section{Background}
% \labsec{sec:Background}
The classical paper which proves that pure 2D isotropic material can not be magnetic.

\section{Methodology}
% \labsec{sec:Methodology}

% \subsection{Experiment}
% Substration: Bilayer graphene.

\subsection{Model}
The proof exploits Bogoliubov's inequality.
\begin{theorem}
    Bogoliubov's inequality: \\
    \begin{equation}
        \frac{1}{2} \left\langle \left\{ A, A^{\dagger} \right\} \right\rangle \left\langle \left[ \left[ C, H \right], C^{\dagger} \right] \right\rangle 
        \ge k_B T \left|\langle\left[ C, A \right]\rangle\right|^2
    \end{equation}
\end{theorem}

Finally we get the conclusion that:
\begin{theorem}
    Mermin-Wagner theorem: \\
    \begin{eqnarray}
        s_z^2 &<& \frac{2\pi\rho S(S+1)}{k_0^2} \frac{\omega}{kT} \frac{1}{\ln(1 + \omega/ \left| h s_z \right|)} \labeq{eqn:2D_Case}\\
        \left| s_z \right|^3 &<& \left| h \right| \omega \left( \frac{S(S+1)}{2kT \tan^{-1}\left[ \omega/ \left| hs_z \right|^{-1/2} \right]} \right) \labeq{eqn:1D_Case}\\
        \omega &=& \sum_{\vec{k}} S(S+1) k_0^2 R^2 \left| J(\vec{R}) \right| \nonumber
    \end{eqnarray}
    \labthm{MerminWagnerTheory}
\end{theorem}
Where the 2D case follows \refeq{eqn:2D_Case}, the 1D case follows \refeq{eqn:1D_Case}. $h$ means sufficiently small fields.
\section{Conclusion}
% \labsec{sec:Conclusion}
It is rigorously proved that at any nonzero temperature, a one- or two-dimensional isotropic spin-8 Heisenberg model with finite-range exchange interaction can be neither ferromagnetic nor antiferromagnetic. The method of proof is capable of excluding a variety of types of ordering in one and two dimensions.

\section{Defect}
% \labsec{sec:Defect}
There' re some limitation that applies this theory.
\begin{enumerate}
    \item If the coupling is anisotropic the argument is inconclusive, but if $J_y = J_z \neq J_x$, then the same conclusions are reached for aligning fields in the $z$ direction.
    \item Our inequality rules out only spontaneous magnetization or sublattice magnetization. It does not exclude the possibility of other kinds of phase transitions.
    \item A very similar argument rules out the existence of long-range crystalline ordering in one or two dimensions, without making the harmonic approximation.
    \item Since our conclusions hold whatever the magnitude of 8, one would expect them to apply to classical spin systems. We can, in fact, prove them directly by purely classical arguments in such cases.
\end{enumerate}
% \section{Notes}
% % \labsec{sec:Notes}
% We have some great notes here.
% ewwew\setchapterpreamble[u]{\margintoc}
\chapter{Magnetic Transition in Monolayer VSe2 via Interface Hybridization\cite{doi:10.1021/acsnano.9b02996}}
% \labch{ch:SampleTitle}

\section{Summary Table}
% \labsec{sec:SummaryTable}

\begin{table}[h]
    \begin{tabular}{lccc}
    \hline
    Item  & Material         & Classification & Topic        \\  \hline
    Value & VSe2             & Experiment+DFT & Magnetic     \\  \hline
    Item  & Range            & System         & Publish Year \\  \hline
    Value & Magmom           & Monolayer      & 2019         \\  \hline
    \end{tabular}
\end{table}

\section{Abstract}
Magnetism in monolayer (ML) VSe2 has attracted broad interest in spintronics, while existing reports have not reached consensus. Using element-specific X-ray magnetic circular dichroism, a magnetic transition in ML VSe2 has been demonstrated at the contamination-free interface between Co and VSe2. Through interfacial hybridization with a Co atomic overlayer, a magnetic moment of about 0.4 $\mu_B$ per V atom in ML VSe2 is revealed, approaching values predicted by previous theoretical calculations. Promotion of the ferromagnetism in ML VSe2 is accompanied by its antiferromagnetic coupling to Co and a reduction in the spin moment of Co. In comparison to the absence of this interface-induced ferromagnetism at the Fe/ML MoSe2 interface, these findings at the Co/ML VSe2 interface provide clear proof that the ML VSe2, initially with magnetic disorder, is on the verge of magnetic transition.

\section{Background}
% \labsec{sec:Background}
Bonilla et al. indicated that a ferromagnetic state persists up to room temperature in ML VSe2, but their report has several critical unexplained features. Their most debatable finding is the reported magnetization of $15 \mu_B$ per V atom, which disagrees with the density functional theory (DFT) calculations for ML VSe2.

Subsequently, we obtained the \textbf{first experimental evidence of spin frustration in ML VSe2}, which leads to a highly degenerate ground state and forbids magnetic ordering. Although lacking long-range magnetic order in such a frustrated magnet, perturbations that break the symmetry of the lattice could possibly break the ground-state degeneracy and lead to magnetic order.

\section{Methodology}
% \labsec{sec:Methodology}

\subsection{Experiment}
Substration: HOPG. Method: MBE. Temperature: 550 C.

\subsection{DFT}
\begin{table}[h]
    \begin{tabular}{cccc}
    \toprule
    KPOINTS                 & ENCUT  & Pseudopotential & Hubbard-U \\
    \midrule
    $15 \times 15 \times 1$ & 500 eV & PBE             & Unapplied  \\
    \bottomrule
    \end{tabular}
\end{table}
Vacuum thickness: 12 A, convergence criterion is $10^{-5}$.
\section{Conclusion}
% \labsec{sec:Conclusion}

\begin{figure}[ht] 
    \includegraphics[width=0.8\linewidth]{./images/10.1021.acsnano.9b02996_STMTopographyOfMLVSe2}
	\caption[STM topography and XA/XMCD spectroscopies of ML VSe2]{
        STM topography and XA/XMCD spectroscopies of ML VSe2.
	}
	\labfig{10.1021.acsnano.9b02996_STMTopographyOfMLVSe2}
\end{figure}

Step height is ~7 A. XA spectroscopy at the V $L_{2,3}$ edge shows in \reffig{10.1021.acsnano.9b02996_STMTopographyOfMLVSe2}b. These spectral features are akin to those measured from other $3d^1$ vanadium compounds, thus providing a spectroscopic fingerprint of the \textbf{1T phase of the monolayer}. XMCD measurement at the V $L_{2,3}$ edge shows in \reffig{10.1021.acsnano.9b02996_STMTopographyOfMLVSe2}c. The XMCD contrast, within the scale of experimental error, is negligible and thus \textbf{rules out the existence of intrinsic ferromagnetism in the monolayer}, which demonstrating ML VSe2 as a frustrated magnet, in which its spins exhibit subtle correlations albeit in the absence of a long-range magnetic order.

As shown in \reffig{10.1021.acsnano.9b02996_MicroscopicMagneticMoments}, the orbital moment of V, mL,V, carries the opposite sign as the spin moment, mS,V, in agreement with Hund’s rule for a less than half-filled 3d shell. Moreover, the resulting total moment of V, mtot,$V = m_{L,V} + m_{S,V}$, has an opposite sign to that of Co (mtot,Co), confirming an antiferromagnetic coupling between Co and ML VSe2.

\begin{figure}[ht] 
    \includegraphics[width=0.8\linewidth]{./images/10.1021.acsnano.9b02996_MicroscopicMagneticMoments}
	\caption[Microscopic Magnetic Moments at the Co/ML VSe2 Interface]{
        Microscopic Magnetic Moments at the Co/ML VSe2 Interface.
	}
	\labfig{10.1021.acsnano.9b02996_MicroscopicMagneticMoments}
\end{figure}

Through interfacial hybridization with a Co atomic overlayer, a magnetic moment of about $0.4 \mu_B$ per V atom in ML VSe2 is revealed, approaching values predicted by previous theoretical calculations. Promotion of the ferromagnetism in ML VSe2 is accompanied by its antiferromagnetic coupling to Co and a reduction in the spin moment of Co (See in \reffig{10.1021.acsnano.9b02996_XMCDHysteresisLoops}). 

\begin{marginfigure}
    \includegraphics{./images/10.1021.acsnano.9b02996_XMCDHysteresisLoops}
	\caption[XMCD hysteresis loops obtained at Co and V $L_3$ edge at 65 K]{
        XMCD hysteresis loops obtained at Co and V $L_3$ edge at 65 K, which confirms the antiparallel alignment, i.e., antiferromagnetic coupling, between Co and V spins.
	}
	\labfig{10.1021.acsnano.9b02996_XMCDHysteresisLoops}
\end{marginfigure}

In comparison to the absence of this interface-induced ferromagnetism at the Fe/ML MoSe2 interface, these findings at the Co/ML VSe2 interface provide clear proof that the ML VSe2, initially with magnetic disorder, is \textbf{on the verge of magnetic transition}.

\section{Defect}
% \labsec{sec:Defect}
\begin{enumerate}
    \item The accuracy of DFT calculation is not high, as well as the density of KPOINTS.
    \item The machanism of magnetization is still unclear.
\end{enumerate}

\section{Notes}
% \labsec{sec:Notes}
This paper believes that ML VSe2 is nonmagnetic, while Co/ML VSe2 is ferromagnetic.
% \setchapterpreamble[u]{\margintoc}
\chapter{Structural phase transitions in VSe2: energetics, electronic structure and magnetism\cite{C9CP03726H}}
% \labch{ch:SampleTitle}

\section{Summary Table}
% \labsec{sec:SummaryTable}

\begin{table}[h]
    \begin{tabular}{lccc}
    \hline
    Item  & Material         & Classification & Topic        \\  \hline
    Value & VSe2             & DFT            & Magnetic     \\  \hline
    Item  & Range            & System         & Publish Year \\  \hline
    Value & Structure+Magmom & Monolayer      & 2019         \\  \hline
    \end{tabular}
\end{table}

\section{Background}
% \labsec{sec:Background}
ML VSe2 has attracted special attention from the scientific community due to several recent discoveries, including in-plane piezoelectricity, a pseudogap with a Fermi arc at temperatures above the charge density wave transition (220 K for the monolayer), and especially the existence of ferromagnetism in the 2D system. 

Experimental results are rather contradictory. Strong room-temperature ferromagnetism with a huge magnetic moment per formula unit has been reported for monolayer VSe2
epitaxially grown on graphite. A local magnetic phase contrast has also been observed by magnetic force microscopy at room temperature at the edges of VSe2 flakes exfoliated from a threedimensional crystal. XMCD measurements evidence a spinfrustrated magnetic structure in VSe2 on graphite. Paramagnetism of bulk VSe2 makes these observations more intriguing. In both studies the absence of exchange splitting of the vanadium 3d bands observed in angle-resolved photoemission spectroscopy experiments was reported. This result contradicts other studies that revealed a magnetization value not higher than $5 \mu_B$.

Theoratically, based on these results we can conclude that the influence of the substrate is important for description of the magnetic properties of these materials. It has been proposed that the \textbf{presence of charge density waves} could cause the quenching of monolayer ferromagnetism due to the band gap opening induced by \textbf{Peierls distortion}. This modeling motivates us to study the interplay between magnetism and structural phase transitions in VSe2.

\begin{marginfigure}
    \includegraphics{./images/10.1039.c9cp03726h_AtomicStructureOfThe2DVSe2Monolayer}
	\caption[Atomic structure of the 2D VSe2 monolayer in the H phase and in the T phase.]{
        Atomic structure of the 2D VSe2 monolayer (top and side view) in the H phase (a) and in the T phase (b). Vanadium atoms are denoted with red circles, and the upper and bottom selenium layers are denoted with light green and dark green circles, respectively. The (c and d) Panels represent the corresponding spin-polarized band structures. The red lines correspond to spin up states and the black ones to spin down, the Fermi level corresponds to 0 eV.
	}
	\labfig{10.1039.c9cp03726h_AtomicStructureOfThe2DVSe2Monolayer}
\end{marginfigure}
\section{Methodology}
% \labsec{sec:Methodology}

% \subsection{Experiment}
% Substration: Bilayer graphene.

\subsection{DFT}
\begin{table}[h]
    \begin{tabular}{ccccc}
    \toprule
    KPOINTS                 & ENCUT  & Pseudopotential & Hubbard-U & Conver.      \\
    \midrule
    $10 \times 10 \times 1$ & 400 eV & DFT(PBE)-D2     & Unknown   & $10^{-6}$ eV \\
    \bottomrule
    \end{tabular}
\end{table}
Other: Vacuum thickness is 10 A. KPOINTS for bulk is $8 \times 8 \times 8$.
\section{Conclusion}
% \labsec{sec:Conclusion}
\begin{marginfigure}
    \includegraphics{./images/10.1039.c9cp03726h_SchematicVisualization}
	\caption[Schematic visualization of the plane and arc types of the Se atom rotation]{
        Schematic visualization of the plane (a and c) and arc (b and d) types of the Se atom rotation. The (a and b) and (c and d) panels correspond to side and top views, respectively. The initial and final positions of Se are presented with orange and green circles, respectively. The intermediate configurations of selenium atoms obtained with a 20 degree step are denoted with light blue circles.
	}
	\labfig{10.1039.c9cp03726h_SchematicVisualization}
\end{marginfigure}
\subsection{Structure}
The optimized atomic positions for the T-phase and lattice parameters a = b = 3.31 Å and c = 6.20 Å are in good agreement with experiment.In particular, the corresponding interlayer distance in bulk VSe2 is 3.04 Å. The calculated band structures of the VSe2 monolayer in the T and H phases are in good agreement with previous work.18 The calculated magnetic moment of 0.68 $\mu_B$ for the initial configuration without rotation of the selenium atoms also agrees with the results of previous work. Detailed band structure and atomic structure shows in \reffig{10.1039.c9cp03726h_AtomicStructureOfThe2DVSe2Monolayer}.

\subsection{Rotation Model}
This paper defines two types of rotation model that construct the routine from T phase to H phase, which could be seen in \reffig{10.1039.c9cp03726h_SchematicVisualization}(Monolayer structure). The phase transition can not be applied in plane model because of divergence of energy difference. The author calculates energy difference and magnetic moment during rotation from T $\rightarrow$ H phase within the arc model in \reffig{10.1039.c9cp03726h_Three-layerVSe2Systems}, which gives the H phase ground state in momolayer structure and T phase ground state in bulk structure.

\begin{figure}[ht] 
    \includegraphics[width=0.8\linewidth]{./images/10.1039.c9cp03726h_EvolutionOfTheTotalEnergyAndMmagneticMoment}
	\caption[Evolution of the total energy and magnetic moment during rotation for monolayer/bulk VSe2]{
		Evolution of the total energy (a) and magnetic moment (b) during rotation of thewhole upper Se layer of the VSe2 within the arc model.
	}
	\labfig{10.1039.c9cp03726h_EvolutionOfTheTotalEnergyAndMmagneticMoment}
\end{figure}

\subsection{Staking Effect}
\begin{marginfigure}
    \includegraphics{./images/10.1039.c9cp03726h_SchematicRepresentation}
	\caption[unit cells used for simulating VSe2 trilayers]{
        Schematic representation of the unit cells used for simulating VSe2 trilayers characterized by different stacking models.
	}
	\labfig{10.1039.c9cp03726h_SchematicRepresentation}
\end{marginfigure}

Different staking method in sevel layers VSe2 is shown in \reffig{10.1039.c9cp03726h_SchematicRepresentation}. The configuration of the H type corresponds to the structural ground state for all types of stacking in the few-layer case. The energy required for the transition from the T to the H phase is about 0.60 eV for AA- and AB-stacking in the bilayer. In the trilayer the most energetically favorable stacking orders are AAA and ABC (\reffig{10.1039.c9cp03726h_Three-layerVSe2Systems}).

\begin{figure}[ht] 
    \includegraphics[width=0.8\linewidth]{./images/10.1039.c9cp03726h_Three-layerVSe2Systems}
	\caption[Total energy and magnetic moment of two- and three-layer VSe2 systems]{
		Total energy (left panels) and magnetic moment (right panels) of two- (a and b) and three-layer (c and d) VSe2 systems estimated for H, T and intermediate structures.
	}
	\labfig{10.1039.c9cp03726h_Three-layerVSe2Systems}
\end{figure}

\subsection{Phonon dispersion}
The analysis of the calculated phonon dispersions in \reffig{10.1039.c9cp03726h_PhononDispersions} has demonstrated a principal role of ferromagnetism in stabilization of the atomic structure of the VSe2 monolayer in the H phase and similar systems.
\begin{figure}[ht] 
    \includegraphics[width=0.8\linewidth]{./images/10.1039.c9cp03726h_PhononDispersions}
	\caption[Phonon dispersions calculated for the nonmagnetic and the ferromagnetic state of monolayer and bulk VSe2.]{
		Phonon dispersions calculated for the nonmagnetic (red dashed line) and the ferromagnetic state (blue solid line) of monolayer and bulk VSe2. Both T and H phase structures are presented.
	}
	\labfig{10.1039.c9cp03726h_PhononDispersions}
\end{figure}
\section{Defect}
% \labsec{sec:Defect}
\begin{enumerate}
    \item The parameter selected for calculation is not accurate enough (ENCUT/KPOINTS). Reliance is questionable.
    \item We can't repeat this paper's conclusion.
\end{enumerate}

\section{Notes}
% \labsec{sec:Notes}
Try to repeat calculation, failed.
\setchapterpreamble[u]{\margintoc}
\chapter{Emergence of a Metal–Insulator Transition and High-Temperature Charge-Density Waves in VSe2 at the Monolayer Limit\cite{doi:10.1021/acs.nanolett.8b01764}}
% \labch{ch:SampleTitle}

\section{Summary Table}
% \labsec{sec:SummaryTable}

\begin{table}[h]
    \begin{tabular}{lccc}
    \hline
    Item  & Material         & Classification & Topic        \\  \hline
    Value & VSe2             & Experiment     & CDW+Phase Transition \\  \hline
    Item  & Range            & System         & Publish Year \\  \hline
    Value & Phase Transition & Monolayer+Bulk & 2018         \\  \hline
    \end{tabular}
\end{table}

\section{Abstract}
Emergent phenomena driven by electronic reconstructions in oxide heterostructures have been intensively discussed. However, the role of these phenomena in shaping the electronic properties in van der Waals heterointerfaces has hitherto not been established. By reducing the material thickness and forming a heterointerface, we find two types of charge-ordering transitions in monolayer VSe2 on graphene substrates. Angleresolved photoemission spectroscopy (ARPES) uncovers that Fermi-surface nesting becomes perfect in ML VSe2. Renormalization- group analysis confirms that imperfect nesting in three dimensions universally flows into perfect nesting in two dimensions. As a result, the charge-density wave-transition temperature is dramatically enhanced to a value of 350 K compared to the 105 K in bulk VSe2. More interestingly, ARPES and scanning tunneling microscopy measurements confirm an unexpected metal -insulator transition at 135 K that is driven by lattice distortions. The heterointerface plays an important role in driving this novel metal -insulator transition in the family of monolayer transition-metal dichalcogenides. 

\section{Background}
% \labsec{sec:Background}
Here, we report systematic studies on the electronic and atomic structures of ML VSe2 epitaxially grown on bilayer graphene (BLG) on silicon carbide (SiC) using both angleresolved
photoemission spectroscopy (ARPES) and scanning tunneling microscopy (STM). We observe the emergence of a \textbf{metal-insulator transition (MIT)} and a \textbf{high-temperature CDW phase}, associated with heterointerface coupling and reduced dimensionality, respectively. 

Temperature-dependent ARPES measurements reveal that \textbf{perfect FS nesting enhances the CDW transition temperature} , such that $T_{CDW} = 350 \pm 8$ K. Renormalization-group (RG) analysis confirms that the two dimensional (2D) nature of the ML VSe2 drives the perfect FS nesting. STM measurements confirm that the CDW order exists both at 300 and 79 K. We also observe an \textbf{unexpected MIT} with a transition temperature of $T_{MIT} = 135 \pm 10$ K driven by the \textbf{strong lattice distortion} of Se atoms. The lattice distortion is attributed to the dimerization of V atoms, which stabilizes the insulating phase.

\section{Methodology}
% \labsec{sec:Methodology}

\subsection{Experiment}
Experiment: MBE. \\
Substration: bilayer graphene (BLG) on SiC.

\subsection{Theory}
It turns out that perturbative RG analysis cannot be applied to the present problem in a straightforward manner. When there are FSs, these FS electrons are strongly correlated near quantum phase transitions in 2D, referred to as FS problems.33 Recently, the technique of “graphenization” has been proposed as a way to controllably evaluate Feynman diagrams in the FS problem. Based on this recently developed dimensional regularization technique, we perform the perturbative RG analysis, in which all interaction parameters are renormalized self-consistently.
% \begin{table}[h]
%     \begin{tabular}{ccccc}
%     \toprule
%     KPOINTS                 & ENCUT  & Pseudopotential & Hubbard-U & Conver.      \\
%     \midrule
%     $14 \times 14 \times 2$ & Unknown& PBE             & Unknown   & $10^{-6}$ eV \\
%     \bottomrule
%     \end{tabular}
% \end{table}

\section{Conclusion}
% \labsec{sec:Conclusion}

\begin{marginfigure}
    \includegraphics{./images/10.1021.acs.nanolett.8b01764_Morphology}
	\caption[Morphology of ML VSe2 on BLG]{
        Morphology of ML VSe2 on BLG. (a) Top- and side-view schematics of ML VSe2 with a BLG substrate where the green, purple, blue, and gray balls represent the top Se, bottom Se, V, and C atoms, respectively. (b) Topographic STM image of 0.9 ML VSe2 grown on BLG (Vb = -1.2 V, It = 40 pA). (c) Line profile along the arrow in panel b.
	}
	\labfig{10.1021.acs.nanolett.8b01764_Morphology}
\end{marginfigure}

The structure of VSe2 on BLG substration shows in \reffig{10.1021.acs.nanolett.8b01764_Morphology}. The line profile along the arrow in \reffig{10.1021.acs.nanolett.8b01764_Morphology}b shows that the apparent height of the VSe2 film is 6.9 Å, which agrees well with the unit cell height of bulk VSe2 (6.1 Å). Although the lattice mismatch between VSe2 and graphene is quite large, about 26.5\%, their crystal axes are aligned within the rotational misalignment $\theta \le \pm 5$°.

In \reffig{10.1021.acs.nanolett.8b01764_BandStructure}g, we track the electronic gap ($\Delta$) along the dashed line in \reffig{10.1021.acs.nanolett.8b01764_BandStructure}a, which reveals two types of electronic ordering phenomena in ML VSe2. Considering the partial gap opening near $\beta$ in the temperature range from 150 to 300 K and the strong FS nesting, it is suggested that the CDW phase exists above 300 K. At 135 K, the gap is fully opened for all values of k with a minimum gap size of $9 \pm 4$ meV at $\alpha$, directly indicating an MIT with $T_{MIT} = 135 \pm 10$ K. Our RG analysis suggests that the emergence of such perfect FS nesting is universal.

\begin{figure}[ht] 
    \includegraphics[width=0.8\linewidth]{./images/10.1021.acs.nanolett.8b01764_BandStructure}
	\caption[Band structure and temperature dependence of ML VSe2]{
		Band structure and temperature dependence of ML VSe2.
	}
	\labfig{10.1021.acs.nanolett.8b01764_BandStructure}
\end{figure}

Panels a and c of \reffig{10.1021.acs.nanolett.8b01764_STMAnalysis} show filled-state STM images of ML VSe2 obtained at 79 and 300 K, respectively. At a lower temperature of 79 K, the atomic structure drastically changes, giving rise to $\sqrt{3} \times 2$ and $\sqrt{3} \times \sqrt{7}$ superstructures. These observed distortions in \reffig{10.1021.acs.nanolett.8b01764_STMAnalysis}g are related to the formation of Se-Se dimers, which are laterally paired to be ~2.8 Å (12.5\% reduced) with respect to the undistorted Se-Se distance (3.2 Å, 300 K).

\begin{figure}[ht] 
    \includegraphics[width=0.8\linewidth]{./images/10.1021.acs.nanolett.8b01764_STMAnalysis}
	\caption[STM analysis of ML VSe2]{
		STM analysis of ML VSe2.
	}
	\labfig{10.1021.acs.nanolett.8b01764_STMAnalysis}
\end{figure}

\begin{marginfigure}
    \includegraphics{./images/10.1021.acs.nanolett.8b01764_SchematicPhaseDiagram}
	\caption[Schematic phase diagram for the electronic reconstruction of VSe2]{
        Schematic phase diagram for the electronic reconstruction of VSe2. (a) Schematic phase diagram in parameter space defined by thickness, lattice mismatch, and temperature based on both ARPES and STM measurements. Each phase contains the schematic FS model, where solid (dashed) contours correspond to the ungapped (gapped) section in one-sixth of the Brillouin zone. This phase diagram summarizes how strongly correlated electron physics in ML VSe2 emerges from that of weakly interacting electrons in 3D bulk VSe2, both by reducing the film thickness to the ML limit and by introducing a heterointerface with graphene. (b) The renormalization group flow diagram for the Fermion-boson interaction parameter (e) and both electron and order parameter velocities (v and c). All these parameters flow in a zero fixed-point value at low temperatures, which confirms the emergence of perfect FS nesting in 2D.
    	}
	\labfig{10.1021.acs.nanolett.8b01764_SchematicPhaseDiagram}
\end{marginfigure}

we propose a schematic phase diagram for the electronic reconstruction of VSe2 systems in the parameter plane describing film thickness and temperature(\reffig{10.1021.acs.nanolett.8b01764_SchematicPhaseDiagram}) When the sample dimensionality is reduced from 3D to 2D, the weakly nested FS (phase i) is transformed into the perfectly nested FS with elongated parallel sides (phase ii), resulting in a significant increase of TCDW. Lowering the temperature induces partial gap opening in the nested sections (red dashed lines) of the FS for both the 3D (phase iii) and the 2D (phase iv) CDW phases, while their residual parts (blue solid lines) remain ungapped. The further introduction of interfacial effects in the 2D heterostructure, such as a lattice mismatch, opens a complete gap in the FS (phase v, red and blue dashed lines), indicative of an MIT.

\section{Defect}
% \labsec{sec:Defect}
This paper focus on CDW properties of ML VSe2, not so much magnetic moments are involved.
% \begin{enumerate}
%     \item item1.
%     \item item2.
% \end{enumerate}

\section{Notes}
% \labsec{sec:Notes}
The ML VSe2 in this research is ferromagnetic. The magnetic hysteresis signal shows in \reffig{10.1021.acs.nanolett.8b01764_MagneticMeasurements}.
\begin{figure}
    \includegraphics[width=0.8\linewidth]{./images/10.1021.acs.nanolett.8b01764_MagneticMeasurements}
	\caption[Magnetic measurements of 1.5 ML VSe2]{
		Magnetic measurements of 1.5 ML VSe2.
	}
	\labfig{10.1021.acs.nanolett.8b01764_MagneticMeasurements}
\end{figure}

% \appendix % From here onwards, chapters are numbered with letters, as is the appendix convention

% \pagelayout{wide} % No margins
% \addpart{Appendix}
% \pagelayout{margin} % Restore margins

% \setchapterstyle{lines}
\labch{ch:appendix}

\chapter{Source Code}

\section{Supercell Construction}
\labsec{sec:generateSupercell}

\begin{remark}
    This source code is MATLAB format under Version R2019b/R2020a, other version is not tested. The output "POSCAR" file can not be used in VASP directly because this file DO NOT remove duplicated atoms. It should be imported in VESTA, remove duplicated atoms(built-in function) and then export.
\end{remark}

\begin{lstlisting}[language=Matlab]
    %% Determine the supercell length
    clear variables;
    nMax = 40;
    mMax = 40;
    a = 3.3171310374808005;
    cutMaxA = 10* a;
    
    theta1 = zeros(nMax, mMax);
    theta2 = zeros(nMax, mMax);
    theta3 = zeros(nMax, mMax);
    theta4 = zeros(nMax, mMax);
    length1 = zeros(nMax, mMax);
    length2 = zeros(nMax, mMax);
    length3 = zeros(nMax, mMax);
    length4 = zeros(nMax, mMax);
    
    for numIdxN = 1: nMax
        for numIdxM = 1: mMax
            theta1(numIdxN, numIdxM) = 2*atand((2*numIdxN - 1)/(sqrt(3)* (2*numIdxM - 1)));
            theta2(numIdxN, numIdxM) = 2*atand(numIdxN/(sqrt(3)* numIdxM));
            theta3(numIdxN, numIdxM) = 2*atand((sqrt(3)* (2*numIdxM - 1))/(2*numIdxN - 1));
            theta4(numIdxN, numIdxM) = 2*atand((sqrt(3)* numIdxM)/numIdxN);
            length1(numIdxN, numIdxM) = a/2 * sqrt((2*numIdxN - 1)^2 + 3*(2*numIdxM - 1)^2);
            length2(numIdxN, numIdxM) = a* sqrt(numIdxN^2 + 3* numIdxM^2);
        end
    end
    length3 = length1;
    length4 = length2;
    
    [sortedTheta1, sortedIndex1] = sort(theta1(:));
    [sortedn1, sortedm1] = ind2sub(size(theta1), sortedIndex1);
    [sortedTheta2, sortedIndex2] = sort(theta2(:));
    [sortedn2, sortedm2] = ind2sub(size(theta2), sortedIndex2);
    
    % figure();
    % hold on;
    % scatter(sortedTheta1, length1(sortedIndex1), 5, 'b');
    % scatter(sortedTheta2, length2(sortedIndex2), 5, 'r');
    % hold off;
    
    thetaMix = [theta1, theta2, theta3, theta4];
    lengthMix = [length1, length2, length3, length4];
    [sortedThetaMix, sortedIndexMix] = sort(thetaMix(:));
    [sortednMix, sortedmMix] = ind2sub(size(thetaMix), sortedIndexMix);
    while ~prod(sortedmMix <= 20)
        sortedmMix(sortedmMix > 20) = sortedmMix(sortedmMix > 20) - 20;
    end
    sortedMixType = nan(size(thetaMix));
    sortedMixType = 1*(sortedIndexMix(:) <= (numel(thetaMix)/4)) + 2*(sortedIndexMix(:) > (numel(thetaMix)/4) & sortedIndexMix(:) <= (numel(thetaMix)/2)) ...
        + 3* (sortedIndexMix(:) > (numel(thetaMix)/2) & sortedIndexMix(:) <= (numel(thetaMix)*3/4)) + 4*(sortedIndexMix(:) > (numel(thetaMix)*3/4));
    
    % figure;
    % scatter(sortedThetaMix, lengthMix(sortedIndexMix), 5);
    
    [thetaMixUnique, sortedIndexMixUnique] = uniquetol(thetaMix);
    lengthMixUnique = nan(size(thetaMixUnique));
    sortedmMixUnique = nan(size(thetaMixUnique));
    sortednMixUnique = nan(size(thetaMixUnique));
    sortedMixTypeUnique = nan(size(thetaMixUnique));
    for numIdx = 1: length(thetaMixUnique)
        searchIndex = abs(thetaMixUnique(numIdx) -  sortedThetaMix) < 1e-8;
        lengthMixUnique(numIdx) = min(lengthMix(sortedIndexMix(searchIndex)));
        tmpIndex = searchIndex & abs(lengthMix(sortedIndexMix) - lengthMixUnique(numIdx))< 1e-8;
        sortedmMixUnique(numIdx) = sortedmMix(tmpIndex);
        sortednMixUnique(numIdx) = sortednMix(tmpIndex);
        sortedMixTypeUnique(numIdx) = sortedMixType(tmpIndex);
    end
    
    % figure;
    % scatter(thetaMixUnique, lengthMixUnique, 5);
    % plot(thetaMixUnique, lengthMixUnique);
    
    cutIndex = lengthMixUnique < cutMaxA;
    thetaMixUniqueCut = thetaMixUnique(cutIndex);
    lengthMixUniqueCut = lengthMixUnique(cutIndex);
    sortedmMixUniqueCut = sortedmMixUnique(cutIndex);
    sortednMixUniqueCut = sortednMixUnique(cutIndex);
    sortedMixTypeUniqueCut = sortedMixTypeUnique(cutIndex);
    
    figure;
    scatter(thetaMixUniqueCut, lengthMixUniqueCut, 5);
    xlim([0, 60]);
    
    % Clear variables
    clear length1 length2 length3 length4 lengthMix theta1 theta2 theta3 theta4 thetaMix
    clear cutMaxA mMax nMax numIdx numIdxM numIdxN tmpIndex cutIndex
    clear lengthMixUnique searchIndex sortedmMix sortedMixType sortedMixTypeUnique sortedIndexMix sortedmMix sortedmMixUnique
    clear sortedIndexMixUnique sortednMix sortednMixUnique sortedThetaMix thetaMixUnique
    %% Generates the lattice
    % a = 3.441;  % a-axies of the lattice
    b = a;      % b-axies of the lattice
    gamma = 120;% angle of <a, b>
    sizeLattice = 21;  % Scale of the system (should be odd)
    
    % Initialize of the variables
    centerOrder = (sizeLattice + 1) ./ 2;
    lattice.x = zeros(sizeLattice);
    lattice.y = zeros(sizeLattice);
    % Difference in x and y in the primitive cell
    deltaA = [a, 0];
    deltaB = [b*cosd(gamma), b*sind(gamma)];
    % Set the zero of the axies
    lattice.x(sizeLattice, 1) = 0;
    lattice.y(sizeLattice, 1) = 0;
    % Initialize the position of the center
    lattice.x(centerOrder, centerOrder) = ((centerOrder - 1)*deltaA(1) + (sizeLattice - centerOrder)*deltaB(1));
    lattice.y(centerOrder, centerOrder) = ((centerOrder - 1)*deltaA(2) + (sizeLattice - centerOrder)*deltaB(2));
    % Calculate the distance from the center
    for row = sizeLattice: -1: 1
        for column = 1: sizeLattice
            lattice.x(row, column) = ((column - 1)*deltaA(1) + (sizeLattice - row)*deltaB(1));
            lattice.y(row, column) = ((column - 1)*deltaA(2) + (sizeLattice - row)*deltaB(2));
            lattice.hoppingA(row, column) = column - centerOrder;
            lattice.hoppingB(row, column) = centerOrder - row;
        end
    end
    % Move the center to the axis zero
    lattice.x = lattice.x - lattice.x(centerOrder, centerOrder);
    lattice.y = lattice.y - lattice.y(centerOrder, centerOrder);
    
    %% Rotation
    % Construct layers
    d = 1.58106107700766;   % h-phase:1.59498653898723, t-phase:1.58106107700766
    D = 2.809105554065346;  % h-phase:3.67176692979057, t-phase:3.12122839340594
                            % Origin: 3.12122839340594
    vacuumLength = 20;
    numberOfLayers = 2;
    rotationNumIdx = 23;
    % for rotationNumIdx = 13: 13
    rotationDegree = thetaMixUniqueCut(rotationNumIdx)/2;
    supercellLattice = lengthMixUniqueCut(rotationNumIdx);
    supercellBoundx = [0, supercellLattice, supercellLattice/2, -supercellLattice/2, 0];
    supercellBoundy = [0, 0, sqrt(3)/2 * supercellLattice, sqrt(3)/2 * supercellLattice, 0];
    offsetSe1 = [deltaA', deltaB']*[2/3; 1/3];
    % offsetSe1 = [deltaA', deltaB']*[1/3; 2/3];
    offsetSe2 = [deltaA', deltaB']*[1/3; 2/3];
    layer{1}.Se1.x = lattice.x + offsetSe1(1);
    layer{1}.Se1.y = lattice.y + offsetSe1(2);
    layer{1}.Se1.z = zeros(size(lattice.x));
    layer{1}.V.x = lattice.x;
    layer{1}.V.y = lattice.y;
    layer{1}.V.z = zeros(size(lattice.x)) + d;
    layer{1}.Se2.x = lattice.x + offsetSe2(1);
    layer{1}.Se2.y = lattice.y + offsetSe2(2);
    layer{1}.Se2.z = zeros(size(lattice.x)) + 2*d;
    
    layer{2} = layer{1};
    layer{2}.Se1.z = layer{1}.Se1.z + D + 2*d;
    layer{2}.Se2.z = layer{1}.Se2.z + D + 2*d;
    layer{2}.V.z = layer{1}.V.z + D + 2*d;
    
    % Rotation Process
    for i = 1: numberOfLayers
        % layer0: counterclock; layer1: clock
        switch i
            case 1
                rotationMatrix = ...
                    [cosd(rotationDegree), -sind(rotationDegree), 0; sind(rotationDegree), cosd(rotationDegree), 0; 0 0 1];
            case 2
                rotationMatrix = ...
                    [cosd(-rotationDegree), -sind(-rotationDegree), 0; sind(-rotationDegree), cosd(-rotationDegree), 0; 0 0 1];
        end
        switch sortedMixTypeUniqueCut(rotationNumIdx)
            case {1, 2}
                correctMatrix = [cosd(-30), -sind(-30), 0; sind(-30), cosd(-30), 0; 0 0 1];
            case {3, 4}
                correctMatrix = [1 0 0; 0 1 0; 0 0 1];
        end
        for numIdx = 1: numel(layer{i}.V.x)
            tmpMat = [layer{i}.Se1.x(numIdx), layer{i}.Se1.y(numIdx), layer{i}.Se1.z(numIdx)]';
            tmpMat = correctMatrix * rotationMatrix * tmpMat;
            layer{i}.Se1.x(numIdx) = tmpMat(1);
            layer{i}.Se1.y(numIdx) = tmpMat(2);
            layer{i}.Se1.z(numIdx) = tmpMat(3);
            
            tmpMat = [layer{i}.Se2.x(numIdx), layer{i}.Se2.y(numIdx), layer{i}.Se2.z(numIdx)]';
            tmpMat = correctMatrix * rotationMatrix * tmpMat;
            layer{i}.Se2.x(numIdx) = tmpMat(1);
            layer{i}.Se2.y(numIdx) = tmpMat(2);
            layer{i}.Se2.z(numIdx) = tmpMat(3);
            
            tmpMat = [layer{i}.V.x(numIdx), layer{i}.V.y(numIdx), layer{i}.V.z(numIdx)]';
            tmpMat = correctMatrix * rotationMatrix * tmpMat;
            layer{i}.V.x(numIdx) = tmpMat(1);
            layer{i}.V.y(numIdx) = tmpMat(2);
            layer{i}.V.z(numIdx) = tmpMat(3);
        end
        % Cut the supercell
        layer{i}.V.inSupercell = inpolygon(layer{i}.V.x, layer{i}.V.y, supercellBoundx, supercellBoundy);
        layer{i}.Se1.inSupercell = inpolygon(layer{i}.Se1.x, layer{i}.Se1.y, supercellBoundx, supercellBoundy);
        layer{i}.Se2.inSupercell = inpolygon(layer{i}.Se2.x, layer{i}.Se2.y, supercellBoundx, supercellBoundy);
    end
    
    % 
    % fig = figure;
    % hold on;
    % % scatter3(layer{1}.Se1.x(:), layer{1}.Se1.y(:), layer{1}.Se1.z(:), 'green');
    % scatter3(layer{1}.V.x(:), layer{1}.V.y(:), layer{1}.V.z(:), 'red');
    % % scatter3(layer{1}.Se2.x(:), layer{1}.Se2.y(:), layer{1}.Se2.z(:), 'green');
    % % scatter3(layer{2}.Se1.x(:), layer{2}.Se1.y(:), layer{2}.Se1.z(:), 'green');
    % scatter3(layer{2}.V.x(:), layer{2}.V.y(:), layer{2}.V.z(:), 'blue');
    % % scatter3(layer{2}.Se2.x(:), layer{2}.Se2.y(:), layer{2}.Se2.z(:), 'green');
    % hold off
    % maxLim = max(abs([fig.Children.XLim, fig.Children.YLim]));
    % fig.Children.XLim = [-maxLim, maxLim];
    % fig.Children.YLim = [-maxLim, maxLim];
    
    fig = figure;
    hold on;
    % scatter3(layer{1}.Se1.x(:), layer{1}.Se1.y(:), layer{1}.Se1.z(:), 'green');
    scatter3(layer{1}.V.x(layer{1}.V.inSupercell), layer{1}.V.y(layer{1}.V.inSupercell), layer{1}.V.z(layer{1}.V.inSupercell), 'red');
    % scatter3(layer{1}.Se2.x(:), layer{1}.Se2.y(:), layer{1}.Se2.z(:), 'green');
    % scatter3(layer{2}.Se1.x(:), layer{2}.Se1.y(:), layer{2}.Se1.z(:), 'green');
    scatter3(layer{2}.V.x(layer{2}.V.inSupercell), layer{2}.V.y(layer{2}.V.inSupercell), layer{2}.V.z(layer{2}.V.inSupercell), 'blue');
    % scatter3(layer{2}.Se2.x(:), layer{2}.Se2.y(:), layer{2}.Se2.z(:), 'green');
    hold off
    maxLim = max(abs([fig.Children.XLim, fig.Children.YLim]));
    fig.Children.XLim = [-maxLim, maxLim];
    fig.Children.YLim = [-maxLim, maxLim];
    
    % Calculate POSCAR
    supercellC = vacuumLength + 4*d + D;
    VatomNumber = 0;
    SeatomNumber = 0;
    superCell.V.x = layer{1}.V.x(layer{1}.V.inSupercell);
    superCell.V.y = layer{1}.V.y(layer{1}.V.inSupercell);
    superCell.V.z = layer{1}.V.z(layer{1}.V.inSupercell);
    superCell.Se.x = cat(1, layer{1}.Se1.x(layer{1}.Se1.inSupercell), layer{1}.Se2.x(layer{1}.Se2.inSupercell));
    superCell.Se.y = cat(1, layer{1}.Se1.y(layer{1}.Se1.inSupercell), layer{1}.Se2.y(layer{1}.Se2.inSupercell));
    superCell.Se.z = cat(1, layer{1}.Se1.z(layer{1}.Se1.inSupercell), layer{1}.Se2.z(layer{1}.Se2.inSupercell));
    for i = 1: numberOfLayers
        VatomNumber = VatomNumber + length(layer{i}.V.x(layer{i}.V.inSupercell));
        SeatomNumber = SeatomNumber + length(layer{i}.Se1.x(layer{i}.Se1.inSupercell)) + length(layer{i}.Se2.x(layer{i}.Se2.inSupercell));
        if i >= 2
            superCell.V.x = cat(1, superCell.V.x, layer{i}.V.x(layer{i}.V.inSupercell));
            superCell.V.y = cat(1, superCell.V.y, layer{i}.V.y(layer{i}.V.inSupercell));
            superCell.V.z = cat(1, superCell.V.z, layer{i}.V.z(layer{i}.V.inSupercell));
            superCell.Se.x = cat(1, superCell.Se.x, layer{i}.Se1.x(layer{i}.Se1.inSupercell), layer{i}.Se2.x(layer{i}.Se2.inSupercell));
            superCell.Se.y = cat(1, superCell.Se.y, layer{i}.Se1.y(layer{i}.Se1.inSupercell), layer{i}.Se2.y(layer{i}.Se2.inSupercell));
            superCell.Se.z = cat(1, superCell.Se.z, layer{i}.Se1.z(layer{i}.Se1.inSupercell), layer{i}.Se2.z(layer{i}.Se2.inSupercell));
        end
    end
    % Write POSCAR
    fileId = fopen("POSCAR", 'w');
    fprintf(fileId, "Degree: %.5f\n", thetaMixUniqueCut(rotationNumIdx));
    fprintf(fileId, "   1.00000000000000 \n");
    fprintf(fileId, "%24.15f %24.15f %24.15f\n", [supercellLattice, 0, 0]);
    fprintf(fileId, "%24.15f %24.15f %24.15f\n", [-supercellLattice/2, sqrt(3)/2*supercellLattice, 0]);
    fprintf(fileId, "%24.15f %24.15f %24.15f\n", [0, 0, supercellC]);
    fprintf(fileId, "   V    Se\n");
    fprintf(fileId, "%6d %6d\n", [VatomNumber, SeatomNumber]);
    fprintf(fileId, "Cartesian\n");
    for numIdx = 1: VatomNumber
        fprintf(fileId, "% 18.16f % 21.16f % 21.16f\n", [superCell.V.x(numIdx), superCell.V.y(numIdx), superCell.V.z(numIdx)]);
    end
    
    for numIdx = 1: SeatomNumber
        fprintf(fileId, "% 18.16f % 21.16f % 21.16f\n", [superCell.Se.x(numIdx), superCell.Se.y(numIdx), superCell.Se.z(numIdx)]);
    end
    fclose(fileId);
    % end    
\end{lstlisting}

\section{Primitive cell}
\labsec{sec:primitiveCell}


\begin{lstlisting}[style=kaolstplain, linewidth=1.5\textwidth]
V1 Se2 T-Phase Structure
    1.00000000000000
      3.3171310374808005    0.0000000000021438   -0.0000000000000300
     -1.6585655187866546    2.8727197461676814    0.0000000000000371
     -0.0000000000002724    0.0000000000001877   27.2846672137727637
    V    Se
      2     4
 Direct
  -0.0000000000000002  0.0000000000000003  0.0456342266658590       
   0.0000000000000002 -0.0000000000000003  0.2751370207888941       
   0.6666666746654853  0.3333333495243738 -0.0126915455264852       
   0.3333333497015403  0.6666666993837396  0.1031743004030291       
   0.6666666746654855  0.3333333495243735  0.2175692596287150       
   0.3333333497015404  0.6666666993837395  0.3334908956017944       
 
   0.00000000E+00  0.00000000E+00  0.00000000E+00
   0.00000000E+00  0.00000000E+00  0.00000000E+00
   0.00000000E+00  0.00000000E+00  0.00000000E+00
   0.00000000E+00  0.00000000E+00  0.00000000E+00
   0.00000000E+00  0.00000000E+00  0.00000000E+00
   0.00000000E+00  0.00000000E+00  0.00000000E+00
\end{lstlisting}

%----------------------------------------------------------------------------------------

\backmatter % Denotes the end of the main document content
\setchapterstyle{plain} % Output plain chapters from this point onwards

%----------------------------------------------------------------------------------------
%	BIBLIOGRAPHY
%----------------------------------------------------------------------------------------

% The bibliography needs to be compiled with biber using your LaTeX editor, or on the command line with 'biber main' from the template directory

\defbibnote{bibnote}{Here are the references in citation order.\par\bigskip} % Prepend this text to the bibliography
\printbibliography[heading=bibintoc, title=Bibliography, prenote=bibnote] % Add the bibliography heading to the ToC, set the title of the bibliography and output the bibliography note

%----------------------------------------------------------------------------------------
%	NOMENCLATURE
%----------------------------------------------------------------------------------------

% The nomenclature needs to be compiled on the command line with 'makeindex main.nlo -s nomencl.ist -o main.nls' from the template directory

% \nomenclature{$c$}{Speed of light in a vacuum inertial frame}
% \nomenclature{$h$}{Planck constant}

% \renewcommand{\nomname}{Notation} % Rename the default 'Nomenclature'
% \renewcommand{\nompreamble}{The next list describes several symbols that will be later used within the body of the document.} % Prepend this text to the nomenclature

% \printnomenclature % Output the nomenclature

%----------------------------------------------------------------------------------------
%	GREEK ALPHABET
% 	Originally from https://gitlab.com/jim.hefferon/linear-algebra
%----------------------------------------------------------------------------------------

% \vspace{1cm}

% {\usekomafont{chapter}Greek Letters with Pronunciations} \\[2ex]
% \begin{center}
% 	\newcommand{\pronounced}[1]{\hspace*{.2em}\small\textit{#1}}
% 	\begin{tabular}{l l @{\hspace*{3em}} l l}
% 		\toprule
% 		Character & Name & Character & Name \\ 
% 		\midrule
% 		$\alpha$ & alpha \pronounced{AL-fuh} & $\nu$ & nu \pronounced{NEW} \\
% 		$\beta$ & beta \pronounced{BAY-tuh} & $\xi$, $\Xi$ & xi \pronounced{KSIGH} \\ 
% 		$\gamma$, $\Gamma$ & gamma \pronounced{GAM-muh} & o & omicron \pronounced{OM-uh-CRON} \\
% 		$\delta$, $\Delta$ & delta \pronounced{DEL-tuh} & $\pi$, $\Pi$ & pi \pronounced{PIE} \\
% 		$\epsilon$ & epsilon \pronounced{EP-suh-lon} & $\rho$ & rho \pronounced{ROW} \\
% 		$\zeta$ & zeta \pronounced{ZAY-tuh} & $\sigma$, $\Sigma$ & sigma \pronounced{SIG-muh} \\
% 		$\eta$ & eta \pronounced{AY-tuh} & $\tau$ & tau \pronounced{TOW (as in cow)} \\
% 		$\theta$, $\Theta$ & theta \pronounced{THAY-tuh} & $\upsilon$, $\Upsilon$ & upsilon \pronounced{OOP-suh-LON} \\
% 		$\iota$ & iota \pronounced{eye-OH-tuh} & $\phi$, $\Phi$ & phi \pronounced{FEE, or FI (as in hi)} \\
% 		$\kappa$ & kappa \pronounced{KAP-uh} & $\chi$ & chi \pronounced{KI (as in hi)} \\
% 		$\lambda$, $\Lambda$ & lambda \pronounced{LAM-duh} & $\psi$, $\Psi$ & psi \pronounced{SIGH, or PSIGH} \\
% 		$\mu$ & mu \pronounced{MEW} & $\omega$, $\Omega$ & omega \pronounced{oh-MAY-guh} \\
% 		\bottomrule
% 	\end{tabular} \\[1.5ex]
% 	Capitals shown are the ones that differ from Roman capitals.
% \end{center}

%----------------------------------------------------------------------------------------
%	GLOSSARY
%----------------------------------------------------------------------------------------

% The glossary needs to be compiled on the command line with 'makeglossaries main' from the template directory

% \setglossarystyle{listgroup} % Set the style of the glossary (see https://en.wikibooks.org/wiki/LaTeX/Glossary for a reference)
% \printglossary[title=Special Terms, toctitle=List of Terms] % Output the glossary, 'title' is the chapter heading for the glossary, toctitle is the table of contents heading

%----------------------------------------------------------------------------------------
%	INDEX
%----------------------------------------------------------------------------------------

% The index needs to be compiled on the command line with 'makeindex main' from the template directory

\printindex % Output the index

%----------------------------------------------------------------------------------------
%	BACK COVER
%----------------------------------------------------------------------------------------

% If you have a PDF/image file that you want to use as a back cover, uncomment the following lines

%\clearpage
%\thispagestyle{empty}
%\null%
%\clearpage
%\includepdf{cover-back.pdf}

%----------------------------------------------------------------------------------------

\end{document}
