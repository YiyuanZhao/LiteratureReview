ewwew\setchapterpreamble[u]{\margintoc}
\chapter{Magnetic Transition in Monolayer VSe2 via Interface Hybridization\cite{doi:10.1021/acsnano.9b02996}}
% \labch{ch:SampleTitle}

\section{Summary Table}
% \labsec{sec:SummaryTable}

\begin{table}[h]
    \begin{tabular}{lccc}
    \hline
    Item  & Material         & Classification & Topic        \\  \hline
    Value & VSe2             & Experiment+DFT & Magnetic     \\  \hline
    Item  & Range            & System         & Publish Year \\  \hline
    Value & Magmom           & Monolayer      & 2019         \\  \hline
    \end{tabular}
\end{table}

\section{Abstract}
Magnetism in monolayer (ML) VSe2 has attracted broad interest in spintronics, while existing reports have not reached consensus. Using element-specific X-ray magnetic circular dichroism, a magnetic transition in ML VSe2 has been demonstrated at the contamination-free interface between Co and VSe2. Through interfacial hybridization with a Co atomic overlayer, a magnetic moment of about 0.4 $\mu_B$ per V atom in ML VSe2 is revealed, approaching values predicted by previous theoretical calculations. Promotion of the ferromagnetism in ML VSe2 is accompanied by its antiferromagnetic coupling to Co and a reduction in the spin moment of Co. In comparison to the absence of this interface-induced ferromagnetism at the Fe/ML MoSe2 interface, these findings at the Co/ML VSe2 interface provide clear proof that the ML VSe2, initially with magnetic disorder, is on the verge of magnetic transition.

\section{Background}
% \labsec{sec:Background}
Bonilla et al. indicated that a ferromagnetic state persists up to room temperature in ML VSe2, but their report has several critical unexplained features. Their most debatable finding is the reported magnetization of $15 \mu_B$ per V atom, which disagrees with the density functional theory (DFT) calculations for ML VSe2.

Subsequently, we obtained the \textbf{first experimental evidence of spin frustration in ML VSe2}, which leads to a highly degenerate ground state and forbids magnetic ordering. Although lacking long-range magnetic order in such a frustrated magnet, perturbations that break the symmetry of the lattice could possibly break the ground-state degeneracy and lead to magnetic order.

\section{Methodology}
% \labsec{sec:Methodology}

\subsection{Experiment}
Substration: HOPG. Method: MBE. Temperature: 550 C.

\subsection{DFT}
\begin{table}[h]
    \begin{tabular}{cccc}
    \toprule
    KPOINTS                 & ENCUT  & Pseudopotential & Hubbard-U \\
    \midrule
    $15 \times 15 \times 1$ & 500 eV & PBE             & Unapplied  \\
    \bottomrule
    \end{tabular}
\end{table}
Vacuum thickness: 12 A, convergence criterion is $10^{-5}$.
\section{Conclusion}
% \labsec{sec:Conclusion}

\begin{figure}[ht] 
    \includegraphics[width=0.8\linewidth]{./images/10.1021.acsnano.9b02996_STMTopographyOfMLVSe2}
	\caption[STM topography and XA/XMCD spectroscopies of ML VSe2]{
        STM topography and XA/XMCD spectroscopies of ML VSe2.
	}
	\labfig{10.1021.acsnano.9b02996_STMTopographyOfMLVSe2}
\end{figure}

Step height is ~7 A. XA spectroscopy at the V $L_{2,3}$ edge shows in \reffig{10.1021.acsnano.9b02996_STMTopographyOfMLVSe2}b. These spectral features are akin to those measured from other $3d^1$ vanadium compounds, thus providing a spectroscopic fingerprint of the \textbf{1T phase of the monolayer}. XMCD measurement at the V $L_{2,3}$ edge shows in \reffig{10.1021.acsnano.9b02996_STMTopographyOfMLVSe2}c. The XMCD contrast, within the scale of experimental error, is negligible and thus \textbf{rules out the existence of intrinsic ferromagnetism in the monolayer}, which demonstrating ML VSe2 as a frustrated magnet, in which its spins exhibit subtle correlations albeit in the absence of a long-range magnetic order.

As shown in \reffig{10.1021.acsnano.9b02996_MicroscopicMagneticMoments}, the orbital moment of V, mL,V, carries the opposite sign as the spin moment, mS,V, in agreement with Hund’s rule for a less than half-filled 3d shell. Moreover, the resulting total moment of V, mtot,$V = m_{L,V} + m_{S,V}$, has an opposite sign to that of Co (mtot,Co), confirming an antiferromagnetic coupling between Co and ML VSe2.

\begin{figure}[ht] 
    \includegraphics[width=0.8\linewidth]{./images/10.1021.acsnano.9b02996_MicroscopicMagneticMoments}
	\caption[Microscopic Magnetic Moments at the Co/ML VSe2 Interface]{
        Microscopic Magnetic Moments at the Co/ML VSe2 Interface.
	}
	\labfig{10.1021.acsnano.9b02996_MicroscopicMagneticMoments}
\end{figure}

Through interfacial hybridization with a Co atomic overlayer, a magnetic moment of about $0.4 \mu_B$ per V atom in ML VSe2 is revealed, approaching values predicted by previous theoretical calculations. Promotion of the ferromagnetism in ML VSe2 is accompanied by its antiferromagnetic coupling to Co and a reduction in the spin moment of Co (See in \reffig{10.1021.acsnano.9b02996_XMCDHysteresisLoops}). 

\begin{marginfigure}
    \includegraphics{./images/10.1021.acsnano.9b02996_XMCDHysteresisLoops}
	\caption[XMCD hysteresis loops obtained at Co and V $L_3$ edge at 65 K]{
        XMCD hysteresis loops obtained at Co and V $L_3$ edge at 65 K, which confirms the antiparallel alignment, i.e., antiferromagnetic coupling, between Co and V spins.
	}
	\labfig{10.1021.acsnano.9b02996_XMCDHysteresisLoops}
\end{marginfigure}

In comparison to the absence of this interface-induced ferromagnetism at the Fe/ML MoSe2 interface, these findings at the Co/ML VSe2 interface provide clear proof that the ML VSe2, initially with magnetic disorder, is \textbf{on the verge of magnetic transition}.

\section{Defect}
% \labsec{sec:Defect}
\begin{enumerate}
    \item The accuracy of DFT calculation is not high, as well as the density of KPOINTS.
    \item The machanism of magnetization is still unclear.
\end{enumerate}

\section{Notes}
% \labsec{sec:Notes}
This paper believes that ML VSe2 is nonmagnetic, while Co/ML VSe2 is ferromagnetic.