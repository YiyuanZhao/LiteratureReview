\setchapterpreamble[u]{\margintoc}
\chapter{Dimensional crossover of the charge density wave transition in thin exfoliated VSe2\cite{P_sztor_2017}}
% \labch{ch:SampleTitle}

\section{Summary Table}
% \labsec{sec:SummaryTable}

\begin{table}[h]
    \begin{tabular}{lccc}
    \hline
    Item  & Material         & Classification & Topic        \\  \hline
    Value & VSe2             & Experiment     & Magnetic+CDW \\  \hline
    Item  & Range            & System         & Publish Year \\  \hline
    Value & Structure        & Monolayer      & 2018         \\  \hline
    \end{tabular}
\end{table}

\section{Abstract}
Isolating single unit-cell thin layers from the bulk matrix of layered compounds offers tremendous opportunities to design novel functional electronic materials. However, a comprehensive thickness dependence study is paramount to harness the electronic properties of such atomic foils and their stacking into synthetic heterostructures. Here we show that a dimensional crossover and quantum confinement with reducing thickness result in a striking non-monotonic evolution of the charge density wave transition temperature in VSe2. Our conclusion is drawn from a direct derivation of the local order parameter and transition temperature from the real space charge modulation amplitude imaged by scanning tunnelling microscopy. This study lifts the disagreement of previous independent transport measurements. We find that thickness can be a non-trivial tuning parameter and demonstrate the importance of considering a finite thickness range to accurately characterize its influence.

\section{Background}
% \labsec{sec:Background}
This article focus on the unsolved problem that \dots

\section{Methodology}
% \labsec{sec:Methodology}

\subsection{Experiment}
Substration: Bilayer graphene.

\subsection{DFT}
\begin{table}[h]
    \begin{tabular}{ccccc}
    \toprule
    KPOINTS                 & ENCUT  & Pseudopotential & Hubbard-U & Conver.      \\
    \midrule
    $14 \times 14 \times 2$ & Unknown& PBE             & Unknown   & $10^{-6}$ eV \\
    \bottomrule
    \end{tabular}
\end{table}

\section{Conclusion}
% \labsec{sec:Conclusion}
External pressure has great impact on the ground state of TMDCs.

\begin{marginfigure}
    \includegraphics{./images/10.1021.acsnano.9b02996_XMCDHysteresisLoops}
	\caption[XMCD hysteresis loops obtained at Co and V $L_3$ edge at 65 K]{
        XMCD hysteresis loops obtained at Co and V $L_3$ edge at 65 K, which confirms the antiparallel alignment, i.e., antiferromagnetic coupling, between Co and V spins.
	}
	\labfig{10.1021.acsnano.9b02996_XMCDHysteresisLoops}
\end{marginfigure}

\begin{figure}[ht] 
    \includegraphics[width=0.8\linewidth]{./images/10.1039.c8nr09258c_TEMImageOfVSe2Nanosheet}
	\caption[TEM image of a half-hexagonal VSe2 nanosheet on a Cu grid]{
		TEM image of a half-hexagonal VSe2 nanosheet on a Cu grid.
	}
	\labfig{fig:10.1039.c8nr09258c_TEMImageOfVSe2Nanosheet}
\end{figure}

\section{Defect}
% \labsec{sec:Defect}
The measured data in this paper is contrast with the other paper. Mean field approximation may not be adequate to describe interactions.
\begin{enumerate}
    \item item1.
    \item item2.
\end{enumerate}

\section{Notes}
% \labsec{sec:Notes}
We have some great notes here.