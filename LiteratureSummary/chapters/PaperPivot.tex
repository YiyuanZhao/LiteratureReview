\setchapterpreamble[u]{\margintoc}
\chapter{General TMDCs}
% \labch{ch:GeneralTMDCs}

\section{Ground State Determination}
% \labsec{sec:GroundStateDetermination}
\subsection{Experiment Facts}
Here're some expressions that describes goundstate of TMDCs.

\subsection{DFT Calculation}
Here're some expressions that describes goundstate of TMDCs.

\section{Magnetic Properties}
% \labsec{sec:MagneticProperties}
The \refthm{MerminWagnerTheory}(Mermin-Wagner theory) shows that a one- or two-dimensional isotropic spin-8 Heisenberg model with finite-range exchange interaction can be neither ferromagnetic nor antiferromagnetic\cite{PhysRevLett.17.1133}.

\section{CDW Configuration}
% \labsec{sec:CDWConfiguration}
Here're some expressions that describes CDW of TMDCs.

\section{Other}
% \labsec{sec:Other}
Particularly attractive materials are the transition-metal dichalcogenides (TMDs), due to their van der Waals layered structures and their wide range of material properties. Controlling the material thickness down to the single-atom scale enables tuning of the interacting electronic states and phases, and it is thus highly desirable to fabricate high-quality monolayer (ML) TMDs\cite{doi:10.1021/acsnano.9b02996}(Introduction).

\chapter{VS$_2$}
% \labch{ch:VS2}

\section{Ground State Determination}
% \labsec{sec:GroundStateDetermination}
\subsection{Experiment Facts}
Here're some expressions that describes goundstate of TMDCs.

\subsection{DFT Calculation}
Here're some expressions that describes goundstate of TMDCs.

\section{Magnetic Properties}
% \labsec{sec:MagneticProperties}
Here're some expressions that describes Magnetic Properties of TMDCs.

\section{CDW Configuration}
% \labsec{sec:CDWConfiguration}
Here're some expressions that describes CDW of TMDCs.

\section{Other}
% \labsec{sec:Other}
Here're some expressions that describes Other Physics of TMDCs.

\chapter{VSe$_2$}
% \labch{ch:VS2}

\section{Ground State Determination}
% \labsec{sec:GroundStateDetermination}
\subsection{Experiment Facts}
VSe2 is the only TMDC to exhibit such characteristics experimentally on the monolayer scale. The peculiar physical properties of VSe2 such as high electrical conductivity, chargedensity-wave (CDW) transition, the Kondo effect, and the weak localization e!ect were revealed by electrical transport measurements only recently(Introduction). 
\subsubsection{Bulk}
The crystal structure of VSe2 was confirmed to be a hexagonal phase with lattice parameters of a = b = 3.3587 Å and c = 6.1075 Å (space group P3!m1, JCPDS card no. 89-1641)\cite{C8NR09258C}.
\subsubsection{Monolayer}

\subsubsection{Nanosheets}
We obtained the in-plane and out-of-plane lattice constants of 3.26 ± 0.2 Å and 6.27 ± 0.3 Å. The structural phase of the as-grown VSe2 nanosheets was 1T\cite{C8NR09258C}.
\subsection{DFT Calculation}
VSe2 is one of the few TMDCs that have been theoretically predicted to show ferromagnetic characteristics\cite{C8NR09258C}(Introduction). Fumega and Pardo reported that DFT calculations yield a ferromagnetic ground state, which is on the verge of instability. They demonstrated that structural rearrangement due to the charge density wave causes an energy gap opening at the Fermi level and, in turn, the quenching of ferromagnetism in the ML limit, offering a clue to explain the controversy\cite{doi:10.1021/acsnano.9b02996}. 
\subsubsection{Bulk}

\subsubsection{Monolayer}
Our calculations demonstrate that the transition from the experimentally observed T configuration to the H configuration is accompanied by a considerable change in the electronic structure, which is a redistribution of 3d electrons of vanadium between orbitals. On the other hand, the values of the magnetic moments and total energies of the ferro- and antiferromagnetic configurations change gradually between the two structural phases. The analysis of the calculated phonon dispersions has demonstrated a principal role of ferromagnetism in stabilization of the atomic structure of the VSe2 monolayer in the H phase and similar systems \cite{C9CP03726H}(Conclusion).

\subsubsection{Nanosheets}
The calculations for bi- and trilayers demonstrate that the energy barrier of the transition is similar to the monolayer. Strain, possibly induced by a substrate, provides the change of the most energetically favorable structure from H to T. Therefore, the experimental observation of the T configuration can result from the VSe2 structure stretching by more than 3 percent on substrates \cite{C9CP03726H}.
\section{Magnetic Properties}
% \labsec{sec:MagneticProperties}
\subsection{Experiment}
Our measurements indicate no ferromagnetism down to 10 K. Instead, they reveal that a charge-density wave that gaps the entire high-temperature Fermi surface is the dominant instability at low temperatures in monolayer VSe2 \cite{doi:10.1021/acs.nanolett.8b01649}. Bonilla et al. indicated that a ferromagnetic state persists up to room temperature in ML VSe2, but their report has several critical unexplained features. Their most debatable finding is the reported magnetization of $15 \mu_{B}$ per V atom, which disagrees with the density functional heory (DFT) calculations for ML VSe2. On the other hand, Feng et al.\cite{doi:10.1021/acs.nanolett.8b01649} showed using X-ray magnetic circular dichroism (XMCD) that there is a zero magnetic moment on the V atoms down to 10 K in an applied magnetic field of 9 T(Introduction). Our work provides clear experimental evidence that ML VSe2, without the signature of ferromagnetism, is on the verge of magnetic transition\cite{doi:10.1021/acsnano.9b02996}(Conclusion).

ML VSe2 has attracted special attention from the scientific community due to several recent discoveries, including in-plane piezoelectricity, a pseudogap with a Fermi arc at temperatures above the charge density wave transition (220 K for the monolayer), and especially the existence of ferromagnetism in the 2D system. Experimental results are rather contradictory. Strong room-temperature ferromagnetism with a huge magnetic moment per formula unit has been reported for monolayer VSe2 epitaxially grown on graphite. A local magnetic phase contrast has also been observed by magnetic force microscopy at room temperature at the edges of VSe2 flakes exfoliated from a threedimensional crystal. XMCD measurements evidence a spinfrustrated magnetic structure in VSe2 on graphite. Paramagnetism of bulk VSe2 makes these observations more intriguing. Another situation was reported for monolayers grown on bilayer graphene/silicon carbide substrates. In both studies the absence of exchange splitting of the vanadium 3d bands observed in angle-resolved photoemission spectroscopy experiments was reported \cite{C9CP03726H}(Introduction).
\subsection{Calculation}
It has been proposed that the presence of charge density waves could cause the quenching of monolayer ferromagnetism due to the band gap opening induced by Peierls distortion. Phonon spectra of several VSe2 and similar systems were also considered theoretically \cite{C9CP03726H}(Introduction).
\section{CDW Configuration}
% \labsec{sec:CDWConfiguration}
\subsection{Bulk}
VSe2 bulk is known to host a CDW in the bulk with an onset temperature of $T_c \approx 110 K$. $4 \times 4 \times 3$ CDW reconstruction of the bulk has a significant component in the
out-of-plane direction, which has been attributed to a nesting of the strongly three-dimensional Fermi surface at the corresponding wave vector\cite{doi:10.1021/acs.nanolett.8b01649}(Introduction).

\subsection{Monolayer}
The pronounced energy gaps develop over the entire Fermi surface with decreasing temperature below $T_c = 140 \pm 5$ K\cite{doi:10.1021/acs.nanolett.8b01649}.
\section{Other}
% \labsec{sec:Other}
Here're some expressions that describes Other Physics of TMDCs.

\chapter{VTe$_2$}
% \labch{ch:VS2}

\section{Ground State Determination}
% \labsec{sec:GroundStateDetermination}
\subsection{Experiment Facts}
Here're some expressions that describes goundstate of TMDCs.

\subsection{DFT Calculation}
Here're some expressions that describes goundstate of TMDCs.

\section{Magnetic Properties}
% \labsec{sec:MagneticProperties}
Here're some expressions that describes Magnetic Properties of TMDCs.

\section{CDW Configuration}
% \labsec{sec:CDWConfiguration}
Here're some expressions that describes CDW of TMDCs.

\section{Other}
% \labsec{sec:Other}
Here're some expressions that describes Other Physics of TMDCs.