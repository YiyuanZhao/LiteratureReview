\setchapterpreamble[u]{\margintoc}
\chapter{Atomistic real-space observation of the van der Waals layered structure and tailored morphology in VSe2.\cite{C8NR09258C}}
% \labch{ch:SampleTitle}

\section{Summary Table}
% \labsec{sec:SummaryTable}

\begin{table}[h]
    \begin{tabular}{lccc}
    \hline
    Item  & Material         & Classification & Topic        \\  \hline
    Value & VSe2             & Experiment     & Structure+Magmom    \\  \hline
    Item  & Range            & System         & Publish Year \\  \hline
    Value & Structure        & Nanosheets     & 2019         \\  \hline
    \end{tabular}
\end{table}

\section{Abstract}
Two-dimensional (2D) transition metal dichalcogenides with van der Waals gaps have attracted much attention due to their peculiarly distinctive physical properties from their bulk counterparts. Among them, vanadium diselenide (VSe2) has been considered to be a promising candidate for future spintronic devices, as room temperature ferromagnetism was reported recently. However, detailed crystallography and properties of VSe2 nanosheets have been less explored. Here, we report the atomistic real-space observation of the van der Waals layered structure of VSe2 for the first time. Furthermore, simply by controlling the carrier gas flow rate, a morphological variation of the surface area and thickness of VSe2 nanosheets was observed. The room temperature ferromagnetic feature of single VSe2 nanosheets was also revealed by magnetic force microscopy measurements. Our fndings will play a significant role in the research of intrinsic 2D ferromagnetic materials.

\section{Background}
% \labsec{sec:Background}
To progress further and explore the wide electrical and magnetic diversities of VSe2, the control over the sample preparation and the corresponding physical properties is necessary.

\section{Methodology}
% \labsec{sec:Methodology}

\subsection{Experiment}
Single crystalline VSe2 nanosheets is synthesized by chemical vapor deposition (CVD). Illustration of experiment shows in \reffig{fig:10.1039.c8nr09258c_Schematic}.

\begin{figure}[ht] 
    \includegraphics[width=0.8\linewidth]{./images/10.1039.c8nr09258c_Schematic}
	\caption[Schematic of VSe2 nanosheet growth]{
		Schematic of VSe2 nanosheet growth.
	}
	\labfig{fig:10.1039.c8nr09258c_Schematic}
\end{figure}

% \subsection{DFT}
% Software tools: Wien2k.
% \begin{table}[h]
%     \begin{tabular}{cccc}
%     \toprule
%     KPOINTS                 & ENCUT  & Pseudopotential & Hubbard-U \\
%     \midrule
%     $14 \times 14 \times 2$ & Unknown& PBE             & Unknown   \\
%     \bottomrule
%     \end{tabular}
% \end{table}
% Other parameter: Vacuum thickness 14.85 A.
\section{Conclusion}
% \labsec{sec:Conclusion}
By an X-ray di!raction analysis, the crystal structure of VSe2 was confirmed to be a hexagonal phase with lattice parameters of a = b = 3.3587 Å and c = 6.1075 Å (space group P3!m1, JCPDS card no. 89-1641). In particular, the angle between Se–V–Se bonds was estimated to be 97°, confirming that the structural phase of the as-grown VSe2 nanosheets was 1T.
\begin{figure}[ht] 
    \includegraphics[width=0.8\linewidth]{./images/10.1039.c8nr09258c_TEMImageOfVSe2Nanosheet}
	\caption[TEM image of a half-hexagonal VSe2 nanosheet on a Cu grid]{
		TEM image of a half-hexagonal VSe2 nanosheet on a Cu grid.
	}
	\labfig{fig:10.1039.c8nr09258c_TEMImageOfVSe2Nanosheet}
\end{figure}

The MFM is a reliable technique that can observe the magnetic regions of nanoscale magnetic materials. The MFM image of the VSe2 nanosheet with 30 nm thickness and a flat surface is shown brightly around the edge (See \reffig{fig:10.1039.c8nr09258c_TEMImageOfVSe2Nanosheet}), indicating the presence of magnetism in the nanosheet.

Here, we report the \textbf{atomistic real-space observation of the van der Waals layered structure of VSe2} for the first time. Furthermore, simply by controlling the carrier gas flow rate, a morphological variation of the surface area and thickness of VSe2 nanosheets was observed. The \textbf{room temperature ferromagnetic} feature of single VSe2 nanosheets was also revealed by magnetic force microscopy measurements. By \textbf{controlling the carrier gas flow rate} during the synthetic reaction, we observe the \textbf{morphological variation} of the surface area and thickness of the VSe2 nanosheets.

\section{Defect}
% \labsec{sec:Defect}

\begin{enumerate}
    \item Using CVD methods, which could not gain monolayer usually(Nanosheets at this time).
    \item The paper does not gives the magnetic moment of the nanosheets.
\end{enumerate}

\section{Notes}
% % \labsec{sec:Notes}
VSe2 nanosheets(~ 40 nm) are magnetic. However the temperature is unclear. 1T goundstate for nanosheets.