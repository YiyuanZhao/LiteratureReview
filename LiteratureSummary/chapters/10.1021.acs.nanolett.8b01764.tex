\setchapterpreamble[u]{\margintoc}
\chapter{Sample Title}
% \labch{ch:SampleTitle}

\section{Summary Table}
% \labsec{sec:SummaryTable}

\begin{table}[h]
    \begin{tabular}{lccc}
    \hline
    Item  & Material         & Classification & Topic        \\  \hline
    Value & VSe2             & Experiment     & Magnetic+CDW \\  \hline
    Item  & Range            & System         & Publish Year \\  \hline
    Value & Structure        & Monolayer      & 2018         \\  \hline
    \end{tabular}
\end{table}

\section{Background}
% \labsec{sec:Background}
This article focus on the unsolved problem that \dots

\section{Methodology}
% \labsec{sec:Methodology}

\subsection{Experiment}
Substration: Bilayer graphene.

\subsection{DFT}
\begin{table}[h]
    \begin{tabular}{ccccc}
    \toprule
    KPOINTS                 & ENCUT  & Pseudopotential & Hubbard-U & Conver.      \\
    \midrule
    $14 \times 14 \times 2$ & Unknown& PBE             & Unknown   & $10^{-6}$ eV \\
    \bottomrule
    \end{tabular}
\end{table}

\section{Conclusion}
% \labsec{sec:Conclusion}
External pressure has great impact on the ground state of TMDCs.

\begin{marginfigure}
    \includegraphics{./images/10.1021.acsnano.9b02996_XMCDHysteresisLoops}
	\caption[XMCD hysteresis loops obtained at Co and V $L_3$ edge at 65 K]{
        XMCD hysteresis loops obtained at Co and V $L_3$ edge at 65 K, which confirms the antiparallel alignment, i.e., antiferromagnetic coupling, between Co and V spins.
	}
	\labfig{10.1021.acsnano.9b02996_XMCDHysteresisLoops}
\end{marginfigure}

\begin{figure}[ht] 
    \includegraphics[width=0.8\linewidth]{./images/10.1039.c8nr09258c_TEMImageOfVSe2Nanosheet}
	\caption[TEM image of a half-hexagonal VSe2 nanosheet on a Cu grid]{
		TEM image of a half-hexagonal VSe2 nanosheet on a Cu grid.
	}
	\labfig{fig:10.1039.c8nr09258c_TEMImageOfVSe2Nanosheet}
\end{figure}

\section{Defect}
% \labsec{sec:Defect}
The measured data in this paper is contrast with the other paper. Mean field approximation may not be adequate to describe interactions.
\begin{enumerate}
    \item item1.
    \item item2.
\end{enumerate}

\section{Notes}
% \labsec{sec:Notes}
We have some great notes here.