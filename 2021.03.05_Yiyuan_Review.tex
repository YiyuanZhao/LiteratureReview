\documentclass[reprint, aps, prb, showkeys]{revtex4-2}

\usepackage{graphicx}% Include figure files
\usepackage{dcolumn}% Align table columns on decimal point
\usepackage{bm}% bold math
\usepackage{ctex}
\usepackage{amsmath}
\usepackage[colorlinks, linkcolor=blue]{hyperref}

\begin{document}

\title{Report: Exotic Dielectric Behaviors Induced by Pseudo-Spin Texture in \\
Magnetic Twisted Bilayer}

\author{Yiyuan Zhao}
\affiliation{Department of Physics, Tongji University, Shanghai, 200092 P. R. China}
\date{\today}

\begin{abstract}
扭转van der Waals双层给固体电子关联的研究提供了理想平台。对于30°蜂巢状晶格系统,由于具有不匹配的摩尔条纹,由于相关联缺失造成的新奇电子行为,受到了广泛的关注。基于第一性原理计算,我们给出在接近30°的旋转角时,可以在左旋/右旋的不同情况给出显著不同的电子电介质行为的差异。进一步研究给出此类有趣的电介质性质可以归结于与堆叠结构有关的、扭转引起的电荷重新分布,这组成了扭转依赖的赝自旋图样。我们给出这类赝自旋图案在小电场下也十分剧烈。作为结果,对于右旋的情况,当施加电场时,几乎没有超过单层厚度的电场偶极矩的形成。然而对于左旋的情况,系统甚至可以出现负极化率。也就是说,引入的偶极矩与施加的电场方向相反,这种现象在自然中十分罕见。
\begin{description}
    \item[DOI] \url{http://cpl.iphy.ac.cn/10.1088/0256-307X/38/3/037501}
\end{description}
\end{abstract}

\keywords{negative susceptibility, pseudospin}

\maketitle

\section{Introduction}
基于vdW晶体的双层扭转结构在理论和实验领域具有重要的意义,近期原子层厚度的制备进展成功实现了层间的扭转。在这些二维系统中,具有长周期的莫尔图样可以由误导(misoriented)的堆叠方式所引入。在vdW晶体中层间耦合图样显著影响了低能的能带结构。例如,以平带为特征的双层扭转石墨烯(tBLG),Dirac速度在扭转角为“魔幻角”时会变成零。tBLG中的强电子关联效应服从多种有趣的物理行为,例如从半金属到Mott绝缘体的转变,甚至包括非常规超导。

\begin{figure*}[t]
    \includegraphics[width=0.70\textwidth]{./img/20210305/1}
    \caption{\label{fig:structure} 
    2H-VSe$_2$左旋的匹配结构(左图)、右旋结构(右图)和30°扭转的不匹配结构(中图)。草绿色和红色的蜂巢结构代表底层的Se和V原子。黄色和蓝色六边形代表顶层的Se和V原子。(b)超胞的第一布里渊区(BZ)由灰色的六边形表示。
    }
\end{figure*}

与魔角双层扭转石墨烯形成相同大小的超胞情况不同,当两层之间的扭转角为30度附近时,奇点性质显现,使得系统缺乏长周期性,变得不匹配,具有12重简并的旋转对称性。更有趣的是,对于双层过渡族金属二硫化物(TMD) 2H-MX$_2$(M = Mo, W且X = S, Se),由于不哦她那个堆叠方式的存在,在接近30°扭转角左右对应不同的原子结构。结果上,产生的两种摩尔系统在超胞中拥有电子波函数的不同相调制。扮演不同依赖堆叠的性质在tBLG系统中并不存在。例如,交换相互作用的调制在两种扭转情况下形成了不同的图样。更进一步的,由于二维系统中,磁矩对结构变化极端敏感,我们希望具有内禀交换耦合的非自旋简并扭转双层系统可以证明更多的有趣电子关联现象。这类考虑因此激励我们将摩尔TMD双层延展到磁性系统。

在这项工作中,基于典型的此行系统2H-VSe$_2$,我们证明,当施加竖直电场来打破层与层之间的势场平衡时,将在扭转角接近30°时发生新奇的电介质现象。对于右旋的情况,当施加电场时,几乎没有超过单层厚度的电场偶极矩的形成。然而对于左旋的情况,系统甚至可以出现负极化率。进一步研究证明了特殊的电介质响应的深层起源在于,较小的电场几乎不能改变这两种摩尔系统的赝自旋图景。

\section{Results}
\begin{figure*}[t]
    \includegraphics[width=0.70\textwidth]{./img/20210305/2}
    \caption{\label{fig:band} 
    使用SOC计算的32.2°(a-c)和27.8°(d-f)情况。对于K谷抛物线型能带的描述,我们使用点划线和实线来表示顶层和底层,自旋向上(红色)和自旋向下(蓝色)区分。
    }
\end{figure*}

我们选择铁磁2H-VSe$_2$,典型的2D铁谷材料,将其双层设置为反铁磁耦合,此为绝大多数三斜双层系统的基态。使用这种方式,这种反铁谷双层系统同时具有自旋-谷锁定和自旋-层锁定的性质。使用$30 \pm 2.2°$的扭转情况作为例子,两层之间的面内滑移被设置为零。不匹配(无限大)和匹配(有限大)的晶格结构如图\ref{fig:structure}(a)所示,此处两种匹配的部分具有相同的尺寸[图\ref{fig:structure}(b)]。在此处我们定义$\delta < 0 (>0)$的情况对应上层相对于下层右旋/左旋$30 + \delta °$。注意左右旋的双层系统对应于不同单层能带的叠加。这可以通过超胞沿着$\Gamma - K$轴的两个第一布里渊区的镜像对称来理解,如图\ref{fig:structure}(b)所示。考虑折叠到第一布里渊区的能带。我们得到对于左旋的超胞,$K_+$态由底层的$K_+$谷态和顶层的$K_-$谷叠加而成(H-型堆叠)。而对于右旋的超胞,则有两层的$K_+$谷态叠加而成(R-型堆叠)。

\begin{figure*}[t]
    \includegraphics[width=0.70\textwidth]{./img/20210305/3}
    \caption{\label{fig:polarization} 
    (a)$\Delta P(E)$计算值随着电场强度E的演化;(b)引入的底层在0.001V/A的电场中的平面电荷分布,左旋情况(左图)和右旋情况(右图)。引入电场的电荷密度$\Delta\rho = \rho(E) - \rho(0)$;黄色和青色代表电荷的积累和耗尽。橘色箭头标记了电场的方向。(c)引入的平均电子电荷,沿着z轴的$\Delta\rho = \rho(R) - \rho(L)$,当双层系统从右旋转到左旋。
    }
\end{figure*}

对于这两种情况,我们通过施加竖直电场研究了两种系统的电介质响应。在这里,正场定义为上层指向下层。有趣的是,尽管两种系统仅仅相差不到5°的旋转角,我们发现了其在外场下具有相当强烈的外场相应,在外电场下,能带的显著不同位移可以解释这种响应。如图\ref{fig:band}所示,当E = 0.02V/A时,右旋情况在$\Gamma$点处具有大约0.1 eV的价带最高点的分裂。是左旋情况的10倍(~0.01 eV)。对于左旋情况,能带对于外加电场不敏感,即抵抗外电场。对于右旋情况,存在增幅效应,即给出一个小的外电场,可以引入金属-绝缘体相变,远远比H-型双层情况小。

除此以外,使用第一性原理计算的Berry phase算法给出的电子极化$\Delta P(E)$如图\ref{fig:polarization}(a)所示。以电场强度$E = 0.001 V/A$为例,计算结果表明对于左旋和右旋,极化率分别为$\Delta P = -0.98$和$\Delta P = 1.26$。在此处,我们定义P为每层的单位体积偶极矩,单位为$10^{-4} e/A^2$。与单层系统的$\Delta P = 1.03$相比,扭转双层明显具有被压制/被强化的电介质极化。双层系统非正常电介质响应在电子云中可以看的更清晰,如图\ref{fig:polarization}(b)所示,对于左旋情况,引入了负极化率,即引入的偶极矩与施加的电场相反,与右旋的情况相反,几乎不引入超越单层厚度的偶极子。实际上,这种响应十分新奇。由于电介质效应,引入的电偶极矩通常都沿着外加电场的方向,这种效应对电介质和磁性金属薄层都有效。

新奇的电介质差异可以通过不考虑电场的电荷初始分布来实现,沿着z轴的电荷分布不均$\Delta\rho = \rho_R - \rho_L$。如图\ref{fig:polarization}所示,当双层系统从左旋向右旋转化时,存在层间引入的净电荷密度的增加或减少。对于类似反铁磁耦合的双层铁磁,额外从30°开始的$\delta$扭转将同时调制面内和面间的电子云分布。除此之外,例如电场效应,层内磁交换耦合的增加进一步将顶部和底部层的能带分裂。结果上看,电场下的能带分裂是二重的,可以被看作是电场增幅效应。

\begin{figure}[b]
    \includegraphics[width=0.40\textwidth]{./img/20210305/4}
    \caption{\label{fig:pseudospin} 
    (a)超胞第一布里渊区中价带顶点归一化的赝自旋图案,包含了上层和下层的自旋贡献。平面内和平面外成分由红色箭头和颜色图表示。(b)Rashba SOC效应在我们例子中的示意图。
    }
\end{figure}

然而,对于H-类型的堆叠方式,施加的电场倾向于形成不同手性漩涡状的赝自旋,这种现象将会摧毁最初的赝自旋图样。作为拓扑保护的结果,平面电子将会集体的移动到纸平面外或者抵抗外电场。因此形成了负极化强度。如图\ref{fig:pseudospin}(c)所示,这种机制在非微扰态保持系统尽可能的互相接近。左旋的负电子极化率并不常见,经常在人工设计的美特材料中才会出现。我们假设这种新型性质会导致光学波导中的广泛应用。我们同样发现在扭转的情况下,静电能将被转换成静磁能。因此,这类系统具有大的储能潜力。

\end{document}

