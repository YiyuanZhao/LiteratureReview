\documentclass[reprint, aps, prb, showkeys]{revtex4-2}

\usepackage{graphicx}% Include figure files
\usepackage{dcolumn}% Align table columns on decimal point
\usepackage{bm}% bold math
\usepackage{ctex}
\usepackage{amsmath}
\usepackage[colorlinks, linkcolor=blue]{hyperref}

\begin{document}

\title{VSe$_2$ DFT paper list}

\author{Yiyuan Zhao}
\affiliation{Department of Physics, Tongji University, Shanghai, 200092 P. R. China}
\date{\today}

\begin{abstract}
以下清单给出了目前DFT计算给出的基本结果,理论与实验结合的paper主要描述了第一性原理计算的部分,纯实验的paper不包含在本清单内。
\end{abstract}

\keywords{DFT VSe$_2$}

\maketitle

\section{Paper List}

有研究\cite{doi:10.1063/1.5092846}给出了VSe$_2$中随着压力变化而导致的各向异性。Zishen Wang et al.\cite{wang2020controllable}通过第一性原理计算指出在电荷dopping和单轴形变中,二维VSe$_2$存在$\sqrt{7} \times \sqrt{3}$、$4 \times 4$、$3 \times \sqrt{3}$等CDW基态。Adolfo O. Fumega et al.\cite{doi:10.1021/acs.jpcc.9b08868}根据XMCD和第一性原理计算,利用Stoner判据提出由于CDW的作用,VSe$_2$中会出现铁磁性的消失。Guannan Chen et al.\cite{PhysRevB.102.115149}在200和450摄氏度情况下生长了1T薄层,观察到了在薄层中出现的$4a \times 4a$的CDW,并通过计算证明一种在特定波矢下出现的软声子模式导致了CDW的出现。Matthew J et al.\cite{trott2020fermi}通过重整化群的方式建立了Fermi Nesting造成CDW出现的模型,模型虽然不能解释哪个Q矢量形成了CDW、超导序参量的相对相位,但是仍然可以形成CDW和超导序。J. G. Si et al.\cite{PhysRevB.101.235405}提出Fermi Surface Nesting和电声耦合对$4 \times 4 \times 3$的CDW形成有贡献,而与动量有关的电声耦合则是形成单层1T-VSe$_2$中$\sqrt{7} \times \sqrt{3}$CDW的主要原因。

Junyi He et al.\cite{he2020confinement}根据第一性原理计算提出空间限域效应(confiment effect)导致了内在的Stoner不稳定性,消除层间耦合,除此以外VHS的出现导致了DOS的急剧增大,从而导致了磁性的出现和消失。Marco Esters et al.\cite{PhysRevB.96.235147}给出了电子关联效应对单层/双层VSe$_2$磁性、动力学稳定性的巨大影响,增加Hubbard U会导致1T的磁各向异性,但对2H结构几乎没有影响, FS-nesting对CDW vector的出现没有贡献,但软声子模式的出现会导致$4 \times 4$CDW的出现。Wei Yu et al.\cite{https://doi.org/10.1002/adma.201903779}给出了化学剥离单层VSe$_2$的方法,并通过DFT计算给出Se空位会导致磁性的出现。Taek Jung Kim et al.\cite{Kim_2020}通过DMFT的计算,给出了体块无磁性,单层有磁性的基态,且单层的铁磁序对于电子掺杂格外敏感,电子掺杂会导致磁性的猝灭。Nijing Guo et al.\cite{GUO2020109540}对于V族单层TMDs进行了第一性原理计算,得到除了1H-VSe$_2$和1H-VTe$_2$以外,所有1H/1T相的TMDs都是金属。H. M. R. Ahamd et al.\cite{doi:10.1063/1.5139061}通过在两层之间插层轻的金属原子给出了调控双层V族二硫化物磁性从反铁磁变为铁磁的方法。A. H. M. Abdul Wasey et al.\cite{doi:10.1063/1.4908114}给出层状VX$_2$(X = S, Se)的金属通过量子尺寸效应转变为半金属(H-VS$_2$)和半导体(H-VSe$_2$)的相变。Fengyu Li et al.\cite{doi:10.1021/jp507093t}计算了体块、层状、单层、纳米带、纳米管状态下VSe$_2$的金属性。Yandong Ma et al. \cite{doi:10.1021/nn204667z}通过DFT计算研究了VX$_2$(X = S, Se)的电子性质,给出H相单层VX$_2$具有磁性,且磁性大小随着-5\%-5\%的应力变化敏感。

Georgy V. Pushkarev et al.\cite{C9CP03726H}建立了T相和H相可能的两种过渡态,并给出了相应的energy barrier,声子计算给出了单层/体块H-VSe$_2$相中虚频的存在。Liu Jian\cite{cnki1}给出了VSe$_2$的谷极化现象。Lei Bao\cite{cnki2}建立了VSe$_2$和V$_4$Se$_7$的模型,给出的V$_4$Se$_7$形成能更低,且Se原子空位处会在STM模拟中出现暗条纹。Hui Pan\cite{doi:10.1021/jp503030b}使用第一性原理计算得到V单层X$_2$(X = S, Se, Te)可以通过氢化的方法从半金属/铁磁变为非磁/反铁磁。
\bibliography{VSe2_DFT_List_Ref}
\end{document}