\documentclass[reprint, aps, prb, showkeys]{revtex4-2}

\usepackage{graphicx}% Include figure files
\usepackage{dcolumn}% Align table columns on decimal point
\usepackage{bm}% bold math
\usepackage{ctex}
\usepackage{amsmath}
\usepackage[colorlinks, linkcolor=blue]{hyperref}

\begin{document}

\title{Report: Monolayer Modification of VTe2 and Its Charge Density Wave}

\author{Yiyuan Zhao}
\affiliation{Department of Physics, Tongji University, Shanghai, 200092 P. R. China}
\date{\today}

\begin{abstract}
过渡族金属二硫化物的层间相互作用在描述其电子性质时非常重要。在此,我们证明了单层VTe$_2$层间耦合的缺失也导致了其从体块/多层$1T^{'}$构型变为单层时的六角1T构型。X射线光电子谱指出,其结构相变与V原子d轨道的电子向单层的Te原子转移导致的。这种电荷转移可能产生面内d轨道杂化,因此体系更倾向于形成1T结构。光子色散计算给出,与$1T^{'}$结构相反,1T结构显示出虚声子模式,导致了CDW不稳定性的出现,这也由低温扫描隧道显微镜观察到的$4 \times 4$周期晶格畸变所证明,这种性质可以通过层间相互作用调控。

\begin{description}
    \item[DOI] \url{https://doi.org/10.1021/acs.jpclett.9b01949}
\end{description}
\end{abstract}

\keywords{Substration effect}

\maketitle

\section{Introduction}
\begin{figure}[t]
    \includegraphics[width=0.4\textwidth]{./img/20210709/1}
    \caption{\label{fig:STM} 
    Characterization of multilayer MBE-grown VTe2.
    }
\end{figure}
通过层间相互作用修饰vdW材料的性质是一种设计新材料和器件的手段。这种方法的有效性通过在TMDCs中的相互作用依赖的性质得到补充。材料性质修饰的例子可以通过层间相互作用的(1)VI族过渡族金属在层间相互作用被移除时,非直接到直接的带隙转变;(2)由于层间杂化导致的强带隙随着过渡族金属层数变化的调控性;(3)在TiSe2和TiTe2中的电荷密度波变化的调控性;(4)与多层相比,单层TaSe2中更大的Mott绝缘体态和独特的轨道图样。在此处,我们第一次在VTe2中研究了层数依赖的TMDCs。我们还将其在单层极限下,转换为简单1T结构。层间相互作用的缺乏也产生了新的CDW材料。之前对于单层的研究主要作用于六角、高对称1T和2H晶相的TMDCs。相反,V族Te的畸变$1T^{''}$结构可能可以被过渡族金属双zigzag行描述,具有多种人们熟知的单zigzag行$1T^{'}$结构。这三种相互关联的结构都可以在下方被找到。这些畸变结构的共同点在于,过渡族金属元素(TMs)的键长都因为原子间成键而减小。丝带状结构可以被描述为$3 \times 1$重组合,zigzag结构具有六角形网格的$2 \times 1$重组。无论是卷形还是zigzag形的畸变1T结构,都被人们解释为从面内金属坐标d轨道之间的杂化。$t_{2g}$轨道在$1T^{'}$和$1T^{''}$中被晶格畸变劈裂分开,导致了$d_{xz}, d_{xy}$成键轨道能量的降低。对于每个成键轨道都被两个电子占据的情况下,这种情形有可能发生。在ribbon结构中,成键轨道通过三个TMs被共享,因此每个TM原子得到4/3个d电子。相反,zigzag类型的被两个TM共享,因此每两个原子得到了一个电子。

这些讨论给出,结构可以通过电荷的得失来调控。V族Te化合物的层间相互作用已经被提出确定d轨道占据方式格外重要。理论计算证明,Te-Te层间耦合在费米面之上的Te 5p轨道出现,因此导致了从Te占据态到V 3d轨道的电荷转移。结果上,层间耦合控制了d电子数,因此,结构控制可以通过修饰层间相互作用来实现。在这里我们给出在单层结构中,层间耦合的缺失导致了结构畸变的出现和简单1T结构的出现。在这种新型1T构型的VTe2中,我们观察到了单层结构特有的$4 \times 4$CDW。CDW相变的层数依赖关系是理解TMDCs中CDW起源的中心观点。更一般地说,在这些研究中,从体块到单层的CDW性质旨在一些不同转变温度下被观察到。CDW与层数依赖的结构转变的耦合将VTe2与其他TMDCs分离开来,使得CDW行为被限制在单层极限下。
\begin{figure*}[t]
    \includegraphics[width=0.8\textwidth]{./img/20210709/2}
    \caption{\label{fig:Characterization} 
    Characterization of monolayer VTe2
    }
\end{figure*}

图\ref{fig:CDW}给出了多层样品(大约2-3层)的STM和低能电子衍射图样。STM图样揭示了这些多层样品暴露区具有不同的层数,给出了阶梯状的表面形貌。在更高分辨率的图像下,结构区域观察到三个等效的旋转区域。原子分辨率图样给出了结构是简单六角晶格。三种旋转区域具有$2 \times 1$的周期性,与体块中报道的$3 \times 1$周期性不同。与这些结构的ribbon构型一致。观察到的$2 \times 1$结构与zigzag $1T^{'}$结构一致,因此给出MBE生长的多层VTe2显示出与体块样品不同的结构。相同的结构也在HOPE,石墨烯/SiC,MoS2上被发现,因此基底对结构的影响不敏感。

如果薄层厚度降低到单层,结构修饰就会发生。图\ref{fig:Characterization}给出了近乎完全单层结构的STM图样。原子分辨率的图像给出结构是简单六角,与1T结构一致。LEED图样也显示出六角图样。除了源自于VTe2单层的LEED图样,MoS2基底的衍射也被人们所分辨。使用已知MoS2的晶格常数$a_{MoS2} = 0.316$ nm,我们测量VTe2的晶格常数为0.36 nm,与对应体块的晶格参数相近,预示着VTe2通过vdW在基底上外延生长,没有受到压力。为了确认1T结构,我们进行了APRES测量,将沿着$\Gamma \rightarrow K \rightarrow M$路径测量,并将结果与DFT进行比较。尽管石墨烯是单晶相的,生长的VTe2不是完美外延。尽管如此,倾向的对齐足够得到良好质量的APRES数据,与其他vdW系统的结果一致。测量的能带结构与DFT一致。接近$\Gamma$点,强烈的源自于V原子的平带被观察到,接近费米面。在$\Gamma - K$方向的一半处超过费米面。两个源自于Te 5p的空穴口袋在$\Gamma$附近接近费米面,在$\Gamma - K$方向向下色散。在M点,V 3d能带形成了电子口袋。

\begin{figure}[t]
    \includegraphics[width=0.4\textwidth]{./img/20210709/3}
    \caption{\label{fig:Phonon} 
    Charge density wave instability in monolayer 1T VTe2.
    }
\end{figure}

为了进一步研究VTe2这些相的结构稳定性,我们通过DFT,使用LDA进行了声子色散的计算。图\ref{fig:Phonon}(a, b)给出1T$^{'}$相是稳定的,而1T相则表现出负频率,意味着虚频率的出现。这样的声子不稳定预示着1T相具有CDW相变。负数模式出现在$\Gamma - M$的一半区域。使用折叠$4 \times 4$的方式,我们可以给出所有折叠布里渊区内的虚频率模式。CDW不稳定性通过STM在20K得到确认,给出了清晰的$4 \times 4$超结构。
\end{document}
